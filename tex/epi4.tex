% EPI4.TEX (RMCG19980422)

\prolog Pilares para una ética subjetivista (ÉTICA)

\chapter La epistemología

La epistemología estudia los fundamentos del saber. No explica ningún saber
específico, sino el propio saber, de modo que es un saber sobre el saber.
Determina, en primer lugar, si el saber es posible. Así, las escuelas
escépticas, que predican la imposibilidad del saber, no necesitan decir nada
más, y aun ni esto si son completamente coherentes. Para los no escépticos
queda la tarea de mostrar cuál es la esencia del saber y, en función de
ésta, cómo puede adquirirse y con qué grado de confianza.

\chapter El sujeto y el objeto

La epistemología distingue sujeto y objeto. El objeto es lo sabido y el
sujeto es quien lo sabe. Cuál sea la naturaleza de sujeto y objeto y cómo
sean las relaciones entre ellos, define las diferentes epistemologías. No
pasa inadvertida la similitud de la estructura `sujeto verbo objeto', de
frases como `papá come pan', con esta estructura de explicación final, que
tiene la forma `sujeto sabe objeto'. El orden de los elementos, que depende
de la sintaxis del idioma, no es importante.

\chapter El subjetivismo y el objetivismo

Siempre podemos dividir un tema en dos, basta elegir cualquier propiedad y
tratar la parte que cumple la propiedad por un lado y, separadamente, la
parte que no la cumple. En el caso que nos ocupa, y desde
\person{Descartes}, que explicó que el yo es la única fuente de certeza, se
hace imposible prescindir del sujeto. La duda se cierne sobre el objeto, de
manera que clasificaremos las diferentes epistemologías en dos grupos: las
subjetivistas que postulan que el objeto está condicionado al sujeto, y las
objetivistas que sostienen lo contrario.

\goodpage

\chapter La diferencia

La posición objetiva postula que la piedra que vemos existe por sí misma,
independientemente de todo. La posición subjetiva condiciona la existencia
de la piedra al sujeto que la percibe. La diferencia entre ambas posiciones
es, en la práctica y para el sujeto, nula, porque la condición se cumple
siempre. Dicho de otro modo, los sujetos sólo podemos estudiar las
diferencias teóricas entre el subjetivismo y el objetivismo. La posición
subjetivista es más segura, ya que como sujeto que soy únicamente puedo
verificar que existe la piedra si yo soy. Dejar de ser, crudamente morir,
para comprobar si la piedra sigue existiendo, o no, es un experimento no
recomendable.

\chapter Un matiz

No se diferencia el objetivismo del subjetivismo porque piense que la piedra
que he visto sigue estando donde estaba cuando cierro los ojos. Esto lo
puedo creer tanto siendo objetivista como subjetivista. Como objetivista
pensaría, además, que la piedra seguiría siendo piedra aun cuando no fuera
para mi algo manejable y duro, tan {\em dura} como para matar o herir si se
emplea la fuerza suficiente. Como subjetivista pienso, sin embargo, que si
mi tamaño fuera mil veces mayor, la piedra no sería tal sino que se
confundiría con otras y todas juntas serían para mi arena {\em blanda} donde
poder acostarme a reposar.

\chapter La lógica universal

El objetivista puede alegar que la arena no es más que un montón de
piedrecitas, y que si no existieran esas piedrecitas, aunque sean
indistinguibles, desaparecería la arena del subjetivista. La existencia de
objetos invisibles supone la existencia de objetos que nuestros sentidos no
pueden verificar, aunque pueden ser inferidos. Los objetos invisibles del
objetivista son semejantes a los objetos del subjetivista, puesto que
necesitan un sujeto que haga las inferencias. Sólo se puede prescindir del
sujeto si se postula que la lógica en la que se hacen las inferencias es una
lógica universal, esto es, si la lógica tuviera una existencia propia,
independiente de cualquier sujeto. Resumiendo, la posición objetivista ha de
optar entre aceptar que sólo existen los objetos sensibles o afirmar que
existe una lógica universal.

\chapter La epistemología y la ética

Así las cosas, creo que no descubro nada nuevo si digo que lo que se teme
del subjetivismo no es su epistemología, sino su ética. Nada nuevo, en
efecto, ya que toda la acción del viejo \person{Sócrates}\cite{Platón-IV}
tuvo como meta desterrar de Atenas la ética subjetivista personificada en el
sofista \person{Protágoras}. Y es que \person{Sócrates} entendía que es
peligroso dejar que cada individuo defina lo que es bueno y lo que es malo.


\goodpage
\chapter El yoísmo

\person{Sócrates} estaba en lo cierto, porque el subjetivismo es estéril
cuando sólo consigue dar cuenta del sujeto. A este tipo de subjetivismo lo
llamaremos yoísmo. Si lo que yo sé solamente me vale a mi, como sostenían
algunos sofistas, entonces tampoco tiene interés comunicarlo, el diálogo es
inútil y el saber queda encerrado en el yo. El yoísmo, al predicar la
imposibilidad de la comunicación, es escéptico.

\chapter Un subjetivismo no egoísta

Para que el subjetivismo sea capaz de una ética no egoísta, ha de construir
desde el yo una teoría mayor que yo, aunque no me trascienda. Esto es
posible. Veamos que nuestra teoría cumple estos requisitos.

\chapter Una ética subjetiva

La vida puede tomar consciencia de sí, esto es, puede saberse problema. Esto
ocurre cuando la evolución, al encadenarse una determinada secuencia de
circunstancias, se topa con el sujeto, que es simbólico porque sólo en
símbolos pueden expresarse los problemas. Este sujeto construye la realidad
para sobrevivir y sus objetos son simbólicos porque, como acabamos de ver,
así es su lógica, o sus categorías si preferimos el modo de
\person{Kant}\cite{Kant1787} a la manera de
\person{Wittgenstein}\cite{Wittgenstein1922}. En suma, el saber es
subjetivo y no absoluto. La forma que yo le doy a la realidad está al
servicio de mi ansia compulsiva por sobrevivir. El saber es sólo un medio,
el fin es vivir.

Ya tenemos dos de los cuatro pilares de la ética subjetivista: la vida es lo
que importa y el sujeto es sólo una parte de la vida. El segundo pilar es
trivial, y tampoco el primero es muy original; coincide, por ejemplo, con la
razón vital de \person{Ortega}\cite{Ortega1914}.

\goodpage

El tercero, yo soy libre, también nos lo proporciona la epistemología
subjetiva al declarar que la vida es un problema aparente y que el yo es la
solución del problema vital, que por ser un problema aparente no tiene
solución. Un problema es libertad condicionada y su solución es un uso de la
libertad que satisface la condición. Sin libertad no hay problema, ni hay
libertad sin problema. Luego es por no alcanzar la plenitud de una solución
por lo que el sujeto es libre y, consecuentemente, por lo que tiene sentido
la ética.

Qué he de hacer con mi libertad si lo que importa es la vida y yo soy sólo
un no querer morir. Siendo éste el planteamiento ético subjetivista, cabe
suponer que la ética resultante sea solidaria. Pero para que sea posible la
solidaridad, la epistemología subjetivista aún debe cumplir otro requisito,
o cuarto pilar, a saber, que sea posible la comunicación. La comunicación es
el traspaso de significados, por lo que precisamos de una semántica
subjetivista.

Aquí sostenemos que son los problemas los que proporcionan el significado,
de modo que sólo pueden comunicarse los individuos que comparten un mismo
problema. Siendo la vida el problema original, aquel problema del cual se
derivan todos los demás problemas, todos los seres vivos compartimos, al
menos, el problema vital. Y, en consecuencia, la comunicación es posible
entre todos los seres vivos. Es cierto que a mayor comunidad de problemas,
más rica será la comunicación. El máximo lo alcanza, precisamente, la
comunicación entre sujetos simbólicos. Los sujetos simbólicos disponemos de
un lenguaje, que es un sistema simbólico de comunicación, que, por ser
simbólico, nos permite expresar problemas, soluciones y resoluciones.

La comunicación y el lenguaje hacen posible la solidaridad, pero además son
una prueba de que el sujeto es sólo una parte de la vida y, en consecuencia,
son una prueba de que el yoísmo es insuficiente.

La comunicación y el lenguaje hacen al sujeto parte de una comunidad, que es
comunidad, precisamente, porque hay un problema común. De aquí a proponer
que todos los sujetos ayuden a solucionar el problema, no hay ni un paso.

Si escribo esto es porque quiero comunicarlo. Puedo comunicarme porque hay
un problema común. Habiendo un problema común debo ayudar a solucionarlo.
Así que no tengo más que dos opciones, o extrañarme o ayudar.

\chapter Dos cuestiones retóricas

¿Aceptaría \person{Sócrates}, en estas condiciones, el subjetivismo? Y, en
tal caso, ¿se dejaría ajusticiar?

\goodpage

\chapter Por fin

El simbolismo es el concepto central de esta introducción. Aparece como una
herramienta que permite representar problemas, resoluciones y soluciones,
pero disponer de una lógica simbólica supone disponer, igualmente, de otras
capacidades, que tenemos por distintas, pero que sólo son diferentes
aspectos de la simbolización. Son, en concreto, aquellas capacidades que
tenemos por más humanas. Veámoslas.

El simbolismo nos permite pensar en lo que no es, por ejemplo, en un
proyecto. El simbolismo permite pensar sobre problemas y resoluciones, lo
que nos hace capaces de diseñar herramientas, que son medios de resolución.
El simbolismo permite la completa representación del mundo, que incluye al
propio individuo simbólico, así que solamente los individuos simbólicos
podemos ser sujetos con consciencia de nosotros mismos y de lo demás. Con el
simbolismo aparece la poderosa comunicación sintáctica, es decir, el
lenguaje. Y sólo en el simbolismo cabe la libertad.

La ética prevaleció sobre la epistemología desde \person{Sócrates} hasta que
\person{Descartes} invirtió el orden de prevalencia. Es notorio que
\person{Kant} no pudo congeniar la epistemología y la ética. Pero al
reconocer que la libertad es el ingrediente primordial que ha construido
nuestro pensamiento a la manera simbólica, queda por fin abierta la
posibilidad de una conjunción armónica de ética y epistemología.



\prolog La respuesta aparente (RESPUESTAS)

Se formulan preguntas contestadas más adelante. Las respuestas que aquí se
dan están resumidas, es decir, son sólo primeras aproximaciones, por lo que
se recomienda encarecidamente la interpretación directa de la segunda parte
del libro para alcanzar un mayor entendimiento. Para ayudar al lector, se
señalan las secciones en donde puede proseguir la investigación sobre cada
cuestión.

\Q ¿Sabe sumar una calculadora?

La calculadora hace sumas, pero no sabe lo que significa sumar, ya que la
calculadora no se enfrenta a problema alguno, y son los problemas los que
proporcionan el significado.  Véase la \S\refsc{calculadora}.

\Q ¿Cómo es posible la comunicación?

La comunicación, o sea, el traspaso de significados, sólo es posible entre
individuos que se enfrentan al mismo problema.  Así, por ejemplo, no hay
comunicación entre el autor y el libro que escribe, sino entre el autor y el
lector.  Véanse la \S\refsc{calculadora} y la \S\refsc{comunicación}.

\Q ¿Qué significa significa?

El problema es la fuente del significado.  Los resolutores generales de
problemas son simbólicos, es decir, se componen de dos capas: semántica y
sintaxis.  Los objetos sintácticos son más versátiles, pudiendo representar
problemas y resoluciones, pero sólo tienen significado primario aquéllos que
se corresponden directamente con algún objeto semántico, con alguna
solución.  El resto de los objetos sintácticos sólo tienen un significado
derivado, o ninguno.  Los objetos sintácticos sin posible correspondencia
semántica se denominan paradojas.  Los simbolismos se estudian en la
\S\refsc{El simbolismo}, donde se muestra que los resolutores generales son
necesariamente simbólicos, pero el significado aparece también en la
\S\refsc{El problema} y en la \S\refsc{La solución}.

\Q Esta frase es falsa, ¿lo es?

Si fuera cierta, entonces, según lo que la propia frase enuncia, sería
falsa, pero si fuera falsa, entonces no podría ser lo que la propia frase
dice, o sea, sería cierta.  Como no hay modo de deshacer el enredo, decimos
que ni es falsa ni es cierta, es paradójica.  Se trata de una expresión
simbólica o sintáctica correcta, pero que no tiene significado alguno porque
no se corresponde con ningún objeto semántico. Las paradojas, cuando deben
ser resueltas, esto es, cuando se busca su correspondencia semántica,
resultan en bucles sin final. Véase la \S\refsc{Una digresión paradójica}.

\Q ¿Son peligrosas las paradojas?

No, ya que la condición de parada que evita las resoluciones demasiado
largas, también evitará las resoluciones paradójicas, que no son más que
resoluciones sin final.  Véase la \S\refsc{Una digresión final}.

\Q ¿Qué es un problema?

Un problema es libertad condicionada.  De otro modo, todo problema tiene dos
componentes:  libertad y condición.  La solución del problema es un
determinado uso de la libertad que satisface la condición.  Véase la
\S\refsc{El problema*}.

\Q ¿Hay diferencia entre solución y resolución?

Sí.  La resolución es el proceso de búsqueda de la solución que reduce el
problema, es decir, es el proceso que discurre desde el punto en el que se
tiene el problema, con su condición y su libertad no ejercitada, hasta el
momento en el que se tiene la solución, o sea, hasta el momento en el que se
ejerce la libertad y se satisface la condición. En resumen, que resolver es
a buscar como solucionar es a encontrar. Véase la \S\refsc{El problema*}.

\Q ¿Qué maneras hay de resolver un problema?

Hay tres maneras básicas:  conocer la solución del problema y aplicar tal
solución; tantear entre un conjunto de posibles soluciones, es decir,
aplicar un procedimiento de prueba y error hasta dar con una solución, y
entonces aplicarla; y sustituir el problema dado por otro análogo, o sea,
trasladar el problema, resolver éste, reconvertir la solución y aplicarla.
Estas tres maneras básicas pueden ser combinadas en árboles de resolución
más complejos.  Véanse la \S\refsc{La resolución*} y la \S\refsc{Una
notación resolutiva}.

\Q ¿Qué es la libertad?

Es uno de los dos componentes de todo problema.  Sin libertad no hay
problema. No hay libertad sin problema.  Véase la \S\refsc{El problema*}.

\Q ¿Soy yo una persona?

La pregunta es capciosa.  El mundo, que aquí entendemos como la
representación simbólica del problema de la supervivencia, tiene, como
cualquier problema, dos componentes:  la libertad, que soy yo, y la
condición, que es el resto del mundo, es el universo real.  De modo que las
personas tienen un yo, que es la representación simbólica de la propia
persona en el mundo.  Véanse la \S\refsc{El yo} y la \S\refsc{El mundo}.

\Q ¿Puedo yo ser libre?

El yo ocupa el lugar de la incógnita en el problema de la supervivencia del
sujeto.  Por lo tanto, antes de ser resuelto el problema, es libertad y el
sujeto percibe su yo como el asiento del libre albedrío.  Pero el sujeto
debe ir resolviendo el problema de la supervivencia, y esto supone ir
eliminando la libertad de la incógnita.  Luego el yo es libertad eliminada a
voluntad o, dicho de otro modo, es libertad condicionada por la biografía y
por la necesidad de vivir.  Véase la \S\refsc{sujeto}.

\Q ¿Qué soy yo?

Yo soy libertad para no morir.  La transposición positiva, yo soy libertad
para vivir, no es completamente equivalente, porque esconde la tensión que
encierra el yo.  Esta tensión sólo se muestra cuando me percato de la
finitud de la vida, finitud que la muerte evoca inmediatamente.  Para más
detalles sobre el yo, consúltese la \S\refsc{El yo}.

\Q ¿Por qué morimos?

La muerte no puede ser explicada, sino que, por el contrario, todas las
explicaciones se buscan en un intento de no morir.  Así que la muerte es la
fuente del significado; también la vida.  Estamos aceptando que el problema
de la supervivencia es {\it el problema}, y que todos los demás problemas
derivan de él.  La \S\refsc{La solución}, y en verdad todo el libro,
intentan hallar una respuesta a este interrogante.

\Q ¿Es mi cuerpo parte de mi yo?

No, el cuerpo es la parte física de mi persona capaz de ejecutar los
comportamientos que solucionan los problemas que me encuentro, problemas
todos ellos derivados en último término del problema de la supervivencia.
Mi yo es, por el contrario, la entidad sintáctica, esto es, simbólica, que
representa a mi persona en la traducción a símbolos de mi problema de la
supervivencia.  La distinción entre el cuerpo y su gobierno es lo que define
la adaptación, que es la primera etapa de la evolución resolutiva, véase la
\S\refsc{El adaptador}.

\Q ¿Para qué sirve la consciencia?

Para unificar el complejo cognitivo de los sujetos.  La consciencia es el
órgano de resolución máximo, y se encarga de resolver un único problema en
cada momento.  La solución del problema que llega a la consciencia es el yo
del sujeto.  Véase la \S\refsc{La consciencia}.

\Q ¿Se pueden pensar varias cosas a la vez?

Sí.  Aunque el pensamiento consciente es secuencial, el cerebro de las
personas es capaz de solucionar varios problemas simultáneamente.  La
consciencia, que es única, atiende a un único problema en cada momento, y de
ese modo consigue dar unidad a la persona.  Véase la \S\refsc{sujeto}.

\Q ¿Piensa un perro?

Depende de hasta dónde se lleve la definición de `pensar'.  Si todo
tratamiento de información se considera pensar, entonces hasta animales
mucho menos complejos que el perro piensan.  Por otra parte, es muy posible
que sólo las personas sean capaces de un pensamiento simbólico reflexivo,
mientras que otras varias especies de animales, entre las que se encuentra
el perro, sean capaces de varias maneras de resolución, y por lo tanto se
puedan catalogar como conocedores no simbólicos.  Otras especies menos
complejas serán capaces de elaborar representaciones internas, siendo
aprendices.  Véase, sobre todo, la \S\refsc{El conocedor}.
\looseness=-1

\Q ¿Qué es la felicidad?

Es una de las maneras de la emoción.  La emoción es uno de los datos que el
conocedor utiliza para determinar la estrategia que debe emplear para
resolver cada problema; en concreto, es el dato proviniente del propio
mecanismo de resolución de problemas.  La felicidad es el estado de
complacencia en el que se da por solucionado un problema.  Véase la
\S\refsc{El ánima}.

\Q ¿Puede ser feliz un perro?

Sí.  Suponiendo que un perro, desde el punto de vista cognitivo, es un
conocedor no simbólico, entonces es necesario que tenga emociones que guíen
su estrategia de resolución de problemas.  Siendo la felicidad la emoción
asociada al reconocimiento de que un problema está solucionado, es muy
probable que todo conocedor la necesite.  Véase la \S\refsc{El ánima}.

\Q ¿Se puede fabricar una persona?

Seguramente, aunque de momento hay cuestiones técnicamente arduas.  Lo
fundamental para que un artefacto sea tenido, en sus aspectos cognitivos,
por una persona, es que se enfrente al mismo problema y utilizando la misma
lógica y el mismo aparato de resolución que las personas de carne y hueso.
Esto plantea otro interrogante, ¿qué interés tendría fabricar una máquina
cuya razón de ser fuera perpetuar su aparato biológico y su especie?  Los
peligros para la especie humana sí son evidentes.  Por el contrario, un
robot ayudante tendría como razón de ser un problema muy distinto al de la
persona ayudada.  Véase la \S\refsc{La solución}.

\Q ¿Es inteligente un ordenador?

Sólo puede mostrar inteligencia un resolutor general de problemas tratando
de resolver cierto problema en cierta lógica.  Luego una computadora puede
mostrarse inteligente, por ejemplo, jugando una partida de ajedrez. Dado que
su problema es tan diferente del problema de la supervivencia, su
inteligencia es muy diferente de la de una persona, e incluso puede no ser
tenida por tal.  Véase la \S\refsc{inteligencia}.

\Q ¿Para qué sirve la lógica?

La lógica permite representar internamente el exterior.  Merced a la lógica
los aprendices pueden representar soluciones, o sea, comportamientos.  Si la
lógica es simbólica, entonces también pueden representarse problemas y
resoluciones.  Véanse la \S\refsc{El aprendiz} y la \S\refsc{El simbolismo}.

\Q ¿Qué es un comportamiento?

Es un patrón que determina, para cada posible estímulo externo y cada
posible contenido de la memoria interna, cuál será la acción ejecutada sobre
el exterior y cuál será el nuevo contenido de la memoria interna. Por
ejemplo, programar una computadora consiste en definir su comportamiento. La
definición técnica de comportamiento se encuentra en el Anexo \refsc{El
álgebra automática} (\S\refsc{El comportamiento}).

\Q ¿Hay diferencia entre adaptación y aprendizaje?

Sí, un adaptador es capaz de varios comportamientos que va probando hasta
encontrar uno adecuado, mientras que el aprendiz emplea una manera más
elaborada de adaptación que se sirve de una lógica interna en la que elabora
modelos de lo externo para poder anticipar el resultado de las pruebas sin
tener que realizarlas sobre el exterior y, por ello, sin tener que sufrir
sus consecuencias.  Véanse la \S\refsc{adaptadores} y la
\S\refsc{aprendices} o, en general, la \S\refsc{El adaptador} y la
\S\refsc{El aprendiz}.

\Q ¿Para qué sirve la sintaxis?

La sintaxis es la novedad que introducen las lógicas simbólicas. Merced a la
sintaxis, las lógicas simbólicas pueden representar, además de las
soluciones que toda lógica es capaz de representar, los problemas y las
resoluciones.  De este modo un conocedor simbólico es capaz de resolver un
problema de cualquiera de las maneras.  Véase la \S\refsc{El simbolismo},
sobre todo la \S\refsc{El dominio de la sintaxis}.

\Q ¿Qué razón evolutiva tiene el simbolismo?

Cognitivamente, cada especie es capaz de unas determinadas maneras de
resolver problemas.  Un conocedor simbólico puede anticipar el resultado de
los diversos modos de resolver, esto es, interioriza la propia evolución. Es
la evolución cognitiva que, al parecer, sólo puede desarrollar el {\it homo
sapiens}.  Véanse la \S\refsc{La evolución resolutiva} y la
\S\refsc{Justificación}.

\Q ¿Qué es la abstracción?

Es utilizar un problema con varias soluciones para referirse, en perífrasis,
al conjunto de dichas soluciones.  De otra manera, es referirse a un
conjunto de soluciones por sus propiedades.  Véase la \S\refsc{abstracción}.

\Q ¿Qué es un vaso?

En genérico, el vaso es el útil que soluciona nuestro problema de beber, por
lo que es un ente abstracto.  Por otra parte, un vaso concreto es un objeto,
es decir, una parte del universo que nuestra atención determina que debe ser
tratada como una unidad, en este caso como una de las soluciones al problema
de beber, porque ello simplifica su tratamiento. Es por esto por lo que los
objetos sólo pueden ser definidos teniendo en cuenta nuestras necesidades y
apetencias.  De nuevo se muestra que el significado lo proporciona el
problema.  Véanse la \S\refsc{abstracción} y la \S\refsc{La atención}.

\Q ¿Por qué sufrimos ilusiones?

Porque lo que alcanza el aparato simbólico de las personas no es la
apariencia bruta, sino una apariencia, ya procesada, de objetos.  En nuestra
jerga, porque el resolutor general simbólico está construido sobre un
resolutor especializado.  El sistema visual hace un enorme procesamiento no
simbólico de la apariencia bruta, de manera que, por ejemplo, una zona de
cierto color, que cambia de posición en la retina sin variar de forma, puede
ser entregada al aparato simbólico como un objeto. La ilusión se produce
cuando el proceso simbólico consciente descubre alguna anomalía en los
resultados del procesamiento previo.  La \S\refsc{La partición aparente}
explica el preprocesamiento no simbólico de la apariencia.

\Q ¿Cuál es el último reducto del absolutismo?

El pensamiento.  Que me convenga denominar vaso a un `vaso' no implica que a
otra inteligencia, esto es, a otro resolutor tratando de resolver otro
problema en otra lógica, también le interese usar el concepto abstracto
vaso.  Luego la existencia del `vaso' está subordinada al problema, a su
resolutor y a la lógica de éste. Esto se aplica, igualmente, a cualquier
otro concepto, ya sea de los considerados físicos, ya sea abstracto. Véanse
la \S\refsc{El límite} y la \S\refsc{inteligencia}.

\Q ¿Hay diferencia entre saber y conocimiento?

Sí.  El conocimiento es la descripción simbólica, o sea, inmutable, de la
cambiante apariencia.  El saber es la explicación que yo juzgo buena. El
conocimiento es objetivo, ya que es independiente de la voluntad, pero no es
concluyente, ya que no me alcanza, es descriptivo.  El saber me alcanza, y
por esta razón es libre, subjetivo y concluyente, es decir, explicativo.
Ninguno es completo porque es imposible conocer la apariencia aún no
experimentada y porque la muerte nunca podrá ser sabida.  Véase la
\S\refsc{El saber}.

\Q ¿Explica la ciencia?

No.  Al limitarse a lo objetivo, a lo que queda fuera de la voluntad, la
ciencia sólo describe.  El conocimiento científico no alcanza el yo, y por
esto no explica, sólo describe.  Véanse la \S\refsc{ciencia} y la
\S\refsc{La metafísica}.

\Q ¿Qué conceptos son transcendentes?

Ninguno. Nuestro pensamiento simbólico es, como nuestro cuerpo físico,
un producto de la evolución darwiniana. Incluso el yo de
\person{Descartes} es un producto del proceso evolutivo y, por ello, una
herramienta diseñada por la evolución con el propósito de conservar la vida.
Más allá de esa utilidad inmediata, pierde por completo su significado.  De
manera que la razón es contingente y no absoluta ni transcendente.  Éste es
un tema recurrente que se trata, especialmente, en la \S\refsc{La solución}.
Es instructiva la lectura del libro de
\person{Lakatos}\cite{Lakatos1976}, que desbarata el ideal platónico,
precisamente, en su ejemplo más querido, las matemáticas.

\Q ¿Qué niega el materialismo?

La libertad.  El materialismo postula que todo está regido por las leyes de
la naturaleza, y no deja lugar al libre albedrío.  Sin libertad no hay
problema.  Para el materialismo ni siquiera la muerte es un problema, ya que
también al morir se conserva la energía.  En fin, el materialismo es una de
las maneras de negar que [la mejor manera de entender] la vida es
[considerarla] un problema aparente, y por lo tanto se opone a las tesis
aquí defendidas.  Véase la \S\refsc{materialismo}.

\Q ¿Qué es la vida?

La vida no puede ser explicada, sino que es el motivo que hace necesarias
las explicaciones, al menos cuando la vida alcanza evolutivamente la
complejidad del conocedor simbólico.  Desde el punto de vista cognoscitivo,
la vida tiene como modelo el problema aparente.  Véase toda la \S\refsc{La
solución}, aunque sólo leyendo el libro por completo se consigue una visión
global.

\Q ¿Cuál es el problema aparente?

El problema aparente es aquél en el que sólo está definida la interacción,
es decir, en el que el único dato es la apariencia.  Es como el problema de
aprovechar una caja negra, de la que no se tiene manual de uso y cuyo
interior es inaccesible.  Podemos ejercer acciones sobre ella, anotar sus
reacciones, conjeturar sobre su funcionamiento y, así, elaborar algún modo
de actuar que resulte provechoso.  ``El problema aparente'' postula que ésa
es nuestra situación en el mundo, y extrae las consecuencias.  Véanse la
\S\refsc{El planteamiento} y la \S\refsc{El problema aparente}.


\prolog Sinopsis (SINOPSIS)

\chapter Propósito

\label{Sinopsis}
En la segunda parte del libro se expondrá sistemáticamente la teoría
presentada en esta introducción. El propósito de esta sinopsis es mostrar
algunas de las motivaciones que se esconden detrás de lo expuesto concisa y
tersamente en la segunda parte, y trazar su línea argumental.

\chapter La evolución

Partiendo de la teoría de la evolución de \person{Darwin}, es preciso
concluir que también las personas y sus obras han de ser explicadas,
finalmente, como consecuencia del proceso del que son productos, a saber, el
proceso de selección natural.  Esto, que puede parecer obvio, tiene
consecuencias que no son tan fáciles de asimilar.  Por ejemplo, así como
entendemos que tener cinco dedos en cada mano es un resultado, no necesario,
del azaroso proceso evolutivo, así también deben ser entendidos los
conceptos filosóficos, científicos, éticos, religiosos o prácticos.  En
definitiva, ningún concepto puede ir más allá de la vida, porque también los
conceptos, las ideas y las verdades son herramientas diseñadas por la
evolución para sobrevivir más y mejor.

\chapter La contingencia

La conclusión anterior establece dos requisitos que son aplicables a
cualquier teoría del conocimiento, o epistemología.  Uno es que las
conclusiones de tal teoría cognitiva deben corroborar el carácter
contingente del pensamiento humano.  El otro es que las explicaciones deben
ser evolutivas, esto es, deben contentarse con mostrar la viabilidad de los
hechos si se dan las condiciones adecuadas, quedando en cambio relegadas
las deducciones, más soberbias, que demuestran la necesidad de las
conclusiones.

\chapter La libertad

Pero tampoco el materialismo, que todo lo explica en base al azar y a la
necesidad de las leyes de la naturaleza, negando cualquier posibilidad a la
libertad y el libre albedrío, parece una explicación suficiente.  Y aun
así, la única explicación aceptable es la científica.  Porque la
explicación debe ser una teoría que utilice conceptos precisos para que
sus consecuencias puedan ser verificadas.  De manera que el paradigma que
buscamos no puede ser el autómata finito del materialismo.  Un autómata
finito y determinístico es un mecanismo en el que el conocimiento
completo de su estado inicial y de las acciones que le afectan permite
determinar completamente su comportamiento, esto es, sus reacciones y sus
estados subsiguientes.  Tampoco vale la extensión probabilística, en
que la indeterminación viene dada exclusivamente por el azar.

\chapter El problema aparente

La elección del problema aparente como paradigma teórico tiene varias
consecuencias inmediatas.  Por un lado introduce desde el principio dos
conceptos fundamentales:  la libertad, porque sin posibilidades entre las
que elegir no hay problema, y el significado, porque si todas las
posibilidades son igualmente buenas entonces tampoco hay problema.  Por
otro lado, exige la elaboración de una teoría del problema que ocupe el
lugar de la teoría del autómata finito.  Por último, retoma la
postura filosófica que nace con los empiristas británicos y que corrige
\person{Kant}.  Los empiristas habían eliminado cualquier tentación de
sustancia dejando, como único dato, la apariencia.  \person{Kant}, ante
el escepticismo de \person{Hume}, había postulado la existencia de una
lógica interna, sus categorías, en las que se reflejaba o reconstruía
lo que hubiera tras las apariencias.  El problema aparente añade, a
sugerencia de \person{Darwin} y como origen de todo significado, la
supervivencia.

\chapter La lógica

Para poder atacar el problema aparente de un modo científico, y no sólo
filosófico, hemos de plantearlo matemáticamente.  Por esto introducimos
una lógica, definida a la manera de \person{Wittgenstein} como el marco
que delimita lo que es posible.  La elección del álgebra de los
autómatas finitos como la lógica en la que se plantea y se resuelve el
problema aparente, parece acercarnos a los materialistas, aunque ya veremos
luego que no es así.  Dos son las razones que aconsejan elegir el
álgebra automática:  la primera es que nos permite definir y tratar
matemáticamente los comportamientos, de modo que podemos referirnos
indistintamente a comportamientos o a autómatas, y la segunda es que
\person{Turing} nos mostró que en el autómata finito está implícita
toda la potencia de cálculo simbólico de las computadoras digitales, así que
cualquier cálculo que pueda hacer una computadora, también puede hacerlo un
autómata finito.  Una consecuencia de elegir como lógica el álgebra
automática, es que las soluciones, y por lo tanto los objetos semánticos de
la teoría resultante, serán comportamientos, y no acciones.

\chapter La formalización

Para el problema aparente hay dos clases de reacciones, las buenas, que son
las deseadas, y las malas, que son el resto, y su solución es aquélla que
sólo recibe reacciones buenas.  Cada posible solución ejecuta acciones para
intentar que las reacciones sean buenas, y no malas.  Pero sólo conoce la
apariencia, esto es, las acciones y las reacciones, y nada sabe de lo que
media desde las acciones hasta las reacciones, y aun desconoce si media algo
o no.

En castellano, el planteamiento del problema aparente en el álgebra
automática, puede ser leído así:  hallar aquel comportamiento,
definido como un autómata finito~|A, que enfrentado a {\it cualquier\/}
otro comportamiento, al que denominamos entorno o universo~|U, obtenga de
este universo las reacciones que desee.  Cuando lo consigue, y considerando
al problema aparente como el modelo de la vida, decimos que sobrevive.

Para expresar que sólo la apariencia es conocida, escribimos que el
universo~|U podía ser cualquiera, y como consecuencia el problema así
planteado no tiene una solución definitiva.  No la tiene porque, al quedar
el universo completamente indefinido, es posible encontrar universos que
harían imposible la solución.  Una vez demostrado que el problema aparente
no tiene una solución definitiva, no queda otra alternativa que buscar
soluciones relativamente mejores, para lo cual es preciso definir una
solución de referencia.  Llamamos mecanismo~$|A_0$ a la solución de
referencia.  En este punto finaliza la sección primera, \S\refsc{El
problema}, de la segunda parte del libro.

\vfil

$$\hbox{\rm El problema aparente}\llave{
 El problema: & el mecanismo\cr
 El adaptador: & el gobernador y el cuerpo\cr
 El aprendiz: & la lógica interna\cr
 La resolución: & la teoría del problema\cr
 El simbolismo: & la sintaxis y la semántica\cr
 El conocedor: & la mente resolutiva\cr
 El sujeto: & la consciencia y el yo\cr
 La solución: & una epistemología subjetiva\cr
}$$

\break

\chapter El adaptador

El adaptador~$|A_1$ se compone de dos partes:  un cuerpo capaz de varios
comportamientos y un gobernador que decide en cada momento y circunstancia
cuál de los comportamientos debe ejecutar el cuerpo.  Para que este
adaptador~$|A_1$ aventaje al mecanismo~$|A_0$, es suficiente que se cumplan
dos requisitos: \goodpage
\item{$\bullet$} que uno de los comportamientos del cuerpo sea el
comportamiento del mecanismo~$|A_0$, lo cual se expresa técnicamente
diciendo que el cuerpo del adaptador debe ser una ampliación del
mecanismo~$|A_0$, y
\item{$\bullet$} que el gobernador del adaptador sepa elegir tal
comportamiento cuando el universo~|U al que se enfrenta es uno que
solucionaba el mecanismo; a este requisito lo denominamos {\em condición de
adaptación}.

\chapter El aprendiz

La condición de adaptación es resuelta por el adaptador simple empleando el
procedimiento de tanteo, esto es, probando contra el universo~|U los varios
comportamientos de los que es capaz su cuerpo, hasta que encuentra uno lo
bastante bueno.  El aprendiz~$|A_2$ es un adaptador más complejo, que añade
la posibilidad de predecir el resultado de los comportamientos sin tener que
sufrir las consecuencias de ejecutarlos, pero entonces necesita una lógica
interna en la que llevar a cabo tales cálculos. Para hacer las predicciones
el aprendiz~$|A_2$ ha de buscar un modelo, al que denominamos realidad~|R,
que sustituya al universo~|U exterior en la lógica interna.  Lo que se pide
a este modelo~|R es que cumpla la {\em condición de modelación}, o sea, que
dada una secuencia de acciones produzca las mismas reacciones que produciría
el universo~|U, o, dicho técnicamente, la realidad~|R debe ser
indistinguible del universo~|U. La otra condición necesaria para que el
aprendiz~$|A_2$ mejore al adaptador~$|A_1$ es la llamada {\em condición de
aprendizaje}, que se cumple si el aprendiz~$|A_2$ soluciona en su lógica
interna el problema aparente ya interiorizado en el que la realidad~|R
interior sustituye al universo~|U exterior.

\chapter La teoría del problema

La teoría del problema, \S\refsc{La resolución}, es la clave del libro.
Define problema como una condición puesta a cierta libertad, solución como
aquel uso de la libertad que satisface la condición, y resolución como el
proceso de búsqueda de la solución.  Además descubre tres formas básicas de
resolución, la solución conocida, el tanteo y la traslación o analogía, que
pueden combinarse en árboles de resolución más complejos.  Es importante
advertir que los problemas aparentes, que no tienen una solución {\it a
priori}, sí admiten resoluciones {\it a priori}.  Es decir, que, aunque no
es posible determinar {\it a priori\/} si un comportamiento será o no más
provechoso que otro, sí se puede mostrar {\it a priori\/} que, para
enfrentarse a un problema aparente, algunas estrategias de actuación son
mejores que otras, y por esto hemos podido mostrar que el aprendizaje es
mejor que la adaptación. De manera que el mecanismo~$|A_0$, el
adaptador~$|A_1$ y el aprendiz~$|A_2$ son resolutores capaces de un modo
específico de resolución cada uno de ellos. Constituyen, además, una
secuencia evolutiva en continua mejora, siempre y cuando se cumplan las
condiciones de adaptación, de modelación y de aprendizaje.  De la teoría del
problema se sigue la manera de proseguir esta evolución de resolutores:  el
conocedor~$|A_3$ debe ser capaz de varios modos de resolver, incluyendo el
del mecanismo~$|A_0$, el del adaptador~$|A_1$ y el del aprendiz~$|A_2$.  Por
otra parte, si la lógica interna del conocedor fuera capaz de representar
problemas, soluciones y resoluciones, como veremos luego que es el caso de
las lógicas simbólicas, entonces este conocedor simbólico sería capaz de
desarrollar internamente una evolución resolutiva.  Es recomendable leer
directamente la sección cuarta, \S\refsc{La resolución}, en la que se pueden
encontrar explicaciones aumentadas de todos estos conceptos.

\chapter La lógica simbólica

Una lógica simbólica es aquélla con dos niveles, uno sintáctico en el que
habitan las expresiones con sus símbolos y sus nombres, y el otro semántico,
más primario, que es una lógica completa y en el que han de estar las
soluciones. Si se trata del problema aparente, y la lógica en la que se
plantea es el álgebra de autómatas, entonces las soluciones son
comportamientos. Se muestra, en la \S\refsc{El simbolismo}, que una lógica
simbólica es capaz de representar problemas, soluciones y resoluciones,
tanto los tres modos básicos de resolución como las composiciones en forma
de árboles de resolución más complejos.  Se muestra que tanto el problema
como la resolución son conceptos necesariamente sintácticos. En este punto,
al mostrarse que la libertad es un concepto sintáctico y que la lógica
simbólica es el lenguaje de los problemas, es donde queda superado el
materialismo.  Por añadidura, esta explicación de la lógica simbólica desde
el paradigma del problema permite discernir cuáles son los elementos básicos
\vadjust{\newpage}%
de sus expresiones: los nombres, las variables y las incógnitas, las
condiciones booleanas, los conjuntos, con sus elementos, su relación de
pertenencia y sus cuantificadores, y los algoritmos transformadores de
expresiones.  Por último, interpretando la lógica simbólica como el soporte
expresivo de los problemas, las soluciones y las resoluciones, asuntos
siempre sospechosos como las paradojas y los problemas descubiertos por
\person{G\"odel} en los formalismos matemáticos reciben una
explicación natural.

\chapter El conocedor

El conocedor~$|A_3$ es un resolutor que mejora al aprendiz~$|A_2$, al
adaptador~$|A_1$ y al mecanismo~$|A_0$, porque es capaz de las maneras de
resolver de éstos y, además, de otras.  El conocedor se divide en dos
partes:  el ánima, que decide cuál de los modos de resolución aplicar, y la
mente, que resuelve a la manera determinada por el ánima.  Para que el
conocedor mejore al aprendiz, su ánima debe elegir el modo de resolución del
aprendiz cuando se enfrente a un universo~|U que el aprendiz solucionaba, e
igualmente, {\it mutatis mutandis}, en los casos del adaptador y del
mecanismo.  Ésta es la condición de conocimiento.  Si el conocedor es
simbólico, entonces su mente simbólica es capaz de todas las maneras de
resolver, y el problema del ánima es tan complejo que debería contar con
todos los recursos de resolución que la teoría del problema descubre.  Pero
este problema tiene una solución fácil en el caso del conocedor simbólico,
ya que su característica consiste, precisamente, en disponer de todos los
recursos de resolución.  Para ello es imprescindible que también el ánima
esté construida sintácticamente.  De aquí se concluye que el conocedor
simbólico se mostrará, única y exclusivamente, como un motor sintáctico.

\chapter El sujeto

El sujeto es un conocedor simbólico que se enfrenta a un problema
complejo.  Para ello debe ser capaz de resolver varios subproblemas
simultáneamente.  Para dar coherencia y unidad a este resolutor
múltiple, ha de estar construido jerárquicamente, de modo que exista un
nivel superior, la consciencia, que sea único.  A la consciencia sólo
llega, en cada momento, un único problema, aquél más difícil y que
ha pasado todos los filtros interpuestos en su camino ascendente.  La
solución del problema simbólico que alcanza la consciencia del sujeto es
su yo, de manera que la tarea de la consciencia consiste en definir ese yo.
Como el yo ocupa el lugar de la incógnita a determinar en el problema del
sujeto, el sujeto experimenta el yo como libre.

\chapter Las consecuencias

La \S\refsc{La solución} postula que los fundamentos de todo nuestro saber y
conocimiento son la apariencia y el deseo de vivir, que son las dos partes
del problema aparente, y elabora algunas de las consecuencias filosóficas.
Una de las más controvertidas puede ser la que atañe al orden jerárquico de
las ciencias, ya que, según el argumento presentado, la biología explica la
lógica simbólica y las matemáticas, por lo que sería más fundamental que
ellas.  Pero ya que el argumento es matemático, lo que resulta es un orden
circular o, mejor, de dependencias mutuas. Recomendamos la lectura completa
de la sección octava, \S\refsc{La solución}, donde pueden encontrarse otras
importantes consecuencias.

\chapter El título

Lo aparente (`apparent') es, en inglés, lo obvio, acusando el idioma la
influencia de la filosofía empirista británica.  En castellano, aun
siendo la raíz de la palabra la misma, lo aparente es lo que parece pero
no es, y aparentar es una manera de engañar.  ``El problema aparente''
está escrito, aparentemente, en el idioma equivocado.

\chapter Palabras clave

Las ideas fundamentales de la teoría presentada son:  problema,
conocimiento, evolución, simbolismo, autómata, lógica, adaptación,
aprendizaje, semántica y sintaxis.

%\nobreak\bigskip
%\line{\indent La Coruña, 11 de abril de 1997.\hfil
% R.\ María C.\ Gallego}


\endinput

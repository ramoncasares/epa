% EPI2.TEX (RMCG19980511)

\prolog Reflexiones subjetivistas sobre\cr símbolos y explicaciones
        (SÍMBOLOS Y EXPLICACIONES)

\chapter El problema de la supervivencia

No hay problemas aislados. Por ejemplo, la solución de un problema de
cálculo diferencial, para un determinado estudiante, puede ser abandonar la
carrera de ingeniería para estudiar leyes. Quiero decir que cada problema es
un subproblema de otro problema más general. Siguiendo con el ejemplo,
conocer las leyes tampoco es el objetivo último, sino seguramente ser capaz
de obtener ingresos como abogado. Y aunque en ocasiones lo pensemos, tampoco
el dinero es lo último. Con el dinero puede conseguirse comida y cobijo, que
contribuyen, ya sí, al objetivo final, que es vivir.

En el origen de todos nuestros problemas nos encontramos, siempre, el
problema de vivir, que denominaremos {\em el problema de la supervivencia}.
Esto nos parece claro en el caso de los organismos que, homocéntricamente,
denominamos inferiores. En nuestro propio caso, podemos desdeñar
hipócritamente el problema de la supervivencia y sustituirlo por otros más
sublimes. Todos estos ideales sublimes son simbólicos: el Dios de la
religión, el Amor de la poesía, el Progreso de la ciencia, el Saber de la
filosofía, la Virtud de la ética, la Belleza del arte, el Poder de la
política, la Justicia de la ley, el Valor de la milicia, la Patria del
nacionalismo, la Raza, etc.

De otra forma. Si todavía queda gente que se niega a aceptar la explicación
darwiniana, es porque ésta le obliga a aceptar que, en último término, lo
que nos encontramos es el problema de la supervivencia y no sublimes
ideales. Si usted, amable lector, se encuentra en dicho grupo, considerará
que todo lo dicho en este libro es un puro sinsentido.

\vfil
\centerline{\it To be, or not to be ---
  that is the question\cite{Shakespeare1601}}
\vfil
\break

\chapter La vida

El cráneo protege al cerebro del mundo exterior, pero a la vez lo aísla. El
cerebro no ve el exterior, sino que recibe información de los ojos. Y
tampoco puede actuar sobre el exterior, sólo puede dar órdenes a los
músculos para que éstos actúen.

Las percepciones no son neutras. Sentimos dolor al tocar algo demasiado
caliente. Ésta es una información adicional que condensa un saber aprendido
evolutivamente. Simplemente codifica el dato de que el calor excesivo mata.
Como sabemos desde tiempos de \person{Darwin}\cite{Darwin1859}, esta
codificación no la hizo ningún sujeto inteligente. Ocurre que disponer de
este dato favorece la vida, de modo que los organismos que por algún azar lo
codificaron genéticamente, sobrevivieron mejor y tuvieron una probabilidad
mayor de alcanzar la madurez y de dejar descendencia que aquellos otros
individuos más temerarios que eran insensibles al mucho calor. Lo más
importante es que la prole de los primeros, al heredar esta prevención al
calor excesivo, perpetuó la ventaja de sentir dolor.

Luego el cerebro recibe información valorada del exterior y envía al cuerpo
información ejecutiva. Y esto debería ser todo, es decir, desde
\person{Descartes} no deberíamos añadir ninguna otra información, ya que lo
único cierto e indudable es el yo interior. Sin embargo el cerebro de las
personas contiene otras informaciones que, como las sensaciones de dolor y
de placer, también han sido codificadas a lo largo de la evolución. Por
ejemplo, saber que el corazón ha de latir más rápidamente al estar corriendo
para oxigenar mejor los músculos, es una información de este tipo. Este tipo
de información pasa de padres a hijos a través de la herencia genética.
También se hereda el aparato simbólico. Por contra, ningún conocimiento
simbólico pasa genéticamente de padres a hijos. El idioma no se hereda, se
aprende.

La vida es un proceso que, en determinadas circunstancias, se extiende y
dura. Distinguiendo la vida de su entorno, ocurre que algunas reacciones del
entorno benefician la vida y otras la perjudican. Además lo vivo actúa sobre
su entorno exterior, de modo que nos encontramos con que la vida se
encuentra en una situación análoga al cerebro, pero límite. Límite, porque
la vida no cuenta con información adicional alguna. Lo único que tiene es la
posibilidad de actuar y el hecho de que algunas reacciones del entorno son
perjudiciales y otras beneficiosas.

La similitud entre la situación de la vida tomada como una entidad entera y
la del cerebro de las personas no es casual.


\goodpage
\chapter El problema aparente

Abstrayendo los detalles físicos de la vida, que resultan irrelevantes en
una investigación sobre el saber y el conocimiento, podemos quedarnos con
que el problema aparente describe adecuadamente la situación epistemológica
de la vida. Esto se resume en una frase: el problema de la supervivencia es
un problema aparente.

En un {\em problema aparente} hay dos clases de reacciones, las buenas que
son las deseadas y las malas que son el resto, y la solución del problema
aparente es aquélla que sólo recibe reacciones buenas. Cada posible solución
ejecuta acciones para intentar que las reacciones sean buenas, y no malas,
pero sólo conoce la apariencia, esto es, las acciones y las reacciones, y
nada sabe de lo que media desde las acciones hasta las reacciones, y aun
desconoce si media algo o no. Hay libertad para elegir las acciones y la
condición de que las reacciones sean buenas. Éste es el problema aparente.

La primera conclusión es que el problema aparente no tiene una solución
definitiva, porque al quedar el exterior, que denominaremos universo,
completamente indefinido, podría incluso ocurrir que las reacciones fueran
siempre, e independientemente de todo, malas. Al no tener el problema
aparente una solución definitiva, lo único que se puede hacer es definir
resoluciones que, para cada vez más universos, obtengan una solución. Debe
cuidarse la distinción entre solución y resolución; resolver es a buscar
como solucionar a encontrar.

Si partimos de un {\em mecanismo}, esto es, un ente que aplica mecánicamente
un único comportamiento a la resolución del problema aparente, podemos
establecer que un {\em adaptador}, capaz de varios comportamientos que rige
por un procedimiento de prueba y error, es potencialmente mejor resolutor
que el mecanismo. El adaptador, al enfrentarse a un universo, va probando
sus distintos comportamientos hasta dar con uno que obtiene buenas
reacciones, esto es, hasta encontrar uno que soluciona el problema.

Y un {\em aprendiz}, capaz de varios comportamientos como el adaptador, pero
que además dispone de una lógica interna en la que puede modelar y simular
el comportamiento del universo exterior, mejora al adaptador, porque no
tiene que sufrir las consecuencias de aplicar sus acciones para descubrir si
las reacciones correspondientes son perjudiciales. Con el aprendiz aparece
la modelación interna de lo externo.


\chapter Ingenieros

El problema aparente no tiene interés en ingeniería. Y no lo tendrá mientras
el ingeniero sea capaz de modelar, aunque sea mínimamente, el entorno de
funcionamiento de las máquinas que diseña. Teniendo alguna información sobre
el entorno, lo más útil es usarla.

Cuando hay alguna incertidumbre en el entorno que se encontrará la máquina,
el ingeniero diseña adaptadores, como los termostatos.


\chapter El conocedor simbólico

El mecanismo, el adaptador y el aprendiz realizan, cada uno, una única
manera de resolver. En el caso del aprendiz, y ya que en su lógica interna
pueden representarse comportamientos, se busca el comportamiento que
represente y sustituya al universo exterior, pero no es posible representar
problemas ni resoluciones.

Para que una lógica pueda representar problemas, soluciones y resoluciones
ha de ser simbólica, esto es, ha de tener una sintaxis, que es donde están
las expresiones simbólicas, y una semántica, en donde se encuentran las
soluciones. Los problemas y las resoluciones son necesariamente sintácticos,
porque la manera de expresar la libertad del problema es usar símbolos que
no se refieran a solución concreta alguna, o sea, usar símbolos sin
referente semántico que son, por ello, puros artefactos sintácticos. La
resolución del problema simbólico consiste en la aplicación de un algoritmo
de resolución que determine qué objeto conviene a la libertad del problema
para que se satisfaga la condición. El objeto simbólico así determinado ha
de referirse a un comportamiento que efectivamente pueda emplearse para
solucionar el problema o, dicho de otro modo, las soluciones han de tener
significado. En resumen, que así como los problemas y las resoluciones son
objetos sintácticos, las soluciones son objetos semánticos.

El {\em conocedor simbólico} dispone de una lógica simbólica y en ella puede
representarse, no sólo el comportamiento del exterior como el aprendiz, sino
el problema de la supervivencia entero que él mismo intenta solucionar. Para
que un resolutor pueda representarse a sí mismo, ha de disponer de una
lógica que permita la expresión de resoluciones. Esto implica que el único
tipo de resolutor que puede verse a sí mismo resolviendo es un conocedor
simbólico. Sólo un conocedor simbólico puede ser un {\em sujeto}.

\chapter Reiteración

Sólo un conocedor simbólico puede ser un sujeto.


\chapter La lógica

Estamos utilizando el término lógica a la manera de
\person{Wittgenstein}\cite{Wittgenstein1922}, como la totalidad de lo
posible. Esta totalidad incluye a todos los objetos posibles, a todas las
relaciones posibles entre ellos, a todas sus posibles transformaciones y, en
general, comprende todo lo imaginable, todo aquello que puede ser pensado.

Pero tal vez esto sea más fácil de entender al revés, es decir, explicando
que la imaginación de cada individuo queda limitada por su lógica. Un
aprendiz sólo puede pensar comportamientos. Un conocedor simbólico puede
pensar problemas, soluciones y resoluciones. Ya que la condición de un
problema puede ser un comportamiento, la lógica del conocedor simbólico es
más expresiva que la del aprendiz, y la incluye. Esto requiere más
explicaciones.

Si intento atravesar una pared, me hago daño y no lo consigo. Ésta es una
creencia fuertemente asentada, o sea, una verdad. De modo que cuando quiero
ir a algún lugar busco caminos que no requieran atravesar paredes. Ahora voy
a decir lo mismo, pero introduciendo algunos conceptos. Mi realidad, que es
el modelo que yo tengo del exterior, incluye la imposibilidad de atravesar
paredes, de modo que cuando me planteo un problema, por ejemplo cómo llegar
a la cocina, uso la realidad como condición. Ya que no puedo atravesar las
paredes, en lugar de ir en línea recta, voy por el pasillo.

El problema es, como concepto, más general que el comportamiento, porque un
problema puede incluir un comportamiento, pero un comportamiento no puede
incluir un problema. La condición del problema puede ser un determinado
comportamiento, por ejemplo el comportamiento de las paredes. Un
comportamiento no es un problema. Un comportamiento no tiene ningún grado de
libertad, no se cuestiona, por lo que no es, ni puede ser, un problema.

\vfil
\setbox0=\hbox{\it Ve}\setbox2=\hbox{\it V\noboundary e}
\dimen0=\wd0 \advance\dimen0 by -\wd2
\centerline{\it 5.61 Die Logik erfüllt die W\kern\dimen0 elt}
\vfil
\break

\chapter Los organismos

Una organización compuesta por pocas personas que se ven a diario, como una
familia, suele necesitar poco aparato administrativo, o ninguno. A medida
que aumenta el tamaño de la organización, también se incrementan los
recursos necesarios para mantener la coherencia de la organización. La
organización debe enfrentarse a los problemas como una unidad, a pesar de
estar formada por muchos miembros individuales.

Para que cada miembro reciba información suficiente, deben establecerse
cauces de información, tanto referente a la situación como a los objetivos.
Además es preciso fijar reglas que establezcan las responsabilidades de cada
uno, y así evitar que dos miembros de la organización tomen decisiones
opuestas. Cuando la organización es muy grande, por ejemplo del tamaño de un
estado, el número de reglas es enorme, y la cantidad de recursos dedicados a
la administración es una fracción importante del total.

Otros rasgos de los organismos complejos son la especialización y la
jerarquización. La especialización permite que cada individuo realice una
tarea concreta con toda su dedicación, de modo que su habilidad en esta
función es mayor que si tuviera que diversificar su actividad. La jerarquía
es una manera eficaz de mantener la unidad del conjunto. Si la actuación de
dos individuos colisiona de algún modo, se puede recurrir al nivel superior,
para arbitrar una solución. Cuando el organismo es grande, puede haber
varios niveles jerárquicos. La persona es un organismo complejo que está
organizado jerárquicamente para así procurarse un comportamiento unitario.

La persona es un individuo y es, a la vez, un organismo complejo. Lo primero
interesa a la ética, lo segundo a la epistemología.

\vfil
\MTbeginfigure(120,80);
 \MT: pickup thick_pen;
 \MT: save u, v, alpha, betha;
 \MT: u = w/6; v = w/12; alpha = 0; betha = 0;
 \MT: z1 = (w/3,h/2);
 \MT: z11 = z1 + 1.2u*(right rotated (alpha+0));
 \MT: z12 = z1 + 1.2u*(right rotated (alpha+60));
 \MT: z13 = z1 + 1.2u*(right rotated (alpha+120));
 \MT: z14 = z1 + 1.2u*(right rotated (alpha+180));
 \MT: z15 = z1 + 1.2u*(right rotated (alpha+240));
 \MT: z16 = z1 + 1.2u*(right rotated (alpha+300));
 \MT: draw fullcircle scaled (w/10) shifted z1;
 \MT: for i := 11 upto 16:
 \MT:  draw fullcircle scaled u shifted z[i];
 \MT: endfor
 \MT: z2 = (4w/5,3h/4);
 \MT: z21 = z2 + 1.4v*(right rotated (betha+0));
 \MT: z22 = z2 + 1.4v*(right rotated (betha+60));
 \MT: z23 = z2 + 1.4v*(right rotated (betha+120));
 \MT: z24 = z2 + 1.4v*(right rotated (betha+180));
 \MT: z25 = z2 + 1.4v*(right rotated (betha+240));
 \MT: z26 = z2 + 1.4v*(right rotated (betha+300));
 \MT: draw fullcircle scaled (w/10) shifted z2;
 \MT: for i := 21 upto 26:
 \MT:  draw fullcircle scaled v shifted z[i];
 \MT: endfor
\MTendfigure"Los círculos centrales\cr son iguales"%
             Figura adaptada de \person{Resnikoff} y de \person{Gazzaniga};
\vfil
\break


\chapter ¿Por qué nos gobernamos por símbolos?

La persona es un organismo complejo que está organizado jerárquicamente para
así procurarse un comportamiento unitario. El nivel superior de la jerarquía
es simbólico, porque es más general. Sólo un resolutor {\em simbólico} del
problema aparente de la supervivencia puede tener una representación
completa de la situación, situación que incluye al problema y al propio
resolutor.

Pero también es posible describir lo que ocurre de una manera menos
racional. Sucede, simplemente, que hemos conseguido sobrevivir con un nivel
jerárquico superior que es simbólico. Entender el mundo como un problema nos
permite sobrevivir. Cuando el nivel superior de un organismo es simbólico,
se denomina consciencia. Esto es una mera definición.

No se está diciendo que sea mejor (¿para qué?) poder representarse
problemas, ni que ello redunde en una vida más feliz. Tal vez sí. Lo que sí
es cierto es que sólo teniendo un conocimiento simbólico podemos
representarnos a nosotros mismos como resolutores del problema de la
supervivencia. Y que sólo los conocedores simbólicos nos podemos preguntar
por qué nos gobernamos por símbolos.



\chapter La habitación china

Suponga que usted no sabe una palabra de chino. Suponga que no sabe nada de
papiroflexia. Suponga que lo encierran en una habitación y le piden que por
cada hoja de papel que le pasen, devuelva otra hoja escrita por usted a
partir de la primera, pero siguiendo unas instrucciones muy precisas
escritas en castellano. Suponga que merced a esas instrucciones sus
respuestas resulten ser atinadas respuestas escritas en chino a las
preguntas sobre papiroflexia, también en chino, que están escritas en los
papeles que recibe, aunque esto lo ignora usted.  Suponga que alcanza usted
tal destreza siguiendo las instrucciones, que el tiempo que tarda en
preparar sus respuestas es el mismo que el que emplearía un hábil calígrafo
chino. Es decir, que desde fuera de la habitación, se diría que dentro se
encuentra un experto chino en papiroflexia respondiendo a las preguntas.

\person{Searle}\cite{Searle1980} propone una situación similar a ésta, con
la que intenta colocarle a usted en la situación de una computadora, para
demostrar que la inteligencia no está en el tratamiento automático de los
símbolos, como había propuesto \person{Turing}\cite{Turing1950}.
\person{Searle} tiene razón. Aunque desde fuera parezca que usted sabe
papiroflexia, y hasta chino, nosotros sabemos que usted no es capaz de saber
el significado de ninguna de las palabras que escribe.


\chapter La inteligencia

Una calculadora no es inteligente. Una calculadora sólo sirve, como tal, a
alguien inteligente. Recuérdese que la calculadora sólo ayuda a sumar a
quien ya sabe sumar.

Una computadora puede ser inteligente, por ejemplo, jugando al ajedrez. Lo
es, en ese caso, porque se enfrenta a un problema, cómo ganar la partida.
Pero incluso a algunas personas les parecerá que no lo es. Argüirán que la
computadora no sabe, en realidad, jugar al ajedrez. Explicarán que si en
aquella jugada decisiva la computadora jugó brillantemente el caballo, no
fue porque tenga una visión global del juego, sino porque la función de
evaluación que usa, dada aquella situación del tablero, devuelve
automáticamente como resultado el movimiento del caballo.

Es cierto que la computadora, por ser obra de personas, puede estudiarse
hasta determinar exactamente por qué hizo en cada momento la jugada que
hizo. Es cierto, también, que la manera de jugar y la toma de decisiones de
una computadora puede ser muy diferente de la de una persona.

La primera objeción se fundamenta en una limitación de nuestra especie, {\it
homo sapiens}, que es consecuencia de su diseño, evolutivo y no planeado.
Las personas no somos completamente conscientes de las causas de nuestros
actos.

La consciencia es sólo la capa superior, aún muy frágil y delicada, de
nuestro aparato cognitivo. Hasta \person{Freud}\cite{Freud1900} no se tuvo
conocimiento de este hecho, pero ahora se conocen datos que lo confirman
(véase, a continuación, la sección sobre ``Cerebros partidos'').

Respecto a si la forma de resolver de la computadora es tan diferente de la
nuestra que no se puede considerar inteligente, nos topamos con una cuestión
de definiciones que, como sabemos, es meramente convencional. Si convenimos
que volar es surcar los aires como las aves, entonces los aviones no vuelan.
Si convenimos que inteligencia es la manera de resolver los problemas de las
personas, entonces sólo las personas piensan.

Aquí definimos la inteligencia como la capacidad de resolver problemas.


\chapter Cerebros partidos

La corteza cerebral está dividida en dos hemisferios comunicados,
principalmente, a través del llamado {\it corpus callosum}. Uno de los
tratamientos de la epilepsia más grave consistió en cortar el cuerpo calloso
con la intención de que los ataques epilépticos no se propagasen de uno a
otro hemisferio. Las operaciones, realizadas en los años 60 en California y
en los años 70 en la costa Este de los Estados Unidos, fueron consideradas
un éxito clínico por la disminución obtenida en la severidad de los ataques,
sin merma observable en la capacidad intelectual de los pacientes.

La aparente normalidad intelectual de los pacientes es sorprendente. Se
debe, en parte, a que el tronco cerebral, que proporciona las funciones más
básicas, como por ejemplo la del despertar, no queda dividido, y en parte a
que en circunstancias normales los dos hemisferios reciben la misma
información. Sin embargo \person{Sperry} y sus colaboradores desarrollaron
técnicas que permiten controlar la información que llega a cada uno de los
hemisferios, obteniendo resultados muy interesantes. El resultado más
general es que el hemisferio izquierdo de los adultos diestros, que es quien
controla el lado derecho del cuerpo, es el único capaz de hablar.

\person{Churchland}\cite{Churchland1986} resume estos resultados. Uno
significativo es el siguiente: se hace que el hemisferio derecho de un
paciente vea una escena nevada y el izquierdo la pata de una gallina y se le
pide que seleccione, de entre una serie de ellos y con sus dos manos,
objetos relacionados. El paciente selecciona con la mano izquierda una pala
para quitar la nieve y con la mano derecha la cabeza de una gallina.
Terminado el ejercicio, la información llega a ambos hemisferios y se le
pide al paciente que explique su elección. Éste contesta que obviamente la
cabeza de la gallina va con la pata de la gallina, y que la pala es
necesaria para limpiar el gallinero.

El paciente entiende la operación que ha sufrido y la prueba realizada, y
sin embargo insiste en su explicación. Como situaciones parecidas se
repiten, y no hay evidencia alguna de que los pacientes intenten
conscientemente engañar, cabe suponer que lo que ocurre es que el hemisferio
izquierdo, que controla el habla y la consciencia, ha integrado del modo más
natural los datos con los que cuenta y con ellos ha elaborado su
explicación. Esta racionalización, urdida por las capas superiores del
hemisferio izquierdo, es considerada por el sujeto como su propia voluntad,
y por tanto no está sujeta a discusión por más evidencias que se le
muestren.

Esto, aunque a primera vista pueda parecer extraño, seguramente es lo
normal. Nuestras explicaciones de nuestros propios actos son también
racionalizaciones\cite{Gazzaniga1998}. Tienen que serlo porque no es posible
que al sustrato simbólico consciente del cerebro le llegue toda la
información producida por el cerebro, de modo que sólo le pueden llegar
informaciones parciales.


\chapter La computadora y la mosca

¿Se puede comparar la inteligencia de una mosca con la de una computadora?
La principal dificultad para efectuar esta comparación es que la mosca se
enfrenta directamente al problema de la supervivencia, mientras que la
computadora no se enfrenta a problema alguno, sino que sirve como ayuda
simbólica a la resolución de problemas.

Pero tampoco el hecho de enfrentarse al mismo problema supone que la
comparación pueda hacerse sin mayores complicaciones. Por ejemplo, que haya
más moscas que personas puede indicar que las moscas han solucionado mejor
que las personas el problema de la supervivencia y, sin embargo, pocas
personas estarían dispuestas a considerarse menos inteligentes que una
mosca.

En cuestiones importantes es más inteligente una mosca que huye en cuanto
nos acercamos con perversas intenciones, que la computadora capaz de jugar
al ajedrez mejor que nosotros, pero que reacciona del mismo modo a nuestra
sonrisa que a un ataque con martillo (no reacciona en ninguno de estos
casos). Sin embargo, nos impresiona más la inteligencia de la computadora
capaz de ganar al campeón del mundo de ajedrez.

En general, la inteligencia se asocia con la capacidad para resolver
problemas. Como nuestro razonamiento consciente es simbólico, valoramos más
la inútil, por sí misma, inteligencia simbólica de la computadora que la más
simple inteligencia autónoma de la mosca.

\chapter La rana y la mosca

La rana tiene un sistema nervioso muy primitivo. Sólo es capaz de distinguir
cosas pequeñas que se mueven con rapidez, que llamaremos moscas pero que
pueden ser también otros tipos de insectos y animales pequeños, y cosas
grandes que se mueven despacio, que llamaremos genéricamente predadores.

Una rana puede comerse una mosca, pero una mosca no puede comerse una rana,
y sin embargo, hay más moscas que ranas. Las ranas y las moscas resuelven su
problema de la supervivencia de diferente modo. Las ranas y las moscas
tienen inteligencias distintas.


\vskip3pc
\centerline{\it ¿Qué soy yo?}
\vskip2pc

\chapter L'État, c'est moi

``El estado soy yo'', declaró el rey \person{Luis XIV} de Francia. Pero,
¿qué soy yo? Mi pierna derecha no soy yo, porque hay a quien le amputan su
pierna derecha y no deja de ser él. Y, sin embargo, hay quien pierde a un
hijo y, a muchos efectos, deja de ser él. Voy a responder con precisión a la
pregunta ¿qué soy yo?, pero antes es necesaria alguna preparación (quien
algo quiere, algo le cuesta).

Si fuera malabarista, antes de conseguir realizar con éxito este ejercicio,
fracasaría un par de veces para subrayar su dificultad.

\chapter Primer fracaso

El yo no puede ser un modelo de la persona. De serlo ocurriría una regresión
infinita: la persona tiene
 un modelo del universo en el que está
  el modelo de la propia persona que tiene
   un modelo del universo en el que está
    el modelo de la propia persona que tiene
     un modelo del universo, etc.

\chapter Segundo fracaso

El yo no puede ser una cosa. El cuerpo del sujeto no es el yo. Yo soy libre,
tengo voluntad propia. Mi cuerpo actúa según mi voluntad. El yo es, por
tanto, libertad y voluntad. El yo no puede ser visto, ni oido, ni tocado. No
sabe, ni huele.


\chapter La consciencia

La persona es un organismo complejo cuyo objetivo es solucionar el problema
de la supervivencia. Está organizado jerárquicamente, de modo que al nivel
superior, que es la consciencia simbólica, sólo llegan los problemas más
importantes, y que no pueden ser solucionados en los niveles inferiores. La
tarea de la atención es asegurar que en la consciencia sólo se encuentra, en
un momento dado, un único problema. La tarea de la consciencia es solucionar
el problema que le va llegando sin que ello suponga que los problemas
previamente solucionados se queden sin solución. Veamos esto.

Supongamos que he aprendido que cuando tengo sed, lo mejor es beber agua. Y
supongamos que posteriormente, en cierto momento, descubro que el agua con
cierto olor característico sabe muy mal. Las soluciones al nuevo problema
son varias. Una drástica consiste en no tomar agua, de ningún tipo, pero
esta solución interfiere con otra anterior, aquélla merced a la cual cuando
tengo sed tomo agua. Otras soluciones menos drásticas consisten en no tomar
agua si huele de ese modo o en probar con prevención cualquier líquido con
ese olor.

La consciencia convierte la secuencia de problemas que la atención le va
presentando en un único problema. Lo hace por acumulación de condiciones, y
de este modo la solución del último problema es también solución de todos
los problemas anteriores.

Supongamos que quiere comprarse un coche, y que a usted le gustan los
deportivos. Su problema es entonces elegir un coche deportivo. Pero
supongamos que su marido, al enterarse de sus planes, le diga que le gustan
los coches rojos. Ahora el problema consiste en elegir un coche deportivo y
rojo, ya que la solución de este nuevo problema es también solución del
primero.

El peligro de este método es que la acumulación de condiciones puede hacer
que el problema resultante se quede sin solución. Siguiendo con el ejemplo,
podría ser que una tercera condición, como el dinero disponible, le dejara a
usted sin coche, al no haber coches deportivos, rojos y baratos.

El conocimiento es, básicamente, acumulativo. Cuando conozco una nueva
ciudad y sus playas, este conocimiento no sustituye a otros, sino que se
añade a los conocimientos que ya tenía, y de este modo no interfiere con los
problemas previamente solucionados, o sea, no interfiere con lo que ya sé.


\chapter Yo

La unidad de la persona se consigue en dos pasos. Primero la atención
consigue que, en un momento dado, sólo ocupe la consciencia un problema.
Segundo, la consciencia convierte la secuencia de problemas que la atención
le va presentando en un único problema. Llamamos {\em problema del sujeto} a
este único problema que va definiendo la consciencia simbólica del sujeto.
El problema del sujeto es la representación simbólica consciente del
problema de la supervivencia. El yo es la solución del problema del sujeto.

El problema del sujeto tiene una peculiaridad, a saber, que es un problema
en continua redefinición. Cada vez que la atención se centra en un nuevo
problema, la consciencia intenta expresar el nuevo problema como una
condición adicional del problema del sujeto. Esto consigue que el
comportamiento que soluciona el nuevo problema no deje sin solución a los
problemas que llegaron antes a la consciencia.

La continua redefinición del problema del sujeto obliga, en consecuencia, a
una continua redefinición de su solución, esto es, del yo. El comportamiento
consciente de la persona, en cada momento, resulta de la solución que en ese
instante encuentra a su problema del sujeto: ``por sus obras los
conoceréis''. Sin embargo, para el propio sujeto, no es la información de
los comportamientos pasados lo que marca sus comportamientos futuros, sino
cómo queda reformulado su problema del sujeto. Porque es de éste, con las
condiciones adicionales que se encuentre en el futuro, de donde resultarán
los comportamientos posteriores. Por esta razón tiene sentido usar el
problema del sujeto para referirse perifrásticamente a su solución, que es
el yo, y no tiene sentido identificar el yo con el comportamiento consciente
del sujeto, aunque éste sea su solución actual.

Esto puede explicar que, tal vez, lo que a los ciudadanos nos parece un
cambio de comportamiento innoble, al político le resulte natural. `Yo sigo
siendo el mismo, es el mundo quien ha cambiado'. Los cuidadanos observamos
su comportamiento, pero desconocemos su problema, y por esta razón ``las
apariencias engañan''.

El {\em yo} es la solución del problema que, al resolverse, determina el
comportamiento consciente del sujeto, pero no cristaliza como
comportamiento, sino que permanece definido como problema. Por esto, el yo
es libertad condicionada, siendo los valores y las creencias las condiciones
que lo definen por delimitación. Mi yo me representa, pero no es mi modelo,
porque yo soy la libertad, la incógnita, de mi versión simbólica y
consciente del problema de la supervivencia. ¿Qué soy yo? Yo soy libertad
para no morir.

\vfil
\centerline{\it Yo soy libertad para no morir}
\vfil
\break

\chapter Creencias y valores

Como la acumulación de condiciones puede dejar al problema del sujeto sin
solución, existen medios para evitarla. Uno consiste, simplemente, en
establecer condiciones sobre las propias condiciones. Por ejemplo, el
problema de cómo ir a la playa tiene distintas condiciones dependiendo de la
ciudad en donde me encuentre, lo cual quiere decir que hay una condición de
rango superior que es la que establece qué condiciones son pertinentes.

Las definiciones pueden simplificar las explicaciones, de modo que
introduciremos dos. Llamaremos creencias a las condiciones de orden inferior
que definen el problema, simbólico, del sujeto. Así, la impenetrabilidad de
las paredes es una de mis creencias. A las condiciones de orden superior,
que son condiciones de condiciones, las denominaremos valores. Por esto,
cuando veo una película de fantasmas, cambio mis valores y me creo que las
paredes pueden ser atravesadas.


\chapter El lenguaje del pensamiento

Con esta organización del sujeto, la comunicación más eficiente entre
sujetos se consigue comunicando aquello que alcanza la consciencia y
utilizando la misma sintaxis que utiliza la consciencia del sujeto. Las
razones son dos, a saber, que los problemas, soluciones y resoluciones más
importantes para el sujeto son aquéllos que alcanzan la consciencia, que es
su nivel jerárquico superior, y que usar la misma sintaxis evita una
traducción.

\chapter El sujeto

Un sujeto es un conocedor simbólico capaz de enfrentarse a un problema de la
supervivencia complejo.  Para ello dispone de una estructura de resolución
jerárquica a cuya cúspide denominamos consciencia.  El objetivo de la
consciencia es dar unidad al sujeto. Por esto, aunque el sujeto tiene un
pensamiento básicamente paralelo, únicamente tiene consciencia de una
secuencia simbólica de pensamiento. La herramienta de la consciencia es el
yo.

El subjetivismo parte del sujeto, porque sólo el sujeto es capaz de conocer,
de preguntarse por qué, pero va más allá. El sujeto es la atalaya desde la
que se puede ver la totalidad del territorio, pero no es todo el territorio.

\goodpage

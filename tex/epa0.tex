% EPA0.TEX (RMCG19950910)

\pdfcode%\begingroup\stringactives
 \pdfinfo{
  /Author (Ramón Casares)
  /Title (El problema aparente)
  /Subject (Teoría subjetivista del saber (epistemología))
  /Keywords (problema conocimiento evolución simbolismo autómata
   lógica adaptación aprendizaje semántica sintaxis)}
 \pdfannot width 6cm {/Subtype /Text
  /Contents (Ramón Casares (www.ramoncasares.com))}
%\endgroup
\pdfendcode

\null % [1] Portada

\nopagenumbers

\font\titlefont=texnansi-lmssbx10 at 70pt
\font\subtitlefont=texnansi-lmss10 at 40pt
\font\authorfont=texnansi-lmss10 at 50pt

\pdfliteral{q}% save current graphic state in the stack
%\pdfliteral{0 1 1 0 k 0 1 1 0 K}% set red
%\pdfliteral{1 1 0 0 k 1 1 0 0 K}% set blue
%\pdfliteral{1 0 1 0 k 1 0 1 0 K}% set green
\pdfliteral{0 0.49 1 0.52 k 0 0.49 1 0.52 K}% set cinnamon
\pdfliteral{-72 -576 m}% moves to the origin, 72 * 8 = 576
\pdfliteral{360 -576 l}%  72 * 5 = 360
\pdfliteral{360 72 l}%
\pdfliteral{-72 72 l}%
\pdfliteral{-72 -576 l}% 
\pdfliteral{b Q}% close, stroke, fill, and restore graphic state

\pdfWhite

\hbox{}
\vskip1.2in
\centerline{\titlefont El problema}
\vskip1pc
\centerline{\titlefont aparente}

\vskip10pc
\centerline{\subtitlefont Una teoría del}
\vskip6pt
\centerline{\subtitlefont conocimiento}

\vskip2pc
\centerline{\authorfont Ramón Casares}

\pdfBlack

\vfil

\break % [2] Créditos versión digital

\null \vfill

 Este libro ha sido tipografiado por el autor\par
 usando el sistema del Profesor D.~E.~Knuth (Stanford University).\par
 He utilizado {\em su} programa \TeX\par
 para componer {\em mi} texto,\par
 con {\em sus} tipos Computer Modern,\par
 y para colocar {\em mis} figuras,\par
 que hice con {\em su} programa \METAFONT.\par

\vglue 2pc

Esta versión digital es fiel a la publicada en papel,\par
excepto en los siguientes puntos:\par
esta página, la anterior y la penúltima son nuevas;\par 
en toda esta versión el símbolo `$\prl$' es visible; y\par
la paginación varía, aunque nunca en más de una página.\par

\vglue 2pc

\def\epaversion{\the\year.%
 \ifnum\month<10 0\fi \the\month.%
 \ifnum\day<10 0\fi\the\day:%
  \count255=\time \divide\count255 by 60
 \ifnum\count255<10 0\fi \the\count255.%
  \multiply\count255 by 60 \advance\count255 by -\time
  \multiply\count255 by -1
 \ifnum\count255<10 0\fi \the\count255}

 {\lmtwo El problema aparente}\par
 {\sl Una teoría del conocimiento}\par
 Versión digital: \epaversion \par
 \copyright\ {\bf Ramón Casares} 1999, 2010\par
 \leavevmode \pdfcode \pdfstartlink attr{/Border [0 0 0]}
   user{/Subtype /Link /A << /Type /Action
    /S /URI /URI (http://www.ramoncasares.com/) >>}\pdfendcode
  {\tt www.ramoncasares.com}\pdfcode \pdfendlink \pdfendcode \par

\break % [3] Colección

\centerline{\lmxiibf V{\lmviiibf\kern-1pt ISOR}
        Lingüística y conocimiento}\newpage % [4] Blanca


\null\newpage % [5] Título y subtítulo

\centerline{\lmxiibf El problema aparente}
\centerline{\bf Una teoría del conocimiento}
\newpage %  [6] Blanca
\null\newpage % [7] Autor, título y subtítulo

\centerline{\lmone Ramón Casares}
\vskip 9pt
\hrule height 1pt
\vskip 12pt
\centerline{\lmxiizero El problema aparente}
\vskip 1pc
\centerline{\lmxiione Una teoría del conocimiento}
\vfill \break % [8] Créditos

\centerline{\lmxiibf Lingüística y conocimiento -- 27}
\vglue6pc
\centerline{Colección dirigida por}
\centerline{Carlos Piera}
\vskip4pc
\centerline{Primera edición, 1999}
\vfil
\centerline{\copyright\ Ramón Casares, 1999}
\centerline{\copyright\ De la presente edición:}
\centerline{\sc Visor Dis., S.A., 1999}
\centerline{Tomás Bretón, 55}
\centerline{28045 Madrid}
\centerline{\pdfcode \pdfstartlink attr{/Border [0 0 0]}
   user{/Subtype /Link /A << /Type /Action
    /S /URI /URI (http://www.visordis.es/) >>}\pdfendcode
  {\tt www.visordis.es}\pdfcode \pdfendlink \pdfendcode}
\smallskip
\centerline{ISBN: 84-7774-877-2}
\centerline{Depósito Legal: M.\ 28.094-1999}
\smallskip
\centerline{Impreso en España -- \it Printed in Spain}
\break % [9] Dedicatoria y agradecimiento

\vglue 0pt plus 1fil

\rightline{\it A Pili}

\vskip5pc
\rightline{\it Fernando Sáez Vacas lo comenzó}
\rightline{\it y lo finalizó Carlos Piera.}
\rightline{\it Muchas gracias.}
\rightline{\it R.C.}

\vfil
\vfil

\break % [10]

\footline={\tenrm\ifodd\pageno \hfil\folio \else \folio\hfil \fi}
\footline={\tenrm\ifodd\pageno \docinfo\hfil\folio
            \else \folio\hfil\infodoc \fi\strut}
\endinput


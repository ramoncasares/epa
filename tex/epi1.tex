% EPI1.TEX (RMCG19980422)

\title0 I. Introducción

\prolog Reflexiones subjetivistas sobre\cr problemas y símbolos
        (PROBLEMAS Y SÍMBOLOS)

\chapter ¿Sabe sumar una calculadora?

La calculadora hace las sumas para nosotros. Cuando queremos saber lo que
nos costará la compra, sumamos con la calculadora el precio de cada uno de
los productos elegidos. Si tenemos una calculadora es porque {\em sabe}
sumar. Si la calculadora no supiese sumar, ¿por qué razón íbamos a tenerla?
En lo que sigue intentaremos rebatir esta opinión.

Si la calculadora supiera, efectivamente, sumar, entonces la calculadora
sería una materialización del concepto `suma'. Supongamos, sin embargo, que
le damos la calculadora a alguien que desconoce la notación árabe de los
números y los convencionalismos matemáticos que asignan el signo `+' a la
suma y el signo `=' a la igualdad, aunque sepa sumar usando otro simbolismo.
Seguramente la calculadora no le serviría para sumar, lo que prueba que o
bien la calculadora no es la suma hecha materia o, cuando menos, que el
concepto puro de suma no puede ser usado directamente.

Visto de otro modo, el funcionamiento de la calculadora podría describirse
así: si aprieto la tecla a la que doy el significado dos (2), después la
tecla que interpreto como suma (+), luego vuelvo pulsar la tecla
correspondiente al dos (2), y por fin aprieto la tecla asignada a la
igualdad (=), entonces se iluminan en la pantalla una serie de puntos que
interpreto como cuatro (4). Es decir, que en este otro modo de ver las
cosas, la calculadora es un ingenioso interruptor que enciende unas u otras
luces según cual haya sido la secuencia de teclas operada.

\vskip0pt plus3pc
$$2+2=4$$
\vskip0pt plus3pc
\break

El funcionamiento de la calculadora en el ejemplo anterior se puede anotar
así: $2 + 2 = 4$. Las teclas pulsadas son aquéllas que aparecen hasta el
`=', incluida ésta, y lo que sigue son luces en la pantalla. Es decir, que
si somos capaces de expresar un problema en la forma $x? a + b = x$, siendo
$a$ y $b$ dos cantidades conocidas, entonces podemos solucionarlo con la
ayuda de la calculadora.

Todo esto nos descubre que la calculadora sólo es útil a aquellas personas
que son capaces de expresar algunos de los problemas que se encuentran del
modo que la calculadora está preparada para resolver. Lo primero es saber si
el problema es de los que la calculadora puede ayudar a resolver, pero
averiguar esto, sin ir más lejos, es un problema que la calculadora no ayuda
a resolver.

Es curioso que la calculadora sirva, exclusivamente, a quienes ya saben
sumar. Porque para ser capaz de expresar que la solución de un problema es
el resultado de sumar unas cantidades, y no de restarlas, hay que tener
previamente el concepto de suma. Con lo que concluimos que el concepto de
suma no está en la calculadora, sino en quien usa la calculadora.

\chapter ¿Para qué sirve una calculadora?

Que se vendan calculadoras, aunque sólo sean útiles a quienes ya saben
sumar, parece un contrasentido, pero ya sabemos que no lo es. La calculadora
efectúa una suma concreta.

Nosotros podemos efectuar algunas sumas en la cabeza, dependiendo de nuestra
habilidad y del tamaño de la suma. Otras preferimos hacerlas con la ayuda de
papel y lápiz. En mi caso, escribo las cantidades a sumar una sobre la otra,
alineadas por la derecha, usando la numeración árabe decimal. Luego, de
derecha a izquierda, voy sumando los dígitos llevando la cuenta de los
acarreos. Es éste un procedimiento de tratamiento puramente simbólico del
que resulta un número árabe decimal que es la expresión simbólica de la
cantidad que busco. Si el número representa dinero o tornillos es algo que,
este sistema de representación, pierde. La calculadora automatiza este
proceso, pero hace fundamentalmente lo mismo que yo con mi papel y mi lápiz,
manipula símbolos para ayudarme a efectuar una suma concreta de cantidades
conocidas.
\strut
\setbox0=\vbox to 0pt{\vss
 \halign{\strut\hfil#\hfil&&\kern3pt\hfil#\hfil\cr
   &\it 1&&&\cr
   & 3& 9& 2\cr
   &  & 1& 5\cr
  +& 1& 8& 1\cr
  \noalign{\kern-0.4pt\hrule}%
   &\it 5&\it 8&\it 8\cr}\copy\strutbox}%
\dimen0=\wd0 \advance\dimen0 by 1pc
\vadjust{\kern-\dp\strutbox\box0}%
\hangindent=\dimen0
\hangafter=7

Saber sumar tiene, pues, dos aspectos. Saber que es precisamente una suma lo
que soluciona un determinado problema, y saber hacer la suma concreta que el
problema plantea. Ocurre que si lo que falta para solucionar un problema es
efectuar la suma de dos cantidades conocidas, entonces podemos dar por
solucionado el problema porque, en caso de que no la podamos hacer
directamente en la cabeza, sabemos que existen modos de efectuar
mecánicamente la operación.

La calculadora hace sumas, pero ignora completamente el concepto de suma.
Saber hacer sumas sirve porque hay muchos problemas, o partes de problemas,
que pueden convertirse en una suma.

\goodpage
\chapter ¿Cómo se usa una calculadora?

En la resolución de un problema con la ayuda de una calculadora hay tres
fases:
\point En la primera se expresa simbólicamente el problema de manera
que pueda resolverse con la calculadora.
\point En la segunda se efectúa la operación en la calculadora.
\point En la tercera se interpreta el resultado obtenido en la
calculadora.

Muchas veces lo que muestra la calculadora en su pantalla nos parece la
solución buscada. Sin embargo, casi nunca\footnote*{Nunca digas nunca.} es
la luz emitida por la pantalla de la calculadora lo que soluciona el
problema. Ni tampoco esa luz entendida como una secuencia de dígitos. Y sólo
en problemas puramente matemáticos es la solución el número que resulta de
leer la secuencia de dígitos. En general se necesita una interpretación {\em
semántica} del resultado simbólico.

En muchas ocasiones también ignoramos que hemos realizado una reformulación
simbólica del problema. Pero el único momento en el que no hacemos esta
reformulación es cuando, siguiendo las instrucciones del manual de la
calculadora, pulsamos las teclas tal y como se indica. El propósito del
manual es que nos familiaricemos con el tipo de simbolizaciones, es decir,
con la {\em sintaxis}, que la calculadora acepta y usa al mostrar sus
resultados. Para ello, en general, se nos presenta la operación en una
sintaxis de uso común y, a la vez, como la secuencia de teclas a pulsar.
Como las calculadoras se diseñan con el propósito de facilitar su uso, se
procura que la traducción, de una forma a la otra, sea la mínima posible o
nula. En el caso de las calculadoras que usan la notación polaca inversa, es
patente la reformulación sintáctica del problema, pero sólo hasta que nos
acostumbramos a usarlas.

El uso de la calculadora es un ejemplo de una situación mucho más general.
Nuestro lenguaje, incluido éste en el que estoy escribiendo, es un lenguaje
simbólico. Nuestro pensamiento consciente es también simbólico. Ambos usan
símbolos, que son etiquetas asociadas convencionalmente a objetos,
organizados según convenciones sintácticas. Y sin embargo, ninguna de
nuestras necesidades primarias es simbólica. Pensar no alimenta. Sucede,
como con la calculadora, que la simbolización es sólo una manera más eficaz
de alcanzar la solución de nuestros problemas, pero que necesita para ser
efectiva de un proceso previo de reformulación sintáctica del problema y,
posteriormente, de una interpretación semántica de la solución.

\goodpage
\chapter El diccionario

El diccionario explica todas las palabras del idioma. Y las explica usando
palabras. Como normalmente conocemos el significado de muchas de ellas, el
diccionario nos es útil. Pero cuando usamos el diccionario de un idioma que
estamos aprendiendo, es frecuente que al explicar una palabra el diccionario
use otras palabras cuyo significado también desconocemos, y en este caso el
diccionario no nos ayuda.

Si escribiéramos cada palabra del diccionario en una hoja enorme de papel, y
dibujáramos una flecha desde cada palabra a cada una de las palabras que
aparecen en la definición de la primera, obtendríamos una red que nos
serviría para saber que palabras es necesario conocer para conocer la
primera. Suponiendo que en las definiciones se usaran diez palabras, que es
una suposición conservadora y que facilita los cálculos, entonces de la
palabra saldrían diez flechas, y de cada una de las diez alcanzadas saldrían
otras diez, con lo que al segundo salto alcanzaríamos cien palabras. Según
estos cálculos, con seis saltos se llegaría a un millón de palabras, lo cual
es imposible ya que no hay tantas palabras diferentes en el idioma. Lo que
sucede, naturalmente, es que pronto empezarían a repetirse palabras, esto
es, llegaríamos hasta ellas por varios caminos.

También podemos formar secuencias, o cadenas, de palabras eligiendo, como
siguiente a una palabra, la que figura de primera en la definición de
aquella palabra. Así, por ejemplo, tomando el diccionario de la
Academia\cite{RAE1970}, y partiendo de la palabra {\em problema} obtenemos
la siguiente cadena: problema, cuestión, pregunta, demanda, súplica, acción,
ejercicio, acción, ejercicio, etc. Este ejemplo nos descubre que la
definición de {\em ejercicio}, ``acción de ejercitarse u ocuparse en una
cosa'', es deficiente, y más cuando define {\em ejercitar} como ``dedicarse
al ejercicio de un arte, oficio o profesión''. Es deficiente porque si
alguien consulta la palabra {\em ejercicio} es probable que desconozca
también la palabra {\em ejercitar}, que es de la misma familia, por lo que
las definiciones dadas no le permiten salir de su ignorancia.
%
$$\hbox{{\bf problema.} Cuestión que se trata de aclarar [\dots].}$$

No es mi propósito, sin embargo, criticar a la Academia, sino mostrar la
imposibilidad de alcanzar el significado de las palabras usando únicamente
palabras. Porque es notorio que el diccionario, como todo sistema simbólico,
forma un sistema cerrado que, en sí mismo, no contiene significado alguno.
En resumen, el diccionario sólo sirve a quien ya conoce el idioma.

%\vfil
%\centerline{{\bf problema.} Cuestión que se trata de aclarar [\dots].}
%\vfil
%\break

\chapter Un cuento

Una vez un rey se preguntó quién sería el más tonto del mundo, tal vez
porque tuviera interés en conocerlo, o tal vez porque quisiera abdicar en
él. (No me gusta.) (No seas suspicaz, \person{Piripili}. La verdad es que no
va por ti. Sigo.) El caso es que antes de intentar encontrarlo, y aunque
todo el mundo cree saber con claridad lo que es una persona tonta, (Perdona
que te interrumpa, pero dime, ¿vivías tú cuando aconteció todo esto, o
todavía no habías nacido?) (Yo no vivía aún, pero ¿qué importa eso? Como te
iba diciendo, aunque todo el mundo cree saber lo que es una persona tonta),
el rey pidió a los sabios de su reino que le hicieran una descripción de la
persona que debería ser buscada. (Yo podría hacerlo mejor que esos sabios,
¿me dejas intentarlo?) (Vaya, \person{Piripili}, parece que hoy estás
lúcida.) (La verdad es que tengo la ventaja de vivir en otros tiempos (je,
jé).) (¿Quieres decir que te aprovecharás de la experiencia filosófica
acumulada desde los tiempos de esta historia?) (Quiero decir que ahora puedo
señalar al más grande tonto de todos los tiempos porque, para desgracia de
todos los que ahora vivimos, el más grande tonto de todos los tiempos es
nuestro contemporáneo y, además, el más grande tonto de todos los tiempos
eres tú.) (Estoy dispuesto a aceptar eso si me lo justificas debidamente.)
(Yo creo que la persona más tonta de todos los tiempos es aquélla que es
capaz de aceptar que él mismo es el más tonto de todos los tiempos si se lo
prueban debidamente.) (Vale, pero aunque reconozco que lo que te he dicho es
una gran tontería, no me reconozco el más tonto de todos los tiempos.) ((No
puedo desaprovechar esta ocasión (ja, já).) Puesto que reconoces haber dicho
una ``gran tontería'' (ji, jí), eres un gran tonto y, según lo veo, con
grandes posibilidades de ser el más tonto del mundo y aun el más tonto de
todos los tiempos.) (Ya vale, ¿quieres que continúe?) (Sí, (ju, jú) es el
mejor de los cuentos que me has contado, sigue, por favor.)

Un sabio de los más sabios habló así: ``El más tonto habrá de ser aquél que
no sea capaz de distinguir, siquiera, las dos cosas más dispares, o sea,
aquél que todo lo confunda. Así será incapaz de cualquier discernimiento''.

Pero entonces otro sabio, que era famoso porque, aunque él nunca había
formulado una teoría satisfactoria, tenía una extraordinaria habilidad para
derribar teorías (incluídas las suyas, por lo visto), contestó: ``De ser así
el más inteligente, o el menos tonto, habría de ser aquél que todo lo
distinguiera, que tuviera un absoluto discernimiento. Sin embargo, conozco
el caso de un individuo con tal discernimiento y parece bastante confuso''.
Y les contó la historia que sigue.

Lo que voy a relataros trata de alguien que en un principio fue tomado por
una persona de una preclara inteligencia, pero que más tarde, al transcurrir
el tiempo, fue considerado tonto. Y, sin embargo, el paso de uno al otro
estado fue lento, ininterrumpido y razonable. Ahora, {\it a posteriori}, hay
quien dice que siempre estuvo loco. Por el contrario, otros opinan que
siempre, desde el principio y hasta ahora, ha sido un sabio, pero que nadie,
ni entonces ni ahora, es capaz de comprender sus más profundas conclusiones.

Pero no quiero hablaros más del asunto sin deciros, más concretamente, de
que se trata. Así que pasaré a contaros, comenzando por el principio, y
hasta su desconcertante final, la historia de esta persona.

Mas ahora que debo remontarme al comienzo, recuerdo que ya entonces se le
consideró de una manera especial. Porque el niño nunca llegó a comprender la
razón por la cual se utilizaba el mismo nombre para designar, pongamos por
ejemplo, a dos margaritas, cuando a él le parecía evidente que eran
diferentes, o más exactamente, absolutamente diferentes.

Claro está que al principio la cosa fue algo más confusa. Le enseñaban la
flor y le decían `margarita', y todo iba bien si le preguntaban cómo se
llamaba aquella misma flor, mas si le mostraban otra margarita no sabía qué
contestar. Le mostraban de nuevo la primera y respondía rápidamente
`margarita', pero si volvían a la segunda decía que nunca le habían dicho
cual era su nombre.

Pasado algún tiempo llegó a comprender el significado de `igual', aunque,
según él lo veía, sólo debería ser aplicado a entes abstractos, a
construcciones mentales, y nunca a objetos, ya que entre éstos últimos no
había dos que fueran iguales. Estudió el concepto de igualdad con otros
sabios. Él los convenció de que era incontestable que no podían existir dos
objetos iguales. Siempre se podría encontrar alguna diferencia, en último
caso no podían ocupar ambos el mismo lugar en el espacio.

Los sabios le explicaron que la traslación en el espacio era una operación
que, por definición, conservaba la igualdad. Respecto a la diferencia que
seguía encontrando entre cada par de cosas, se le dijo que no se podía tener
una palabra para cada objeto, porque entonces sería imposible aprender y
recordar todas las palabras. Así, a aquellos objetos cuyo centro era redondo
y amarillo y su periferia una colección de hojillas de color blanco, se les
denominaba margaritas.

Él no pudo dar por buena aquella explicación porque, entrando en mayores
profundidades, hizo notar a los sabios que, por ejemplo, él seguía sin
entender qué era una hojilla, y por qué objetos tan diferentes como los que
le habían mostrado como ejemplos de hojilla podían compartir el mismo
nombre. Concluyó que la única solución válida al problema era enumerar
exhaustivamente todos los objetos cuyo nombre era hojilla o era margarita.

Tal procedimiento extensivo no era del agrado de los sabios por motivos
estéticos, además de ser impracticable. Aun así tuvieron que convenir que
era el teóricamente más correcto. Para contrarrestarlo idearon otro esquema.
Todos los objetos poseen un conjunto de propiedades medibles. Pues bien,
parece posible agrupar en un solo nombre a los objetos que tienen valores
cercanos de dichas propiedades. Se definirían objetos prototípicos y los
objetos reales tomarían el nombre del prototipo más cercano según una
métrica definida sobre el espacio de características antes definido.

Los sabios estaban satisfechos, y él también, así que les pidió a los sabios
que le desarrollaran el esquema de manera que él mismo pudiera utilizarlo
prácticamente. Los sabios acometieron con entusiasmo la empresa pero,
conforme avanzaban, la construcción se iba haciendo más y más compleja, y
aun así descubrían objetos que quedaban mal nombrados. Finalmente los sabios
tuvieron que darse por vencidos. A veces el objeto no era suficiente para
merecer un nombre adecuado, y su entorno y el interés del hablante eran más
importantes en la denominación del objeto. Otras veces el comportamiento
temporal del objeto era más importante que su proyección estática. Estas y
muchas otras dificultades que encontraron fueron las causantes de la ruina
del proyecto, fracaso que dió inmensa fama al extraño individuo que había
propuesto el problema.

A medida que fue perfeccionando su memoria, se fue percatando de que lo que
antes era, ahora ya no lo era. Es decir, que al pasar el tiempo las cosas
cambian, de modo que el objeto que en el instante actual se encuentra en una
determinada posición del espacio era siempre diferente al objeto que se
encuentra en el mismo lugar en el instante siguiente. Y aún más, que ni en
otros lugares ni en otros momentos encontraba un objeto igual. Así que un
objeto sólo merecía su nombre durante el instante en el que existía, ni
antes ni después.

Merced a su completo discernimiento dejó de ver continuidades. La realidad
de un instante y la del instante siguiente se le aparecían absolutamente
diferentes. Se sorprendía cada vez. Cada instante ocurría de nuevo la
creación de todo el universo.

Pero lo más grave fue lo relativo a su autoconsciencia. Conforme más
consciencia de sí mismo ganaba, más se destruía su consciencia. Esta
paradoja se explica porque cuanto más se veía a sí mismo como una parte del
universo, más se asemejaba él mismo al resto del universo. Así su
personalidad estalló de golpe al convertirse en una sucesión inconexa de
yoes. Sucesión que `él', al no existir más que un instante cada vez, era
incapaz de contemplar.

Al menos ésta es la explicación que dieron algunos de los sabios que lo
habían conocido en sus mejores tiempos. Aparentemente su estado era de
imbecilidad completa, pero bien pudiera ser que su yo se hubiera disuelto en
el universo incomprensible. O que hubiera alcanzado el nirvana o el éxtasis.

(Y bien, ¿qué hizo el rey?, ¿cómo termina el cuento?)

\vfil

\chapter Papá, ¿por qué respiramos?

{\obeylines
--- Papá, ¿por qué respiramos?
--- Porque así se oxigena nuestro cuerpo.
--- Y, ¿por qué se oxigena el cuerpo?
--- Porque necesita oxígeno para quemar azúcares
--- Ya, pero ¿por qué quema azúcares?
--- Pues, porque necesita energía.
--- Pero, ¿por qué necesita energía?
--- Para movernos.
--- Y, ¿por qué nos movemos?
--- Porque queremos.
--- Vale, pero ¿por qué queremos?
--- Porque sí.
}


\chapter Señalar

Las palabras más sencillas son aquéllas que designan directamente objetos,
como por ejemplo la palabra `margarita'. Parece que basta señalar una
margarita para explicar lo que es.

El acto de señalar parece primario y directo, pero no lo es. Pruébese, por
ejemplo, a señalar una estrella determinada a alguien que no esté
familiarizado con el cielo nocturno. No es una tarea fácil, lo que demuestra
que lo señalado con el dedo no queda determinado con demasiada exactitud.

Señalar sirve para discriminar entre objetos, pero no para definir. Si no
tengo la más remota idea de lo que es una piedra, y para explicármelo
alargan la mano con el dedo índice extendido, puedo pensar que se trata del
dedo índice, o de la uña del dedo de quien me lo muestra, o de la dirección
noroeste a la que apunta, o del suelo en donde está la piedra, o de la
ribera del río en la que se halla, o de la piedra.

Por otra parte, al señalar un objeto es completamente imposible determinar
si se refiere concretamente al objeto señalado o a alguna de las clases en
las que ese objeto puede quedar clasificado. Es decir, es imposible
determinar si me están señalando un `canto rodado' o una `piedra'. Para
deshacer esta ambigüedad hay que utilizar palabras, pero ya sabemos que, en
última instancia, las palabras tampoco sirven para definir palabras.

\chapter Nombres comunes

¿En qué se diferencia una piedra de una roca o de una arena? En el tamaño.
Si lo pensamos un poco, nos daremos cuenta de que estos tamaños tienen al
hombre como medida. Una piedra es un objeto duro y de un tamaño manejable y
que, por lo tanto, nos puede servir para abrir una nuez. Una roca no nos
sirve porque es tan grande y pesa tanto que no somos capaces de moverla. Una
arena es, en cambio, tan pequeña que apenas nos sirve individualmente para
nada, aunque en montones podemos usarla como asiento. ``El hombre es la
medida de todas las cosas'', ya lo decía
\person{Protágoras}\cite{Barrio1977}.

Cuando definimos algo por sus propiedades, lo que estamos haciendo es
plantear un problema. Al decir que la piedra es una cosa, a la vez, dura y
manejable, estamos planteando el problema de encontrar aquellas cosas que
cumplen estas dos condiciones, la de ser duras y la de ser manejables.
Podemos expresarlo así:
 $x? (x \hbox{ es dura}) \land (x \hbox{ es manejable})$.
El nombre común `piedra' se usaría para referirse a cualquiera de las
soluciones de dicho problema.

Parece entonces interesante convenir que el nombre común es, también, el
nombre del problema cuyas soluciones son cada uno de los ejemplares de dicho
nombre común. Porque
 $$\hbox{piedra} \;=\;
   x? (x \hbox{ es dura}) \land (x \hbox{ es manejable}) $$
puede leerse fácilmente así: una piedra es~($=$) una cosa~($x$) que es dura
y~($\land$) que es manejable. El nombre común es la manera de usar un
problema para referirnos en perífrasis o, lo que es lo mismo, usando un
rodeo o circunloquio, a sus soluciones.

Así que cuando en una frase aparece un nombre común, como `piedra' o
`margarita', se está planteando, resolviendo y solucionando un problema. Que
tratemos los nombres comunes con una total facilidad y sin esfuerzo, nos
impide percatarnos de la poderosísima maquinaria de resolución de problemas
que se esconde en nuestro cerebro.

\chapter ¿Qué es un problema?

Cuando pensamos en un problema, véase \person{Polya}\cite{Polya1957},
pensamos en un enunciado, que expone unas condiciones y proporciona unos
ciertos datos, en un conocimiento de la materia que permite resolver ese
tipo de problemas, y en un conjunto de soluciones posibles.

Esto, por supuesto, se aplica perfectamente a los problemas académicos
típicos de las matemáticas o de la física. Los problemas que nos encontramos
fuera del colegio tienen sólo un cierto parecido con un problema académico.
La mayoría de las veces no hay un enunciado explícito, e incluso en
ocasiones lo más complicado del problema es percatarse de que lo es. Así, es
posible ir modificando el enunciado conforme resolvemos el problema. Y
aunque, en general, las soluciones aceptables están mejor definidas que los
enunciados, tampoco es siempre así.

Si abstraemos al máximo, para alcanzar el concepto más puro de problema, nos
quedamos con cierta libertad, ya que tiene que haber varias posibilidades
para que haya problema, y con cierta condición que establece qué vale como
solución y qué no vale. Ni los datos, ni el conocimiento quedan incorporados
a la definición. Ambos son información que facilita la resolución del
problema, pero hay problemas en los que no se dispone de ella, siendo un
ejemplo conspicuo el problema aparente.

Llegamos así a la definición de problema. Un {\em problema} es una condición
puesta a cierta libertad. Sin libertad no hay problema. La {\em solución}
del problema es aquel uso de la libertad que satisface la condición. La {\em
resolución} del problema es el proceso que discurre desde el punto en el que
se tiene el problema, con su condición y su libertad no ejercitada, hasta el
momento en el que se tiene la solución, o sea, hasta el momento en el que se
ejerce la libertad y se satisface la condición. Es importante hacer la
distinción entre solución y resolución; resolver es a buscar como solucionar
es a encontrar.

\chapter El simbolismo

Merced a los simbolismos podemos representarnos problemas, resoluciones y
soluciones. El simbolismo sirve para la resolución de problemas. Para ello
es menester que cumpla tres condiciones:
\point Que el problema pueda representarse simbólicamente,
es decir, que se pueda expresar sintácticamente el problema.
\point Que el simbolismo disponga de medios suficientes para permitir
la resolución del problema ya expresado sintácticamente.
\point Y que la solución simbólica encontrada se corresponda con alguna
solución del problema, es decir, que la solución simbólica admita una
interpretación semántica, o sea, que tenga significado.


\chapter La solución

Asignando un nombre simbólico a cada posible solución, cumplimos la tercera
de las condiciones.

Cuanto menor sea el trecho que va del dicho al hecho, mayor será el
significado del dicho. Las frases imperativas como, por ejemplo, ¡escapa!,
tienen un mayor contenido semántico porque corresponden casi directamente a
la solución de un problema. Nótese que, además, suelen tener una sintaxis
rudimentaria. Las frases que se dirigen a los animales son de este tipo.

\vfil
\centerline{$\hbox{\rm Problema}\longrightarrow
 \hbox{\rm Resolución}\longrightarrow \hbox{\rm Solución}$}
\vfil\break


\chapter El problema

Para representar un problema hay que representar sus dos componentes: la
libertad y la condición.

La libertad del problema es un nombre simbólico, al que denominaremos
incógnita, de significado indefinido. Es esencial que la incógnita no tenga
significado, es decir, que no se refiera a ninguna solución concreta, porque
de lo contrario no habría problema. La incógnita, mientras lo es, es un
nombre sin significado, es un puro artefacto sintáctico. Siendo la incógnita
un símbolo solamente sintáctico, entonces el problema, que es una expresión
que integra condiciones e incógnitas, ha de ser también sintaxis sola.

Las relaciones entre objetos son condiciones. Las condiciones se pueden
combinar de acuerdo a las reglas de la lógica\cite{Quine1940} o del álgebra
de \person{Boole}\cite{Boole1854}. Hay tres formas básicas de combinar
condiciones, a partir de las cuales es posible expresar cualquier otra
combinación de condiciones:
\point La negación (notada $\lnot$)
de una condición es otra condición que se cumple, exactamente, cuando no se
cumple la primera; si alguien dice que {\em no} quiere un helado de
chocolate, entonces cualquier helado cumple la condición, excepto el de
sabor a chocolate.
\point La disyunción (notada $\lor$)
de dos condiciones es otra condición que se cumple si se cumple alguna de
las dos primeras, o ambas; si lo que quiere es un helado de limón {\em o} de
fresa, está diciendo que un helado de fresa cumple la condición, que un
helado de limón también la cumple, y que incluso uno de limón y fresa la
cumple, los tres le valen.
\point La conjunción (notada $\land$)
de dos condiciones es otra condición que se cumple sólo si se cumplen las
dos primeras; si, por fin, lo que quiere es un helado de limón {\em y} de
fresa, entonces sólo cumple la condición un helado que tenga los dos
sabores, y no le vale un helado de limón solo.

\vfil
\centerline{$\hbox{\rm Problema}
 \left\lbrace\vcenter{\hbox{Libertad}\hbox{Condición}}\right.$}
\vfil\break


\chapter La resolución

La resolución simbólica del problema es un proceso que, tomando la expresión
sintáctica de un problema, la va transformando en otras expresiones hasta
que llega a una expresión que es un problema de solución conocida.

Por ejemplo, supongamos que quiero tener la seguridad de que aunque hoy me
coma dos manzanas, me quedarán otras dos manzanas para mañana, ¿cuántas
manzanas he de tener? La incógnita, $x$, es el número de manzanas que
preciso. El problema se puede expresar así: $x?x-2=2$, es decir, es el
problema de hallar aquel número de manzanas tal que, si me como dos, me
quedan dos.

En el problema propuesto, la incógnita, que representa la libertad, lo
indefinido, es $x$. La condición queda estabecida al tener que cumplirse la
igualdad, que es una relación, de dos cantidades.

%\goodpage
La resolución podría ser como sigue:
$$\eqalign{x? x - 2 &= 2 \cr x? x - 2 + 2 &= 2 + 2 \cr
           x? x - 0 &= 4\cr x? x &= 4 . \cr}$$
La solución de este último problema, $x?x=4$, es conocida, a saber, cuatro.
Esto significa que debo tener cuatro manzanas para que, comiéndome hoy dos,
tenga otras dos para mañana.

El algoritmo para resolver este problema consiste en sumar 2 a las
expresiones a la izquierda y a la derecha del símbolo `$=$'. Esto es válido
porque si sumamos la misma cantidad a dos cantidades que son iguales, las
cantidades aumentadas siguen siendo iguales. Con esto se consigue despejar
la incógnita que, en un principio, estaba acompañada de un incómodo ${}-2$.
Al despejar la incógnita, lo que queda al otro lado del símbolo `$=$' es la
solución buscada.

Se pueden idear otros algoritmos más generales. Por ejemplo, uno capaz de
resolver el problema sea cual sea la cantidad $b$ que se resta a la
incógnita para que el resultado sea otra cantidad conocida cualquiera $a$,
es como sigue:
 $$(x? x - b = a) \implies (x? x = a + b) .$$

Del carácter solo sintáctico de la incógnita derivábamos el carácter
meramente sintáctico del problema. Ahora, de éste, inferimos el carácter
también exclusivamente sintáctico del algoritmo transformador de problemas.
De modo que la resolución, como el problema, es un objeto sintáctico y no
semántico.

\goodpage

\chapter Los teoremas

Resulta que si $a=b$ entonces también es $b=a$, y viceversa. La frase
anterior es resumida por los matemáticos así: $a=b \iff b=a$, y entonces
añaden que la igualdad de los números tiene la propiedad de simetría, pero
sólo vuelven a repetir lo dicho en la primera frase. El caso es que merced a
esta propiedad de la igualdad de los números, si un problema tiene la forma
$x? x = a + b$, entonces resulta que el problema $x? a + b = x$ es
equivalente y tiene la misma solución. Este segundo problema puede ser
resuelto con la ayuda de la calculadora, pero no el primero. Luego ésta es
una transformación de expresiones útil porque permite resolver más problemas
con la calculadora.

La cuestión que quiero plantear es si estamos, o no, obligados a admitir que
$a=b \iff b=a$, o por tratarse de un simbolismo podemos establecer
cualesquiera reglas que queramos. Lo inmediato es pensar que lo único que
generaliza la propiedad anterior es que si cinco es igual al número de dedos
en mi mano, entonces el número de dedos en mi mano es igual a cinco. Esto es
una obviedad que no puede ser negada y, por consiguiente, estamos obligados
a admitir la propiedad simétrica de los números.

A pesar de su aparente sencillez, la explicación anterior da por supuesto
demasiado. Nuestros procesos de simbolización y semantización son tan
potentes y automáticos que apenas nos percatamos de que los usamos. Ahora
intentaremos otra explicación más larga, no por aburrir, sino por evitar
supuestos infundados.

La elección de símbolos es completamente arbitraria, o mejor dicho,
convencional. Nada de la palabra margarita sugiere la flor que así llamamos.
Por esta razón en otros idiomas usan otros nombres. Basta con que todos los
hablantes del idioma convengan usar el mismo nombre para referirse a las
margaritas. La ordenación sintáctica de las palabras en frases también
obedece a reglas arbitrarias. Así, el orden que consiste en comenzar la
frase por el sujeto seguido del verbo y terminarla por los complementos, que
se emplea en castellano, no es el usado en otros idiomas, como el alemán o
el japonés.

Pero una vez establecida la convención, puede resultar que hay varias
maneras de decir lo mismo. Volviendo al ejemplo, resulta que $a=b$ dice lo
mismo que $b=a$. En este caso, todos los que han aceptado la primera
convención están obligados a convenir, también, que $a=b \iff b=a$. Lo
contrario supondría que los discrepantes usan otro convenio, es decir, usan
otro idioma.

\goodpage
\chapter No sé nada

En castellano la doble negación supone una negación reforzada. La frase `no
hay ninguno bueno' significa, en castellano, que `absolutamente todos son
malos' y no que `hay alguno bueno', como ocurre en otros idiomas, por
ejemplo el inglés, en los cuales la doble negación afirma.

\chapter Las paradojas

Los simbolismos sirven para resolver problemas. Tienen dos capas: la
sintaxis con las expresiones, y la semántica con las soluciones. Algunas
expresiones tienen una correspondencia semántica directa, de manera que
sirven para expresar simbólicamente las soluciones. En otros casos la
correspondencia es indirecta, como ocurre con los nombres comunes, y en
otros no la hay, y entonces hablamos de paradojas. Una paradoja es una
expresión sintáctica correcta, que no tiene correspondencia semántica, pero
que se confunde con expresiones que sí la tienen.

Por ejemplo, la expresión `un reloj redondo' corresponde al problema
 $x? (x \hbox{ es reloj}) \land (x \hbox{ es redondo})$,
que tiene muchas soluciones y que, por lo tanto, tiene significado. Y sin
embargo, la expresión de estructura similar `un cuadrado redondo' es
paradójica porque el problema correspondiente,
 $x? (x \hbox{ es cuadrado}) \land (x \hbox{ es redondo})$,
no tiene solución, y por lo tanto no tiene significado.

Otro caso de paradoja es el de los conjuntos infinitos. Por ejemplo, el
conjunto de los números naturales, definido así:
 $${\N} = \{ x? (x = 0) \lor (\exists y \in {\N}:  x = y + 1)\} .$$
Hemos notado el problema entre llaves, `$\{$' y `$\}$', para expresar que
$\N$ es el conjunto de todas las soluciones del problema. La relación $y
\in \N$ es verdadera si $y$ es elemento del conjunto $\N$, y falsa si no lo
es. El signo $\exists$ denota el cuantificador existencial, de modo que
$\exists y \in {\N}$ se cumple si existe algún elemento que pertenezca al
conjunto $\N$. Luego la expresión que define $\N$ dice, en castellano, que
`$\N$ es el conjunto de todos los números que, o bien son el cero, o bien
son el resultado de sumar 1 a algún elemento perteneciente al conjunto
$\N$'. Pues bien, el conjunto de los números naturales, $\N$, es paradójico
porque el problema que lo define tiene infinitas soluciones, siendo
imposible completar su construcción:
 $${\N} = \{ 0, 1, 2, 3, 4, 5, 6, 7, 8, 9 , 10, 11, 12, 13, \ldots \} .$$

\goodpage
\chapter G\"odel

Un tercer tipo de paradoja ocurre cuando la secuencia de resolución no puede
terminar, es decir, no puede alcanzar la solución.

Es el caso de la paradoja `esta frase es falsa', que si fuera cierto lo que
dice, entonces sería falsa, pero entonces sería cierto lo que dice, y
entonces sería falsa, y así por siempre. Llamaremos, a este tipo de
paradoja, paradoja de \person{Epiménides}, ya que se cuenta que este
cretense dijo que todos los cretenses son unos mentirosos. La misma paradoja
fue redescubierta por
\person{Russell} en el formalismo de la teoría de conjuntos, al definir un
conjunto $R$ del siguiente modo:
 $R = \{ x? x \notin x \}$,
que es paradójico porque no se puede afirmar que
 $R \in R$ ni, por el contrario, que
 $R \notin R$.

\person{G\"odel} descubrió que pueden expresarse problemas de este tipo en el formalismo
necesario para expresar la aritmética. Reduciendo a la caricatura el teorema
de indecidibilidad de \person{G\"odel}\cite{G\"odel1931}, éste demuestra que
si un sistema inferencial es lo bastante potente como para permitir expresar
la proposición `esta proposición no es inferible', como es el caso de la
aritmética axiomatizada de los Principia Mathematica de \person{Whitehead} y
\person{Russell}, entonces tal sistema formal admite proposiciones
indecidibles.

La recursividad es la propiedad que permite que una expresión haga
referencia a sí misma. Nos permite decir, `esto es lo que yo digo'. Muchos
procedimientos de uso cotidiano lo son, por ejemplo, todos aquéllos en los
que se dice `y repita este procedimiento hasta que \dots'. Son recursivos
todos los libros que dicen `este libro', a no ser que `este libro' esté
siempre escrito entre comillas. La recursividad ha sido investigada por
\person{Hofstadter}\cite{Hofstadter1979} en disciplinas tan diversas como
la música de \person{Bach} y el dibujo de \person{Escher}.

Todos los simbolismos con poder expresivo no limitado son recursivos, es
decir, admiten expresiones recursivas. En todos los simbolismos que permiten
la recursividad se pueden expresar paradojas de \person{Epiménides}.

\vfil
\centerline{\it Esta frase es falsa}
\vfil
\break

\chapter El problema de la parada

Un algoritmo es un procedimiento simbólico completamente definido, es decir,
definido sin ambigüedad, de tal manera que puede emplearse para instruir a
una computadora.

Uno de los más famosos descubrimientos de la ciencia de la computación
establece que no puede construirse un algoritmo que finalice y que diga, de
cualquier algoritmo, si finalizará o si no tendrá fin\cite{Arbib1987}. Si
tal algoritmo pudiera construirse, entonces las paradojas apenas serían
paradojas.

Es posible darse cuenta de que la frase `esta frase es falsa' es paradójica.
Basta mostrar que las dos únicas posibilidades existentes, que sea verdadera
y que sea falsa, llevan la una a la otra, de modo que forman un círculo
vicioso. Si esto fuera todo, repito, las paradojas apenas serían paradojas.

Pero qué pasa con la paradoja de la tarjeta de
\person{Jourdain}\cite{Smullyan1984}, que en una de sus caras muestra la
frase `la frase del otro lado de esta tarjeta es falsa' y en la otra cara
`la frase del otro lado de esta tarjeta es verdadera'. Pues ocurre que
también puede construirse un algoritmo que detecte esta paradoja.

En realidad, dada una paradoja concreta, siempre es posible construir un
algoritmo finito que la detecte. Un algoritmo finito puede detectar,
incluso, un conjunto infinito de paradojas. Lo que hace que las paradojas
sigan siendo paradojas, es que no puede construirse un algoritmo finito que
sea capaz de detectar todas las paradojas posibles.

\vfil
\MTbeginfigure(80,80);
 \MT: pickup med_pen;
 \MT: save u, v, alpha; u = w/2; v = 12; alpha = -6;
 \MT: z1 = (w/2,h/2) + u*(right rotated alpha);
 \MT: z2 = (w/2,h/2) + u*(right rotated (alpha+120));
 \MT: z3 = (w/2,h/2) + u*(right rotated (alpha-120));
 \MT: z1r = (w/2,h/2) + u*(right rotated (alpha+v));
 \MT: z2r = (w/2,h/2) + u*(right rotated (alpha+120+v));
 \MT: z3r = (w/2,h/2) + u*(right rotated (alpha-120+v));
 \MT: draw z1 .. z1r; draw z2 .. z2r; draw z3 .. z3r;
 \MT: draw z1 .. z3r; draw z2 .. z1r; draw z3 .. z2r;
 \MT: z1m = whatever[z1,z2r]; z1m = whatever[z1r,z3];
 \MT: z2m = whatever[z2,z3r]; z2m = whatever[z2r,z1];
 \MT: z3m = whatever[z3,z1r]; z3m = whatever[z3r,z2];
 \MT: draw z1 .. z2m; draw z2 .. z3m; draw z3 .. z1m;
 \MT: z1ra = z1 reflectedabout (z1r,z1m);
 \MT: z2ra = z2 reflectedabout (z2r,z2m);
 \MT: z3ra = z3 reflectedabout (z3r,z3m);
 \MT: draw z1m .. z1ra; draw z2m .. z2ra; draw z3m .. z3ra;
 \MT: z3x - z1x = whatever*(z3-z1m);
 \MT: z2x - z3x = whatever*(z2-z3m);
 \MT: z1x - z2x = whatever*(z1-z2m);
 \MT: z1ra - z1x = whatever*(z3-z1m);
 \MT: z3ra - z3x = whatever*(z2-z3m);
 \MT: z2ra - z2x = whatever*(z1-z2m);
 \MT: z1y = whatever[z1m,z3]; z1y = whatever[z1x,z2x];
 \MT: z2y = whatever[z2m,z1]; z2y = whatever[z2x,z3x];
 \MT: z3y = whatever[z3m,z2]; z3y = whatever[z3x,z1x];
 \MT: draw z1x .. z3y;
 \MT: draw z2x .. z1y;
 \MT: draw z3x .. z2y;
\MTendfigure"Tribar de Penrose"Figura adaptada de \person{Ernst}
             y \person{Resnikoff}.
             ``La cascada'' de \person{Escher} usa tres tribares;
\vfil


\chapter Platón

El idealismo postula la existencia de las ideas, e incluso les asigna un
mayor carácter de verdad que a las cosas. Así, la idea aritmética $1+1=2$
tiene para los idealistas una existencia plena, eterna, inmutable y
absoluta, o sea, universal. Esto está en abierta oposición a nuestra teoría,
por lo que tenemos que refutar que $1+1=2$ es una verdad universal.

Nuestro primer ataque va contra la notación. Es evidente que el signo 1
podría significar tres tanto como significa uno. En este caso, y manteniendo
el significado de los demás signos, $1+1=2$ no sería verdad. Por lo tanto,
depende de la interpretación de los signos y no es una verdad universal.

El idealista puede, entonces, retroceder un paso y afirmar que la verdad
universal es que `uno más uno son dos'. Si añadimos una moneda a la moneda
que ya teníamos, nos quedamos con dos monedas, lo que quiere decir que este
tipo de problemas se resuelven con la ayuda del `uno más uno son dos'. Pero,
¿qué pasa si añadimos una gota de agua a la gota de agua que ya teníamos?
Que nos quedamos con una gota de agua, eso sí, mayor que aquélla que
teníamos.

El idealista nos objetará que este caso no refuta la suma de uno más uno,
sino que, simplemente, en el caso de las gotas de agua, no se puede sumar.
Luego, en definitiva, no siempre puede aplicarse la suma, por lo que `uno
más uno son dos' es una abreviatura de `un objeto sumable más un objeto
sumable son dos objetos sumables', que ya no tiene excepciones, aunque
plantea el problema de determinar qué es un objeto sumable.

Lo que ocurre es que para que el simbolismo nos sirva, siempre que
expresemos sintácticamente un problema del modo $x? 1 + 1 = x$, debemos
obtener dos como solución. Pero, para expresar un problema como $x? 1 + 1
= x$, antes hemos de haber concluido que los objetos son sumables, que han
sido añadidos, y que por lo tanto es pertinente hacer la suma para hallar la
solución.

Aquellos lectores idealistas y con vocación matemática pueden probar su fe
contra \person{Lakatos}\cite{Lakatos1976}.

La mayoría de los diccionarios definen dos como uno y uno.

\vfil
 \centerline{{\bf dos.} Uno y uno.}
\vfil
\break

\def\opera#1{\setbox0=\hbox{\seveni#1}%
 \setbox2=\hbox{$\mathhexbox20D$}%
 \dimen0=\wd0 \advance\dimen0 by \wd2 \divide\dimen0 by 2
 \mathbin{\box2\kern-\dimen0\vcenter{\copy0}\kern\dimen0\kern-\wd0}}

\chapter Otras operaciones

Además la sintaxis es extensible, de modo que uno es libre de definir
operaciones, y por ejemplo los matemáticos han definido, además de la suma,
muchas otras, como por ejemplo: $1 - 1 = 0$ y $1 \times 1 = 1$. Incluso es
posible definir una operación, notada $\opera 9$, tal que, siendo $a$ y $b$
los dos operandos, resulte que, $a \opera{9} b = a + b + 9$, así que, por
ejemplo, $1 \opera{\hbox{\sevenrm{9}}} 1 = 11$.

Así como la suma puede parecer una verdad universal, la suma con nueve
desplazamientos, $\opera{\hbox{\sevenrm{9}}}$, parece un invento que tiene
la única intención de confundir, porque, además, puede ser reducida a sumas.
Pero, piénsese, por ejemplo, en la multiplicación. La multiplicación también
puede ser reducida a sumas, y sin embargo se considera a la multiplicación
una de las cuatro operaciones aritméticas básicas. Las dos únicas
diferencias entre la multiplicación y la suma con nueve desplazamientos son:
\point que la multiplicación es más útil,
es decir, aparece con más frecuencia en problemas, y
\point que la reducción de una multiplicación a sumas es más
complicada, es decir, resulta en general en muchas más cantidades a sumar.
\par
\noindent La multiplicación se presenta en problemas con tanta frecuencia
que compensa idear atajos que eviten hacer las sumas y, por esto, los niños
en las escuelas aprenden el algoritmo de la multiplicación y las
calculadoras la incluyen en su repertorio de operaciones.

\def\sucesor{\mathop{\cal S}\nolimits}

Pero es que la propia suma puede reducirse a una operación más simple. Como
sabemos también desde niños, sumar puede reducirse a contar. Saber contar
supone conocer cual es el número siguiente a cualquier número dado. Así
sabemos que el número siguiente al 3 es el 4, lo que anotaremos $\sucesor(3)
= 4$. Sumar 1 a cualquier número $n$ es hallar el número siguiente a $n$.
Expresado matemáticamente $1 + n = \sucesor(n)$. Y sumar $m$ a $n$ supone
aplicar $m$ veces $\sucesor$ a partir de $n$:
 $$m+n= \underbrace{\sucesor(\sucesor(\ldots \sucesor}_{m\rm\;veces}
  (n) \cdots)) .$$

Una de las virtudes de los simbolismos es su completa flexibilidad, de modo
que se pueden definir sintaxis del modo que más convenga en cada
circunstancia. Pero esto no significa que todas las sintaxis posibles sean
igualmente útiles. La suma es, posiblemente, la operación más útil, siendo
además la base sobre la que se han definido otras, como la multiplicación,
que es una suma reiterada.

Que la suma sea extraordinariamente útil a las personas no significa que sea
una verdad universal. Hacer lo útil al hombre igual a universal, es un error
de perspectiva denominado homocentrismo.


\chapter No todos los simbolismos son igualmente útiles

Los simbolismos son extensibles, es decir, se pueden extender de modo que
permitan la resolución más fácil de un mayor número de problemas. Esto
supone que unas de sus partes son más fundamentales que las otras.

Por ejemplo, la aritmética de los números naturales es fundamental porque
una gran parte de las matemáticas resulta de su extensión. Esto hace que los
conceptos de la aritmética de los números naturales, como el $1+1=2$, sean
tan útiles que es imposible alterarlos sin que muchos otros conceptos tengan
que ser modificados. Por esta razón es mejor no alterarlos.

Algunas de las normas de los simbolismos son convencionales y otras quedan
determinadas por las anteriores. Cualquier convención es posible, pero no
son todas ellas igualmente útiles, y por lo tanto, si imponemos la condición
de utilidad, algunas se impondrán y otras serán insostenibles. Lo que en
último término importa es que el simbolismo sea útil, que permita solucionar
problemas.

Como resulta imposible que cada hablante de un idioma explicite todas las
convenciones que usa, la situación es, en la práctica, algo confusa (y, a
veces, divertida). Las matemáticas intentan evitar esta dificultad (y, tal
vez, la diversión), por lo que exigen que se haga explícita cada una de las
convenciones usadas. A continuación veremos algunos ejemplos de extensión de
la aritmética, que son algo técnicos aunque se ha procurado que las
matemáticas utilizadas sean simples.



\chapter Sobre el cero

En el caso del infinito, se presentan varias maneras de extender el
simbolismo matemático. Antes de investigar sobre el infinito, vamos a
estudiar el cero. Cero e infinito están relacionados, y el cero es más
manejable que el infinito, de modo que estudiando el cero podremos preparar
el entendimiento del infinito.

El cero, escrito `0', designa la nada. Luego el cero no se refiere a objeto
alguno, o dicho de otro modo, el cero es un símbolo. Como tal admite
diferentes interpretaciones. La primitiva se deriva de la interpretación de
los números naturales como las representaciones de las cantidades. Si un
número representa una cantidad de objetos de cierto tipo, el cero se usa
cuando no hay objetos de ese tipo. Si un problema que usa esta
interpretación de los números tiene como solución cero (0), significa que no
hay ninguno. En otros casos el cero (0) es el valor de referencia, el
origen. En este caso los números representan desviaciones, y una solución
cero (0) significa que no hay desviación.

Desde el punto de vista de la resolución de problemas aritméticos, el 0 se
hace necesario en cuanto se admite la resta como operación, ya que si no el
problema $x? x = 1 - 1$ se quedaría sin solución. Pero una vez admitido el 0
como número, éste también debe poder aparecer en las sumas. Esto hace
necesario definir el resultado de operaciones como $1+0$.

\chapter Sobre el infinito

Si el cero (0) aparecía con la resta, el infinito ($\infty$) aparece con la
división. Precisamente cuando el divisor es 0, $x? x = 1 / 0$.

El infinito ($\infty$), como el cero (0), es un símbolo que admite tantas
interpretaciones como los números. Así, si la solución de un problema es que
se precisa una cantidad infinita de algo, entonces es que la cantidad
necesaria es mayor que cualquier cantidad que se tenga. Si, por otro lado,
los números representan desviaciones, entonces una solución $\infty$
significa que la desviación está siempre más allá de cualquier límite que
impongamos.

Resulta que una solución 0 a un problema significa que, bien mirado, el
problema no es tal, ya que o bien no se necesita ninguna cantidad adicional
o bien la desviación respecto al objetivo ya es nula. En el caso de una
solución $\infty$ la situación es la inversa, es decir, el problema no tiene
solución, ya que es imposible conseguir la cantidad necesaria o la
desviación está fuera de cualquier alcance.


\chapter Cantor

Queda por definir el resultado de las operaciones en las que aparece
$\infty$. Sucede que $\forall a,\; a+1 > a$, de manera que, aplicado a
$\infty$, queda: $$\infty+1 > \infty .$$

\person{Cantor}\cite{Cantor1897}, en vez de extender al infinito esta
verdad finita que establece que el número siguiente es siempre mayor que el
anterior, extendió otra verdad finita, a saber, que si dos conjuntos finitos
tienen el mismo número de elementos, entonces se pueden emparejar los
elementos de uno y otro de tal suerte que ningún elemento se quede sin
pareja.

\goodpage

Llamando $\N^1$ al conjunto de los números naturales sin el cero, es decir,
 $$\N^1 = \{1, 2, 3, \ldots\},$$
y $\N^0$ al conjunto de los números naturales con el cero,
 $$\N^0 = \{0, 1, 2, 3, \ldots\},$$
\person{Cantor} razonó que podía hacerse el emparejamiento completo de los
elementos de ambos conjuntos, del siguiente modo. A cada número de $\N^0$ le
corresponde en $\N^1$ el número siguiente, y viceversa, es decir, al 0 de
$\N^0$ le corresponde el 1 de $\N^1$, al 1 el 2, al 2 el 3, y así
sucesivamente.
$$\vbox{\halign{\hfil#\hfil&&\quad\hfil#\hfil\cr
 $\;\N^0$&$\;\N^1$\cr
 \noalign{\vskip2pt\hrule\vskip3pt}
 0&1\cr 1&2\cr 2&3\cr 3&4\cr
 $\vdots$&$\vdots$\cr$n$&$n+1$\cr$\vdots$&$\vdots$\cr}}$$
Al no quedar sin pareja ningún número de $\N^0$, y tampoco ningún número de
$\N^1$, \person{Cantor} concluyó que ambos conjuntos tienen el mismo número
de elementos. Como quiera que $\N^1$ tiene $\infty$ elementos, y $\N^0$
tiene un elemento más, resulta que: $$\infty + 1 = \infty .$$

Tenemos, al menos, dos posibles maneras de extender al infinito la
aritmética finita; o bien $\infty+1>\infty$, o bien $\infty+1=\infty$.
Interesará extender el simbolismo de aquel modo que permita solucionar los
problemas más fácilmente y en mayor cantidad. Y si ambas fueran útiles,
simplemente tendríamos que desdoblar el concepto de infinito y aplicar en
cada caso el adecuado.

\vfil
$$\aleph_0 + 1 = \aleph_0$$
\vfil\break


\chapter El cambio

La primera contienda filosófica, sostenida por los filósofos griegos más
antiguos, se produjo por la aparente contradicción entre el cambio y la
permanencia. Aquello que captan nuestros sentidos es mudable y, sin embargo,
una vez que pensamos en ello se nos presenta inmutable. Así, mientras
\person{Heráclito} propugnaba que todo cambia, \person{Parménides}
mantenía lo contrario. ¿Quién merece más crédito?

Las simbolizaciones y los símbolos son permanentes, inmutables. Si no
fijamos que $1+1=2$, entonces el simbolismo pierde su utilidad. Gracias a
que $1+1=2$ es una verdad inmutable, podemos resolver los problemas que
pueden expresarse como $x? 1+1=x$.

La frase `una margarita crece en tu jardín' describe un cambio pero puede
ser escrita en piedra. El cambio principal es el captado por el verbo
`crecer', pero también la margarita y el jardín cambian. La simbolización
consiste en la fijación de lo cambiante. Sobre el verbo recae la mayor carga
de este proceso de fijación, pero no toda.

Una frase completa, con sujeto, verbo y complementos, cumple el requisito
periodístico de informar en cada noticia sobre quién (sujeto), qué (verbo),
cómo (adverbio), cuándo (tiempo verbal y complemento temporal) y por qué
(complemento causal) ha sucedido. Ignoro si la gramática universal de
\person{Chomsky}\cite{Chomsky1989} prescribe que todos los idiomas
deben acomodar estos elementos de un modo u otro, pero no me extrañaría dada
la utilidad que tiene que el idioma fuerce, o sugiera, qué elementos de
información conviene reseñar.

El verbo es el nucleo de la frase, la parte que no puede faltar. No puede
faltar porque el verbo es la parte de la frase que explica, precisamente,
qué ha sucedido, qué ha cambiado. Pero debemos ir más lejos. Cada frase
describe un suceso, un fenómeno, un cambio que es especificado por el verbo.
Que cada frase describa un cambio implica que aquello que no cambia no puede
ser dicho. No necesita explicarse aquello que no cambia, de modo que no hay
necesidad de decirlo.



\chapter La permanencia

Sin embargo hay verbos que no describen un cambio, sino una permanencia o
persistencia. En castellano se trata, principalmente, del verbo {\em ser}.
La frase `la margarita {\em es} una flor de centro amarillo rodeado de
pétalos blancos' no describe un cambio, sino una permanencia. En realidad,
la frase anterior sirve para identificar las margaritas, es decir, explica
que el nombre margarita se emplea para referirse a las soluciones de cierto
problema. En otras ocasiones, como cuando se dice `mi lápiz es rojo', sirve
para definir mi lápiz añadiendo una condición a las condiciones que definen
a los lápices en general.

Otros verbos, como estar o saber, describen estados. Los estados son
resúmenes de sucesos. Por ejemplo, para estar aquí he tenido que venir hasta
aquí. Luego los verbos de estado describen indirectamente sucesos.

Todas aquellas frases que sí se refieren a construcciones simbólicas, no
describen cambios. Caben aquí las paradojas, como `esta frase es falsa',
pero también otras frases que sirven para extender el aparato simbólico de
resolución, como las definiciones.

Todas aquellas frases que no se refieren a construcciones simbólicas, sí
describen cambios. Aquí encontramos todas aquellas frases que describen el
exterior, los llamados fenómenos.

\chapter El símbolo es el fenómeno fijado

Luego los conceptos de cambio y de permanencia nos permiten distinguir el
mundo simbólico interior, que es inmutable, del mundo fenoménico exterior,
que es cambiante. Por esto la frase de \person{Heráclito} el oscuro que
tanto me gusta repetir, ``nada permanece excepto el cambio'', es enormemente
profunda y no es paradójica ni oscura, sino que se refiere, en la misma
frase, tanto a lo interior como a lo exterior y explica, concisamente, que
el símbolo es el fenómeno fijado.

Se trata de una traducción libre del fragmento 50 de
 \person{Brun}\cite{Brun1965}
 (que se corresponde con el 84a de \person{Diels} y \person{Kranz}),
 ``$M\varepsilon\tau\alpha\beta\acute\alpha\lambda\lambda o\nu$
 $\mathaccent"7027 \alpha
 \nu\alpha\pi\alpha\acute\upsilon\varepsilon\tau\alpha\iota$''.
Más literalmente dice que ``el fuego permanece en el cambio'', y ya que para
\person{Heráclito} el fuego es el elemento primordial, ``todas las cosas se
cambian por el fuego y el fuego por todas las cosas, al igual que las
mercancías se cambian por oro y el oro por las mercancías'' (fragmento 49
(90 de \person{Diels})), también vale la traducción ``todo permanece en el
cambio''.

\vfil
$$M\varepsilon\tau\alpha\beta\acute\alpha\lambda\lambda o\nu
 \hbox{ }
 \mathaccent"7027 \alpha
 \nu\alpha\pi\alpha\acute\upsilon\varepsilon\tau\alpha\iota$$
\vfil\break

\chapter Zenón de Elea

Históricamente, la polémica sobre el cambio y la permanencia fue avivada por
\person{Zenón} de Elea con sus aporías sobre el movimiento, la más famosa
de las cuales puede ser la de Aquiles y la tortuga.

\person{Zenón}\cite{Northrop1944} demostró que el veloz Aquiles es incapaz de alcanzar a la
lenta tortuga, siempre que la tortuga tenga alguna ventaja inicial. Porque
cada vez que Aquiles llega al lugar en donde estaba la tortuga al principio
del movimiento, ésta ya ha tenido tiempo para desplazarse algo más lejos,
por lo que la tortuga se mantiene siempre por delante de Aquiles.

\goodpage

La aporía de Aquiles y la tortuga se supera admitiendo que una suma de
infinitos términos puede tener un resultado finito. Las matemáticas, para
quien tenga la disposición de seguirlas, son, en el caso de que la velocidad
de Aquiles doble a la velocidad de la tortuga y donde la suma representa el
tiempo que tarda Aquiles en alcanzar a la tortuga, como sigue.

Para cualquier número finito $n$ se cumple que:
$$\sum_{i=1}^n {1 \over 2^i} = 1 - {1 \over 2^n} .$$
Por ejemplo, si se suma $1/2 + 1/4 + 1/8$ se obtiene $7/8 = 1 - 1/8$. Esto
significa que si tomamos valores muy grandes de $n$, entonces la suma será
casi 1. De aquí que se diga que:
$$\sum_{i=1}^{\infty} {1 \over 2^i} = 1 .$$

El paso del caso finito al infinito no es una inferencia, sino una extensión
del simbolismo que, de este modo, vence la aporía planteada por
\person{Zenón}. Una vez que se demuestra la utilidad de una extensión, ésta
se adopta y se toma por verdadera.

\goodpage

Esta verdad se inventó, que no descubrió, unos dos mil años después de que
\person{Zenón} planteara sus aporías, de modo que las aporías estaban
vigentes y eran recientes en los tiempos de \person{Sócrates}. Pues bien, a
pesar del hecho experimental de que Aquiles adelanta efectivamente a la
tortuga, con lo que \person{Parménides} queda refutado, \person{Sócrates} y
sus discípulos \person{Platón} y \person{Aristóteles} desdeñaron por una
razón ética a \person{Heráclito}, a los físicos jónicos y a los hábiles
sofistas. Y aunque hoy nos pueda resultar inverosímil, negaron la evidencia
e interpretaron las aporías de \person{Zenón} como prueba de que el
movimiento no podía existir, es decir, como prueba de la validez de los
postulados inmovilistas de \person{Parménides}.

Las consecuencias de esta postura ética han sido enormes. Al menospreciarse
el fenómeno y el dato empírico, se frenó la ciencia entonces incipiente. Al
ensalzarse la razón pura se impulsó la especulación escolástica, que era
estéril por ser el reflejo de sí misma.

Es, por ello, un enorme contrasentido que \person{Sócrates}, una de las
personas que más han influido en la historia, y que desterró el subjetivismo
sofista durante dos mil años, fuera ajusticiado por sofista\cite{Tovar1984}.


\endinput

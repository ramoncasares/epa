% EPAE.TEX (RMCG19950910)

\labeled El álgebra automática\title1 El ^^e1lgebra autom^^e1tica

\title2 General

Se presenta un álgebra de autómatas.  Esto es, se formaliza el
autómata finito, binario y probabilístico y sobre él se definen tres
operaciones.  Se describe tanto la estática como la dinámica de los
autómatas y de las operaciones.  Se demuestra constructivamente que
cualquier autómata puede ser escrito como una expresión que sólo
utiliza siete palabras diferentes.  También se construye un autómata
universal.  Se definen las siguientes relaciones entre autómatas:
igualdad, indistinguibilidad, equivalencia y ampliabilidad. Varios campos
de las matemáticas resultan ser subálgebras del álgebra automática.


\title2 El vector

\title3 La variable

Una \definition{variable} no cambia sino de \definition{valor}.  Una
variable puede tomar valores distintos en momentos distintos.  Si los
posibles valores son dos, tenemos la \definition{variable binaria}.  De
aquí en adelante supondremos que las variables son binarias, y en
concreto denominaremos $1$ a uno de los dos valores que pueden tomar y
denominaremos $0$ al otro de los valores, $\B = \{ 1, 0 \}$.  Otros tipos
más complejos de variables pueden convertirse en una serie de variables
binarias utilizando la codificación adecuada.

Un conjunto ordenado de $\no N$ variables es un \definition{vector} de
$\no N$ componentes.  Denominaremos \definition{string} al valor de un
vector en un instante.  A la secuencia de strings, según se desarrolla
en el tiempo, la denominaremos \definition{stream}.

A una disposición de números ordenada por índices la denominamos
\definition{array}, y decimos que tiene tantas dimensiones como índices
utiliza.  Así, un {\em string} es un array unidimensional, una
\definition{matriz} es un array bidimensional.

Un \definition{índice} que haya de tomar $n$ valores recorrerá, de
mayor a menor, los $n$ menores números naturales.  Como los
\definition{números naturales}, $\N$, son el cero y los enteros
positivos, es decir, $\N = \{ 0, 1, 2, 3, \ldots \}$, el índice con $n$
posibilidades tomará los valores desde el $n-1$ hasta el $0$, y en ese
orden.

Si denominamos \definition{rango} de un índice al número de valores que
puede tomar, entonces el tamaño de un array se obtiene multiplicando los
rangos de todos sus índices.

\title3 La formalización

Sea un vector, que llamaremos |V, con $\no N$ variables.  Como $\no N$
variables binarias ordenadas son capaces de $2^{\no N}$ combinaciones,
asignamos a cada una de las distintas combinaciones un número diferente,
o índice, comprendido entre el $2^{\no N} - 1$ y el $0$, ambos incluidos.

Escribimos $j \through \no N$ para explicar que el índice $j$
\definition{recorre la potencia} del número $\no N \in \N$, es decir,
para expresar que $j$ toma valores desde $2^{\no N} - 1$ hasta $0$, ambos
incluidos y en orden decreciente.

Nótese que estamos interpretando vectores binarios, de $\no N$
componentes, como variables no binarias, con $2^{\no N}$ posibles valores.
Si la escritura de los números es binaria, entonces la diferencia es
mínima.  Al valor del vector, o string, binario de longitud
 $\ell = \no N$,
 $$S = [b_{\ell-1}\,b_{\ell-2}\, \ldots\, b_2\,b_1\,b_0],$$
con $b_i \in \B = \{1,0\}$ y, por tanto, $S \in \B^\ell$, notando $\B^\ell$
el producto cartesiano de $\B$ por él mismo $\ell$ veces, le corresponde
como índice el número
 $$\overline S = b_{\ell-1}.2^{\ell-1} + b_{\ell-2}.2^{\ell-2}
   + \cdots + b_2.2^2 + b_1.2 + b_0 = \sum_{i=\ell-1}^0 b_i.2^i.$$
Por ejemplo, al string $[1\,0\,1]$ le corresponde el índice binario $101$,
que en decimal se escribe $5$:
 $$\overline{[1\,0\,1]} = 101_2 = 5_{10} .$$
De este modo se justifica que nos refiramos indistintamente al índice o
al correspondiente string de un vector.

Notamos $|V_r \!/ 2^{\no P}$ la probabilidad de que el vector |V tome el
valor correspondiente al índice $r$.  Sólo consideraremos valores naturales
de $|V_r$ y $\no P$.  Con esta notación tenemos que
 $r \through {\no N}$,
con lo que resulta que $(|V_r)$ es una forma ordenada (de mayor a
menor índice) de hacer referencia a $2^{\no N}$ números.  Así que
$(|V_r)$ es un string de tamaño $2^{\no N}$.

Si |V es un vector, entonces podemos establecer la siguiente convención
para describirlo probabilísticamente.  Escribimos, en este orden:  su
número de variables, $\no N$; $\no P$, siendo $2^{\no P}$ el divisor
común de todas las probabilidades; y los $2^{\no N}$ valores que toma
$|V_r$ ordenados de mayor índice ($r = 2^{\no N} - 1$) a menor índice
($r = 0$).
 $$\eqalign{ |V
  = &{} \left< {\no N}, {\no P}; (|V_r) \right> \cr
    & {\no N}, {\no P}, |V_r \in \N \cr
    & r \through {\no N} \cr
    & \textstyle{\sum_r} |V_r = 2^{\no P} .\cr}$$

\title4 Ejemplo

Si $|V = \left< 3, 1; (0, 0, 1, 0, 0, 0, 0, 1) \right>$, entonces es tan
probable que |V tome el valor $[1\,0\,1]$ como el valor $[0\,0\,0]$.

\title3 La igualdad de vectores

La \definition{igualdad de vectores}:  decimos que dos vectores descritos
probabilísticamente, |V y |W, son iguales, y lo notaremos $|V = |W$, si y
sólo si, por definición, ${\no N}_{|V} = {\no N}_{|W}$ y:
 $$\forall r \through {\no N}_{|V} = {\no N}_{|W} : \;
   { |V_r \over 2^{{\no P}_{|V}} } = { |W_r \over 2^{{\no P}_{|W}} }.$$

\labeled\title3 Notación vectorial de strings

Si un vector es determinístico, $\no P = 0$, entonces sólo toma un
valor, aquél cuyo índice $j$ es tal que $|V_j = 1$.  Confundiendo el
vector determinístico con el único valor que toma, podemos abusar de la
descripción de vectores para anotar strings del siguiente modo:
 $S = \left< \ell, 0; (S_r) \right>$,
siendo $S_{\overline S} = 1$ y $S_{r \not= \overline S} = 0$.

\title4 Ejemplo

El string $[1\,0\,1]$ se escribe, a la manera de los vectores, así:
 $\left< 3, 0; (0, 0, 1, 0, 0, 0, 0, 0) \right>$.


\title2 El autómata

Un \definition{autómata finito} se compone de tres vectores y una ley de
dependencia.  Los tres vectores son:  el vector de entrada con las
variables de entrada, el vector de salida con las variables de salida y el
vector de estado con las variables de estado.

La determinación de los valores de las variables de entrada queda fuera
de la definición del autómata.  Se dice que las variables de entrada
son independientes.

Los valores que toman en este instante las variables de salida dependen de
los valores que toman en este instante las variables de estado y las
variables de entrada.

Los valores que tomarán en el instante siguiente las variables de estado
dependen de los valores que toman en este instante las variables de estado
y las variables de entrada.  De otra manera, el string de estado siguiente
depende del string de estado actual y del string de entrada actual.  El
string de estado siguiente y el string de estado actual son dos valores
consecutivos del mismo vector.

\vfil

$$\hbox{\rm Autómata}\llave{
 Estática\llave{
  Notación $|A = \left< \no I,\no O,\no S,\no P; (\ALFA) \right>$\cr
  Igualdad $=$\cr
  Operaciones\llave{
   Composición en serie $\srl$\cr
   Composición en paralelo $\prl$\cr
   Composición de realimentación $\fdb$\cr}\cr
  Función característica $\FA: |A = \fdb^{\no S} \FA$\cr
  Formas canónicas\llave{
   Descodificada $|A = \hbox{Canon}_{|A}$\cr
   Universal $|A = \hbox{Kanon}_{|A}$\cr}\cr}\cr
 Dinámica\llave{
  Ecuación dinámica $|Q = \srl |P \FA = \srl \prl |S |J \FA$\cr
  Ecuación temporal $|N(t) = |S(t+1)$\cr
  Función de transferencia $Z^{|A}_{S_0}$\cr
  Comportamiento\llave{
   Indistinguibilidad $\nodis$\cr
   Equivalencia $\equiv$\cr}\cr
  Dinámica compuesta $|A \nodis |A' \implies |F(|A) \nodis |F(|A')$\cr
 }\cr}$$

\break

\title2 La estática

\title3 La formalización

Sea |A un autómata, con $\no I$ variables de entrada, $\no O$ variables
de salida y $\no S$ variables de estado; todas las variables son binarias.
Decimos, por tanto, que el conjunto de entradas a |A es el conjunto de
strings de longitud $\no I$, $\B^{\no I}$, que el conjunto de salidas de |A
es el conjunto de strings de longitud $\no O$, $\B^{\no O}$, y que el
conjunto de estados de |A es el conjunto de strings de longitud $\no S$,
$\B^{\no S}$.

Escribimos $\ALFA \!/ 2^{\no P}$ para indicar la probabilidad de que estando
|A en el estado $s$ y recibiendo la entrada $\imath$, pase al estado $n$ y
produzca la salida $o$.  Sólo consideraremos valores naturales de $\ALFA$ y
de $\no P$.  De modo que $(\ALFA)$ es un array de cuatro dimensiones y de
tamaño
 $2^{\no S} \times 2^{\no I} \times 2^{\no S} \times 2^{\no O}$.

La definición del autómata |A queda:
 $$\eqalign{
 |A ={} &\left< \no I, \no O, \no S, \no P; (\ALFA) \right>\cr
  & \no I, \no O, \no S, \no P, \ALFA \in \N \cr
  & s \through \no S, \imath \through \no I, n \through \no S,
    o \through \no O \cr
  & \forall s, \imath: \; \textstyle{\sum_n} \textstyle{\sum_o}
    \ALFA = 2^{\no P} .\cr}$$

\label{forma A}
Haciendo
 $p = s.2^{\no I} + \imath$, $q = n.2^{\no O} + o$,
podemos reducir el número de dimensiones de $(\ALFA)$ a dos, $(|A_{pq})$,
y haciendo
 $r = s.2^{\no{I+S+O}} + \imath.2^{\no{S+O}} + n.2^{\no O} + o$,
incluso a una, $(|A_r)$.  Pero debe observarse que
 $p \through (\no S + \no I)$, que $q \through (\no S + \no O)$ y que
 $r \through ({\no{S+I+S+O}})$,
de modo que el tamaño del array de cuatro dimensiones
 $(\ALFA)$, $2^{\no S} \times 2^{\no I} \times 2^{\no S} \times 2^{\no O}$,
es el mismo que el tamaño de la matriz $({|A}_{pq})$,
 $2^{\no  S+\no I} \times 2^{\no S+\no O}$,
y el mismo que el tamaño del string
 $({|A}_r)$, $2^{\no S+\no I+\no S+\no O}$.

\title4 Ejemplos

\label{channel}
Sea
 $$|H = \bigl< 1, 1, 1, 1; \pmatrix{
   2&0&0&0\cr 1&0&0&1\cr \underline{1}&0&0&1\cr 0&0&0&2\cr} \bigr>$$
la forma matricial $(|H_{pq})$ de cierto autómata |H.  El $1$ que ocupa
la séptima posición de la matriz (llevando la cuenta creciendo desde
cero pero comenzando desde el final, o sea, el $1$ subrayado), corresponde
a $|H_{01,11}$ (decimal $|H_{1,3}$), que también es $|H_{0111}$ (decimal
$|H_7$) cuando se expresa como string, y también $|H_{0,1,1,1}$ con sus
cuatro dimensiones escritas de manera explícita.  Por lo tanto el $1$
subrayado explica que si el estado actual del autómata |H es $0$ y la
entrada es $1$, entonces la probabilidad de que el próximo estado sea $1$
y la salida (actual) sea $1$, es $1/2$.

Hagamos ahora el ejercicio inverso, es decir, representemos en esta
notación un autómata conocido.  Empezaremos por uno sencillo, la puerta
$\op{AND}$.  Para ello lo primero es determinar el número de variables,
en este caso dos de entrada y una de salida, $\inp{\op{AND}} = 2$ y
$\out{\op{AND}} = 1$.  Es determinístico, $\pr{\op{AND}} = 0$, y no tiene
variables de estado, o sea, sólo tiene un estado, $\st{\op{AND}} = 0$.
Si optamos por la representación bidimensional, que es la más sencilla,
entonces sabemos que tenemos que rellenar una matriz de cuatro filas por
dos columnas, $2^{0+2} \times 2^{0+1}$.  O sea, hemos llegado hasta:
 $$\op{AND} = \bigl< 2, 1, 0, 0; \bordermatrix{ & 1 & 0 \cr 11 &&\cr 10
   &&\cr 01 &&\cr 00 &&\cr} \bigr>.$$
Los números que rodean a la matriz aún vacía sirven de ayuda para
guiar la siguiente fase.  Sabemos que si las dos variables de entrada valen
$1$, entonces la variable de salida toma el valor $1$, de modo que donde
cruza la fila $11$ y la columna $1$ escribimos un $1$ porque su
probabilidad es $1$, y donde se encuentra la entrada $11$ con la salida $0$
escribimos un $0$ porque su probabilidad es $0$. Igualmente hemos de
considerar las otras posibilidades del vector de entrada.  La solución
final es:
 $$\op{AND} =  \bigl< 2, 1, 0, 0;
 \pmatrix{ 1 & 0 \cr 0 & 1 \cr 0 & 1 \cr 0 & 1 \cr} \bigr>,$$
cuya forma unidimensional es:
 $\op{AND} = \bigl< 2, 1, 0, 0; (1, 0, 0, 1, 0, 1, 0, 1) \bigr>$.

El siguiente nivel de complejidad aparece cuando se consideran autómatas
con variables de estado, o sea, con más de un estado.  Pero es una
complejidad menor, ya que basta percatarse de que las variables de estado
se anteponen tanto a las variables de entrada como a las de salida.  Veamos
como ejemplo el autómata $\op{DEL}$ que representa el retardo elemental,
o sea, que si se presenta a su entrada un stream unitario, a la salida
aparece el mismo stream, pero un instante más tarde.  En este caso la
plantilla a rellenar es:
 $$\op{DEL} = \bigl< 1, 1, 1, 0; \bordermatrix{ & 1.1 & 1.0 & 0.1 & 0.0 \cr
   1.1 &&\cr 1.0 &&\cr 0.1 &&\cr 0.0 &&\cr} \bigr>,$$
donde antes del punto aparece la variable de estado.  La variable de estado
sirve para recordar cuál es el valor actual de la variable de entrada, para
que esté disponible en el instante siguiente.  Al rellenar la matriz tenemos
que hacer consideraciones como la que sigue:  si el estado actual es 0 y
entra un 1, esto significa que la entrada del instante anterior fue 0, de
manera que la salida actual debe ser 0, y como ahora entra un 1 el estado
debe pasar a valer 1, de modo que la combinación (estado, entrada) 0.1
resulta en el par (estado, salida) 1.0, lo que hace que el cruce de la fila
0.1 con la columna 1.0 se lleve el 1 del suceso seguro, y las otras tres
columnas de la misma fila un 0.  Recorriendo todas las combinaciones
(estado, entrada) completamos la matriz
 $$\op{DEL} = \bigl< 1, 1, 1, 0; \pmatrix{ 1&0&0&0\cr 0&0&1&0\cr
   0&1&0&0\cr 0&0&0&1\cr} \bigr>,$$
o linealmente:
 $\op{DEL} = \bigl< 1, 1, 1, 0; ( 1,0,0,0, 0,0,1,0, 0,1,0,0, 0,0,0,1 )
 \bigr>$.

\title3 La igualdad de autómatas

La \definition{igualdad de autómatas}:  diremos que dos autómatas, |A y
$|A'$, son iguales, lo cual notaremos $|A = |A'$, si y sólo si, por
definición,
 $\inpA = \inp{\!A'}$,
 $\outA = \out{\!A'}$,
 $\stA = \st{\!A'}$ y:
 $$\forall r \through ({\stA + \inpA + \stA + \outA}) =
  ({\st{\!A'} + \inp{\!A'} + \st{\!A'} + \out{\!A'}}) : \;
  { |A_r \over 2^{\prA} } = { |A'_r \over 2^{\pr{\!\!A\!'}} }.$$

La igualdad es una relación de equivalencia:
 $$\eqalignrc{\forall |A, |B, |C :
   & |A = |A \cr
   & |A = |B \implies |B = |A \cr
   & (|A = |B) \land (|B = |C) \implies |A = |C .\cr}$$

\title4 Ejemplo

El autómata sin variables, $\O$, es:
 $\op{\O} = \bigl< 0, 0, 0, 0; (1) \bigr>$,
o lo que es lo mismo, según la definición vista:
 $\op{\O} = \bigl< 0, 0, 0, 2; (4) \bigr>$.

\title3 La forma mínima

Si todos los elementos del array $(|A_r)$ son pares, entonces podemos
dividirlos todos ellos entre 2, restar 1 a $\no P$, y tenemos otra manera
de escribir |A.  Preferiremos la manera en la que alguno de los elementos
del array $(|A_r)$ es impar, forma que llamamos de $\no P$ mínimo o,
simplemente, \definition{forma mínima}.


\title3 Las operaciones

Definimos tres operaciones sobre los autómatas:  la composición en
serie, escrita $\srl$; la composición en paralelo, escrita $\prl$; y la
composición de realimentación, escrita $\fdb$.  Usaremos la notación
prefija de las operaciones y de esta manera evitaremos el uso de
paréntesis.  Para definir las operaciones emplearemos tres autómatas
genéricos:
 $$\predisplaypenalty=-50 %%% manual breaking
 \eqalign{|A &= \left< \inpA, \outA, \stA, \prA; ({|A}_r) \right>\cr |B &=
 \left< \inp B, \out B, \st B, \pr B; ({|B}_{\dot{r}}) \right>\cr |C &=
 \left< \inp C, \out C, \st C, \pr C; ({|C}_{\ddot{r}}) \right>.\cr}$$
Nótese que los índices de |B están bajo un punto, que los índices
de |C están bajo dos puntos, y que los índices de |A no están bajo
punto alguno.

\title4 La composición en serie

La \definition{composición en serie}, $|A = \srl |B |C$, sólo
puede ser efectuada si $\out B = \inp C$, en cuyo caso:
 $$\inpA = \inp B,\quad \outA = \out C,\quad
   \stA = \st B + \st C,\quad \prA = \pr B + \pr C.$$
Para calcular $({|A}_r)$ es más sencilla la forma en cuatro
dimensiones, siendo:
 $$s = \dot{s}.2^{\st C} +\ddot{s},\quad n =
 \dot{n}.2^{\st C} + \ddot{n},\quad \imath =
 \dot{\imath},\quad o = \ddot{o},$$
y resulta (nótese que $v \through \out B = \inp C$):
 $$\ALFA = {|A}_{\dot{s}2^{\st C} + \ddot{s}, \dot{\imath},
   \dot{n}2^{\st C} + \ddot{n}, \ddot{o}} =
   \sum_v {|B}_{\dot{s}\dot{\imath}\dot{n}v} \,\cdot\,
   {|C}_{\ddot{s}v\ddot{n}\ddot{o}} .$$%\belowdisplayskip=3pt$$
\vskip-6pt
% Composición en serie
\MTbeginfigure(120,50);%\abovedisplayskip=3pt
 \MT: pickup thin_pen;
 \MT: x1 = 1/2w; y1 = 1/2h;
 \MT: rectangle(1)(w-40u,h); % 1 is automaton + B C
 \MT: pickup thick_pen;
 \MT: x2l = w - x3r = 30u;
 \MT: y2b = y3b = 10v;
 \MT: rectangle(2)(20u,20v); % 2 is B
 \MT: rectangle(3)(20u,20v); % 3 is C
 \MT: z2lbl = z2; z3lbl = z3;
 \MTlabel(2lbl)"|B";
 \MTlabel(3lbl)"|C";
 \MT: x4 = 1/2w; y4 = 0.5[y2t,h];
 \MTlabel(4)"$|A = \srl |B |C$"; % 4 is A = + B C
 \MT: pickup med_pen;
 \MT: x5l = x6l = x7l = 0; x5r = x6r = x7r = x2l;
 \MT: y5l = y5r = 1/4[y2b,y2t];
 \MT: y6l = y6r = 2/4[y2b,y2t];
 \MT: y7l = y7r = 3/4[y2b,y2t];
 \MT: arrow(5l,5r); arrow(6l,6r); arrow(7l,7r); % arrows to B / A
 \MT: x8l = x9l = x2r; x8r = x9r = x3l;
 \MT: y8l = 1/3[y2b,y2t]; y9l = 2/3[y2b,y2t];
 \MT: y8r = 1/3[y3b,y3t]; y9r = 2/3[y3b,y3t];
 \MT: arrow(8l,8r); arrow(9l,9r); % arrows from B to C
 \MT: x10l = x11l = x3r; x10r = x11r = w;
 \MT: y10l = y10r = 1/3[y3b,y3t]; y11l = y11r = 2/3[y3b,y3t];
 \MT: arrow(10l,10r); arrow(11l,11r); % arrows from C / A
\MTendfigure"Composición\cr en serie";

\smallskip
\title4 La composición en paralelo

La \definition{composición en paralelo}, $|A = \prl |B |C$, hace que:
 $$\inpA = \inp B + \inp C,\quad \outA = \out B + \out C,\quad
   \stA = \st B + \st C,\quad \prA = \pr B + \pr C.$$
Para calcular $({|A}_r)$ es más sencillo utilizar la forma en cuatro
dimensiones, siendo:
 $$s = \dot{s}.2^{\st C} + \ddot{s},\quad
   n = \dot{n}.2^{\st C} + \ddot{n},\quad
   \imath = \dot{\imath}.2^{\inp C} + \ddot{\imath},\quad
   o = \dot{o}.2^{\out C} + \ddot{o},$$
y resulta:
 $$\abovedisplayskip=\abovedisplayshortskip
 \ALFA = {|A}_{\dot{s}2^{\st C} + \ddot{s},
 \dot{\imath}2^{\inp C} + \ddot{\imath}, \dot{n}2^{\st C} + \ddot{n},
 \dot{o}2^{\out C} + \ddot{o}} = {|B}_{\dot{s}\dot{\imath}\dot{n}\dot{o}}
 \,\cdot\, {|C}_{\ddot{s}\,\ddot{\imath}\,\ddot{n}\ddot{o}}.$$
% Composición en paralelo
\MTbeginfigure(100,80);
 \MT: pickup thin_pen;
 \MT: x1 = 1/2w; y1 = 1/2h;
 \MT: rectangle(1)(w-40u,h); % 1 is automaton * B C
 \MT: pickup thick_pen;
 \MT: x2 = x3 = x4 = w/2;
 \MT: y2b = 10v;
 \MT: rectangle(2)(20u,20v); % 2 is C
 \MT: y3b = y2t + 10v;
 \MT: rectangle(3)(20u,20v); % 3 is BC
 \MT: z2lbl = z2; z3lbl = z3;
 \MTlabel(2lbl)"|C";
 \MTlabel(3lbl)"|B";
 \MT: y4 = 0.5[y3t,h];
 \MTlabel(4)"$|A = \prl |B |C$"; % 4 is A = * B C
 \MT: pickup med_pen;
 \MT: x5l = x6l = x7l = 0; x5r = x6r = x7r = x3l;
 \MT: y5l = y5r = 1/4[y3b,y3t];
 \MT: y6l = y6r = 2/4[y3b,y3t];
 \MT: y7l = y7r = 3/4[y3b,y3t];
 \MT: arrow(5l,5r); arrow(6l,6r); arrow(7l,7r); % arrows to B
 \MT: x8l = x9l = 0; x8r = x9r = x2l;
 \MT: y8l = y8r = 1/3[y2b,y2t]; y9l = y9r = 2/3[y2b,y2t];
 \MT: arrow(8l,8r); arrow(9l,9r); % arrows to C
 \MT: x10l = x11l = x3r; x10r = x11r = w;
 \MT: y10l = y10r = 1/3[y3b,y3t]; y11l = y11r = 2/3[y3b,y3t];
 \MT: arrow(10l,10r); arrow(11l,11r); % arrows from B
 \MT: x12l = x13l = x2r; x12r = x13r = w;
 \MT: y12l = y12r = 1/3[y2b,y2t]; y13l = y13r = 2/3[y2b,y2t];
 \MT: arrow(12l,12r); arrow(13l,13r); % arrows from C
\MTendfigure"Composición\cr en paralelo";

\title4 La composición de realimentación

La \definition{composición de realimentación}, $|A = \fdb |B$,
solamente puede efectuarse si $\inp B > 0$, $\out B > 0$, y entonces
resulta que:
 $$ |A = \left< \inp B - 1, \out B - 1, \st B + 1, \pr B;
    ({|B}_{\dot{r}}) \right>,$$
esto es, $\inpA = \inp B - 1$, $\outA = \out B - 1$, $\stA = \st B + 1$,
$\prA = \pr B$ y $({|A}_r) = ({|B}_{\dot{r}})$, con $r, \dot{r} \through
\stA + \inpA + \stA + \outA = \st B + \inp B + \st B + \out B$.

Obsérvese que $r$ recorre los mismos valores que $\dot{r}$ pero que sus
significados, al ser trasladados a los cuatro índices
 $(s, \imath, n, o)$,
son diferentes.  La relación entre los índices de |A y los de |B es:
 $$s = \dot{s}.2 + \breve{s}, \quad
   n = \dot{n}.2 + \breve{n}, \quad
   \dot{\imath} = \breve{\imath}.2^{\inp A} + \imath, \quad
   \dot{o} = \breve{o}.2^{\out A} + o ,$$
donde $\breve{s}$, $\breve{\imath}$, $\breve{n}$ y $\breve{o}$ son índices
binarios, que toman el valor $1$ o el valor $0$, es decir,
 $\breve{\jmath} \through 1$.
% Composición de realimentación
\MTbeginfigure(110,60);
 \MT: pickup thin_pen;
 \MT: x1 = 1/2w; y1 = 1/2h;
 \MT: rectangle(1)(w-40u,h); % 1 is automaton & B
 \MT: pickup thick_pen;
 \MT: x2 = w/2; y2b = 20v; z2lbl = z2;
 \MT: rectangle(2)(20u,20v); % 2 is B
 \MTlabel(2lbl)"|B";
 \MT: x3 = 1/2w; y3 = 0.5y2b;
 \MTlabel(3)"$|A = \fdb |B$"; % 3 is A = & B
 \MT: z4o = (x2r,2/3[y2b,y2t]); z4d = (x2l,3/4[y2b,y2t]);
 \MT: feedback(4,4o,4d)(1/2[y2t,h]+5v,10u,10u);
 \MT: x5l = x6l = 0; x5r = x6r = x2l;
 \MT: y5l = y5r = 1/4[y2b,y2t];
 \MT: y6l = y6r = 2/4[y2b,y2t];
 \MT: arrow(5l,5r); arrow(6l,6r); % arrows to B
 \MT: x7l = x2r; x7r = w;
 \MT: y7l = y7r = 1/3[y2b,y2t];
 \MT: arrow(7l,7r); % arrow from B
\MTendfigure"Composición\cr de realimentación";


\title4 Ejemplos

\MTbeginchar(40pt,20pt,1pc);
 \MT: pickup thin_pen;
 \MT: x1 = 1/2w; y1 = 1/2h;
 \MT: rectangle(1)(1/2w,h);
 \MT: pickup med_pen;
 \MT: x3 = x4 = x1l; x5 = x6 = x1r;
 \MT: y3 = y5 = 3/4[y1b,y1t]; y4 = y6 = 1/4[y1b,y1t];
 \MT: draw z3{1,0} .. z6{1,0};
 \MT: draw z4{1,0} .. z5{1,0};
 \MT: z11o = (0,y3); z11d = (x3,y3); draw z11o .. z11d;
 \MT: z12o = (0,y4); z12d = (x4,y4); draw z12o .. z12d;
 \MT: z13o = (x5,y5); z13d = (w,y5); arrow(13o,13d);
 \MT: z14o = (x6,y6); z14d = (w,y6); arrow(14o,14d);
 \MT: z0 = (1/2w,-10pt);
 \MTlabel(0)bc"$\PERM$"; % PERM
\MTendchar;
\setbox0=\box\MTbox

\MTbeginchar(60pt,30pt,1pc);
 \MT: save u,v; u := 0.5pt; v := 0.5pt;
 \MT: pickup thin_pen;
 \MT: x1 = 1/2w; y1 = 1/2h;
 \MT: rectangle(1)(2/3w,h);
 \MT: pickup med_pen;
 \MT: x3 = x4 = 1/3w; x5 = x6 = 2/3w;
 \MT: y3 = y5 = 3/6[y1b,y1t]; y4 = y6 = 1/6[y1b,y1t];
 \MT: draw z3{1,0} .. z6{1,0};
 \MT: draw z4{1,0} .. z5{1,0};
 \MT: z12o = (0,y4); z12d = (x4,y4); draw z12o .. z12d;
 \MT: z14o = (x6,y6); z14d = (w,y6); arrow(14o,14d);
 \MT: y7t = h; y7b = y7t - 1/3h; y7 = 1/2[y7b,y7t];
 \MT: x7l + x7r = w; x7r - x7l = 1/6w;
 \MT: z8 = (1/2[x5,x1r],1/2[y5,y7]);
 \MT: z9 = (1/2[x1l,x3],1/2[y3,y7]);
 \MT: draw z5{1,0} .. z8{0,1} .. (x7r,y7){-1,0};
 \MT: draw (x7l,y7){-1,0} .. z9{0,-1} .. z3{1,0};
 \MT: pickup thick_pen;
 \MT: draw (x7l,y7) -- (x7r,y7b) -- (x7r,y7t) -- cycle;
 \MT: z0 = (1/2w,-10pt);
 \MTlabel(0)bc"$\fdb\PERM$"; % & PERM
\MTendchar;
\setbox2=\box\MTbox

\MTbeginchar(40pt,10pt,1pc);
 \MT: pickup thin_pen;
 \MT: x1 = 1/2w; y1 = 1/2h;
 \MT: rectangle(1)(1/2w,h);
 \MT: pickup thick_pen;
 \MT: x4 = x1; x4r - x4l = 1/4w; x4r + x4l = 2x4;
 \MT: y4 = y1; y4t - y4b = h; y4t + y4b = 2y4;
 \MT: draw (x4r,y4) -- (x4l,y4t) -- (x4l,y4b) -- cycle;
 \MT: pickup med_pen;
 \MT: z2o = (0,y4); z2d = (x4l,y4); draw z2o .. z2d;
 \MT: z3o = (x4r,y4); z3d = (w,y4); arrow(3o,3d);
 \MT: z0 = (1/2w,-10pt);
 \MTlabel(0)bc"$\DEL$"; % DEL
\MTendchar;
\setbox4=\box\MTbox

Puesto que
 $\PERM = \bigl< 2,2,0,0; ( 1,0,0,0, 0,0,1,0, 0,1,0,0, 0,0,0,1 ) \bigr>$
y
 $\DEL = \bigl< 1,1,1,0; ( 1,0,0,0, 0,0,1,0, 0,1,0,0, 0,0,0,1 ) \bigr>$,
resulta que \hfil\break $\op{DEL} = \fdb\op{PERM}$.

\line{\hfil\box0\hfil\box2\hfil\box4\hfil}
\bigbreak

\MTbeginchar(160pt,60pt,0pt);
 \MT: pickup thick_pen;
 \MT: x1l = 100u; y1b = 0; x1lbl = x1; y1lbl.b=y1b+jot;
 \MT: rectangle(1)(20u,30v); % 1 is \string\SEL
 \MTlabel(1lbl)"\sevenrm SEL";
 \MT: x11 = x12 = x1l; x13 = x1r;
 \MT: y11 = 1/4[y1b,y1t]; y12 = 3/4[y1b,y1t]; y13 = 1/2[y1b,y1t];
 \MT: x2r = x1l - 20u; y2 = y12; z2lbl = z2;
 \MT: circle(2)(15u); % 2 is \string\PROB
 \MTlabel(2lbl)"\sevenrm Pr";
 \MT: x3 = x2; y3 = y2 + 20v; z3lbl=z3;
 \MT: or_gate(3)(15u); % 3 is \string\XOR
 \MTlabel(3lbl)"\sevenrm X ";
 \MT: pickup med_pen;
 \MT: draw z11 .. z13; draw z12 .. z13;
 \MT: z14 = (x2r,y2); arrow(14,12);
 \MT: z15 = (x3out,y3); z16 = (x1,y3); z17 = (x1,y1t);
 \MT: arroww(15,16,17);
 \MT: x18 = 0; y18 = 1/2[y11,y3in2]; z19 = z18 + (20u,0);
 \MT: draw z18 .. z19; point(19);
 \MT: x20 = x21 = x19 + 20u; y20 = y11; y21 = y3in2;
 \MT: arroww(19,20,11); arroww(19,21,3in2);
 \MT: z22 = (w,y13); arrow(13,22);
 \MT: z23 = 1/4[z13,z22];
 \MT: forkback(24,23,3in1)(h,10u,10u);
\MTendchar;

Véase que:
 $\srl \FORK \fdb \srl \srl \prl \prl \XOR \PROB \ID \SEL \FORK$, con
 $$\line{$\displaystyle{\eqalign{
 \FORK &= \left< 1, 2, 0, 0; ( 1, 0, 0, 0, 0, 0, 0, 1 ) \right> \cr
 \XOR  &= \left< 2, 1, 0, 0; ( 0, 1, 1, 0, 1, 0, 0, 1 ) \right> \cr
 \PROB &= \left< 0, 1, 0, 1; ( 1, 1 ) \right> \cr
 \ID   &= \left< 1, 1, 0, 0; ( 1, 0, 0, 1 ) \right> \cr
 \SEL  &= \left< 3, 1, 0, 0; ( 1, 0, 1, 0, 0, 1, 0, 1,
                               1, 0, 0, 1, 1, 0, 0, 1 ) \right> ,\cr}}
 \hss\vcenter{\box\MTbox\kern6pt}$%
 \tocfig{Ejemplos\string|Ejemplos de composiciones de autómatas}}$$
resulta
 $\left< 1, 1, 1, 1; ( 2, 0, 0, 0, 1, 0, 0, 1, 1, 0, 0, 1,
 0, 0, 0, 2 ) \right> = |H$,
 que es el autómata presentado en la página {\refpg{channel}}.


\title4 Abreviaturas

Utilizaremos las siguientes abreviaturas:
 $$\eqalign{
  \prl^n |A &= \underbrace{\prl \cdots \prl}_{n-1}
   \underbrace{\vphantom{\prl} |A \ldots |A}_n \quad (\prl^0 |A = \O) \cr
  \srl^n |A &= \underbrace{\srl \cdots \srl}_{n-1}
   \underbrace{\vphantom{\srl} |A \ldots |A}_n \quad (n > 0)\cr
  \fdb^n |A &= \underbrace{\fdb \cdots \fdb}_n |A\cr
 \noalign{\kern-6pt}
  \prli_{i=m}^n |A_i &= \underbrace{\prl \cdots \prl}_{m-n}
    |A_m |A_{m-1} \ldots |A_n \quad (m \geq n) .\cr}$$
 Si $m < n$, entonces $\prli_{i=m}^n |A_i = \O$.
 También: $\prli_{i \through \no N} |A_i = \prli_{i=2^{\no N}-1}^0 |A_i$.


\labeled\title3 La función característica

Llamando \definition{función característica} del autómata |A a:
 $$\FA = \left< \no S+\no I, \no S+\no O, 0, \no P; (|A_r) \right>,$$
resulta que:
 $$|A = \left< \no I, \no O, \no S, \no P; (|A_r) \right> =
  \fdb^{\no S} \left< \no S+\no I, \no S+\no O, 0, \no P; (|A_r) \right> =
   \fdb^{\no S} \FA ,$$
o sea:  $$\forall\! |A: \; |A = \fdb^{\no S} \FA .$$
% Función característica
\MTbeginfigure(110,60);
 \MT: pickup thin_pen;
 \MT: x1 = 1/2w; y1 = 1/2h;
 \MT: rectangle(1)(w-40u,h); % 1 is A
 \MT: pickup thick_pen;
 \MT: x2 = w/2; y2b = 20v; z2lbl = z2;
 \MT: rectangle(2)(20u,20v); % 2 is \FA
 \MTlabel(2lbl)"\FA";
 \MT: x3 = 1/2w; y3 = 0.5y2b;
 \MTlabel(3)"$|A$"; % 3 is A
 \MT: z4o = (x2r,2/3[y2b,y2t]); z4d = (x2l,2/3[y2b,y2t]);
 \MT: feedback(4,4o,4d)(1/2[y2t,h]+5v,10u,10u);
 \MT: x4lbl.r = x4d - jot; y4lbl.b = y4d + jot;
 \MTlabel(4lbl)"$\no S$";
 \MT: x5l = 0; x5r = x2l; y5l = y5r = 1/3[y2b,y2t];
 \MT: arrow(5l,5r); % arrow to B
 \MT: x6r = x5r - jot; y6t = y5r - jot;
 \MTlabel(6)"$\no I$";
 \MT: x7l = x2r; x7r = w; y7l = y7r = 1/3[y2b,y2t];
 \MT: arrow(7l,7r); % arrow from B
 \MT: x8l = x7l + jot; y8t = y7l - jot;
 \MTlabel(8)"$\no O$";
\MTendfigure"Función\cr característica";

\goodpage
\title3 La propiedad asociativa

\title4 La composición en serie

tiene la propiedad asociativa:
 $$\forall\! |A, |B, |C: \;
  (\outA = \inp B) \land (\out B = \inp C) \implies
   \srl\srl |A |B |C = \srl |A \srl |B |C.$$

\title4 Demostración

Es inmediato que
 $$\displaylines{
 \inp{\srl\!\srl\!|A\!|B\!|C} = \inpA = \inp{\srl\!|A\!\srl\!|B\!|C} \cr
 \out{\srl\!\srl\!|A\!|B\!|C} = \out C = \out{\srl\!|A\!\srl\!|B\!|C} \cr
 \st{\srl\!\srl\!|A\!|B\!|C} = \stA+\st B+\st C =
    \st{\srl\!|A\!\srl\!|B\!|C} \cr
 \pr{\srl\!\srl\!|A\!|B\!|C} = \prA+\pr B+\pr C =
    \pr{\srl\!|A\!\srl\!|B\!|C} .\cr}$$
%
Además (con $\outA = \inp B$ y $\out B = \inp C$):
 $$\eqalign{
 \left[\srl\srl |A |B |C\right]_{\acute s \acute\imath \acute n \acute o}
 &= \sum_{v \through \out B} \left( \sum_{v' \through \outA}
 |A_{s \imath n v'} \cdot  |B_{\dot s v' \dot n v} \right) \cdot
 |C_{\ddot s v \ddot n \ddot o}
 =\cr&= \sum_{v \through \out B} \sum_{v' \through \outA}
 |A_{s \imath n v'} \cdot  |B_{\dot s v' \dot n v} \cdot
 |C_{\ddot s v \ddot n  \ddot o}
 =\cr&= \sum_{v' \through \outA} \sum_{v \through \out B}
 |A_{s \imath n v'} \cdot  |B_{\dot s v' \dot n v} \cdot
 |C_{\ddot s v \ddot n  \ddot o}
 =\cr\noalign{\vskip-6pt}&
      = \sum_{v' \through \outA} |A_{s \imath n v'} \cdot  \left(
 \sum_{v \through \out B} |B_{\dot s v' \dot n v} \cdot
 |C_{\ddot s v \ddot n \ddot o} \right)
 =\cr&= \left[\srl |A \srl |B |C\right]_{\acute s \acute\imath \acute n
 \acute o} .}$$

\title4 La composición en paralelo

tiene la propiedad asociativa:
 $$\forall\! |A, |B, |C: \; \prl\prl |A |B |C = \prl |A \prl |B |C.$$

\title4 Demostración

Es inmediato que
 $$\displaylines{
 \inp{\prl\!\prl\!|A\!|B\!|C} = \inpA+\inp B+\inp C =
  \inp{\prl\!|A\!\prl\!|B\!|C} \cr
 \out{\prl\!\prl\!|A\!|B\!|C} = \outA+\out B+\out C =
  \out{\prl\!|A\!\prl\!|B\!|C} \cr
 \st{\prl\!\prl\!|A\!|B\!|C} = \stA+\st B+\st C =
  \st{\prl\!|A\!\prl\!|B\!|C} \cr
 \pr{\prl\!\prl\!|A\!|B\!|C} = \prA+\pr B+\pr C =
  \pr{\prl\!|A\!\prl\!|B\!|C} .\cr}$$
%
Además:
 $$\eqalign{
 \left[\prl\prl |A |B |C\right]_{\acute s\acute\imath\acute n\acute o} &=
 \left( \ALFA \cdot |B_{\dot s \dot\imath \dot n \dot o} \right) \cdot
 |C_{\ddot s \ddot\imath \ddot n \ddot o} =
 \ALFA \cdot \left( |B_{\dot s \dot\imath \dot n \dot o} \cdot
 |C_{\ddot s \ddot\imath \ddot n \ddot o} \right)
 =\cr&=
 \left[\prl |A \prl |B |C\right]_{\acute s\acute\imath\acute n\acute o}
 .}$$

\goodpage

\title3 Los elementos neutros

\title4 La composición en paralelo

tiene elemento neutro, es \O, definido:
 $$\O = \left< 0, 0, 0, 0; (1) \right>.$$
%
Se cumple que, \forallA:
 $$\eqalign{
 |A &= \prl \op\O |A \cr
 |A &= \prl |A \op\O \cr
 |A &= \prl \prl \op\O |A \op\O .\cr}$$
%
La demostración es inmediata.

\title4 La composición en serie

tiene elemento neutro, es $\ID_n$, definido:
 $$\left\{\eqalignrcccl{
  \ID_0 &=& \op\O &=& \left< 0, 0, 0, 0; (1) \right> \cr
  \ID_1 &=& \ID   &=& \left< 1, 1, 0, 0; (1, 0, 0, 1) \right> \cr
  \ID_n &&&=& \prl \ID_{n-1} \ID
    \hfill \hbox{ \siif\ } \; n\geq 2 .\cr}\right.$$
%
Se cumple que, \forallA:
 $$\eqalign{
 |A &= \srl \ID_{\no I} |A \cr
 |A &= \srl |A \ID_{\no O} \cr
 |A &= \srl \srl \ID_{\no I} |A \ID_{\no O} .\cr}$$
%
La demostración es sencilla una vez se reconoce en $\ID_n$ la forma
de la matriz identidad de tamaño $2^n \times 2^n$:
 $$\ID_n = \bigl< n, n, 0, 0; \pmatrix{ 1 & 0 & \ldots & 0 \cr
  0 & 1 & \ldots & 0 \cr \vdots & \vdots & \ddots & \vdots \cr
  0 & 0 & \ldots & 1 \cr} \bigr> . $$

\goodpage
\title3 Notación automática de vectores

Si un autómata sólo tiene variables de salida, o sea, no tiene ni
vector de entrada ni vector de estado, entonces sólo tiene el vector de
salida.  Confundiendo un autómata de estas características con el
único vector que tiene, podemos abusar de la notación que describe
autómatas para escribir vectores.  Queda:
 $$|V = \left< {\no N}, {\no P};
  (|V_r) \right> = \left< 0, {\no N}, 0, {\no P}; (|V_r) \right> .$$


\labeled\title3 Notación automática de strings

Si en \S\refsc{Notación vectorial de strings} escribimos un string
utilizando el aparato descriptivo necesario a los vectores, ahora podemos
describir un string al modo de un autómata.

Sea
 $S = [b_{\ell-1}\,b_{\ell-2}\, \ldots\, b_2\,b_1\,b_0]$
el string notado $S = \left< \ell, 0; (S_r) \right>$ con
 $S_{\overline S} = 1$ y $S_{r \not= \overline S} = 0$ en
\S\refsc{Notación vectorial de strings}.  Su forma automática será:
 $$S = \left< 0,\ell,0,0; (S_r) \right> = \prli_{i=\ell-1}^0 \K_{b_i}$$
donde esta última igualdad se sostiene porque:
 $$\eqalign{
  \K_1 = \ONE  &= \left< 0, 1, 0, 0; (1, 0) \right> \cr
  \K_0 = \ZERO &= \left< 0, 1, 0, 0; (0, 1) \right> .\cr}$$


\title3 La ampliabilidad

Siendo $S$ el string de longitud $\ell$ al que corresponde el índice
$\overline S$, definimos
 $$\Syntax(\overline S,\ell,|B) =
   \srl \prl \prli_{i=\ell-1}^0 \K_{b_i} \ID_{\inp B-\ell} |B .$$
O, abreviadamente, usando la notación automática del string:
 $$\op{Syntax}(\opd{S},|B) =
   \srl \prl \opd{S} \ID_{\inp B-\ell} |B .$$

Por ejemplo y suponiendo que $\inp B = 6$, $\op{Syntax}([0\,1\,1\,1],|B)$
puede representarse así.
% Syntax
\MTbeginfigure(80,45);
 \MT: pickup thin_pen;
 \MT: x1 = 1/2w; y1 = 1/2h; rectangle(1)(w-20u,h);
 \MT: pickup thick_pen;
 \MT: x2r = x1r - 5u; y2b = y1b + 5v; z2lbl = z2;
 \MT: rectangle(2)(35u,35v);
 \MTlabel(2lbl)"|B";
 \MT: pickup med_pen;
 \MT: z3o = (x2r,y2); z3d = (w,y2); arrow(3o,3d);
 \MT: for i := 4 upto 9: z[i]d = (x2l, ((i-3)/7)[y2b,y2t]); endfor
 \MT: z4o = (0,y4d); z5o = (0,y5d); arrow(4o,4d); arrow(5o,5d);
 \MT: for i := 6 upto 9:
 \MT:     z[i]o = z[i]d - (10u,0); arrow([i]o,[i]d);
 \MT:     x[i]lbl.r = x[i]o; y[i]lbl = y[i]o;
 \MT: endfor
 \MTlabel(9lbl)"\sevenrm 0"; \MTlabel(8lbl)"\sevenrm 1";
 \MTlabel(7lbl)"\sevenrm 1"; \MTlabel(6lbl)"\sevenrm 1";
\MTendfigure"Syntax(7,4,|B)";

\label{eqamp}
Decimos que el autómata |B es una \definition{ampliación} del autómata |A,
notado $|B \amp \!|A$, cuando:
 $$|B \amp \!|A \iff \bigl(\inp B > \inpA\bigr)
   \land \bigl(\exists c:  \Syntax(c,\inp B-\inpA,|B) = |A\bigr).$$

Por ejemplo, el autómata
 $\XOR = \left< 2, 1, 0, 0; (0, 1, 1, 0, 1, 0, 0, 1) \right>$
es una ampliación del autómata \NOT, $\XOR \amp \NOT$, y
también del autómata \ID, $\XOR \amp \ID$, ya que:
 $$\eqalign{\Syntax(1,1,\XOR) &= \srl \prl \ONE \ID \XOR = \NOT,\cr
   \Syntax(0,1,\XOR) &= \srl \prl \ZERO \ID \XOR = \ID .\cr}$$


\labeled\title3 La construcción

\title4 Minimum

Cualquier autómata puede ser escrito como una expresión en la que sólo
aparecen las tres composiciones (en serie, en paralelo y de
realimentación) y cuatro autómatas, concretamente \SINK, \FORK, \PROB\ y
\NAND.  Llamaremos conjunto Minimum al constituido por estos cuatro
autómatas, que presentamos en forma de string $(|A_r)$.
 $$\eqalign{
 \op{Minimum} &= \{\, \SINK, \FORK, \PROB, \NAND \,\} \cr
 \SINK &= \left< 1, 0, 0, 0; ( 1, 1 ) \right> \cr
 \FORK &= \left< 1, 2, 0, 0; ( 1, 0, 0, 0, 0, 0, 0, 1 ) \right> \cr
 \PROB &= \left< 0, 1, 0, 1; ( 1, 1 ) \right> \cr
 \NAND &= \left< 2, 1, 0, 0; ( 0, 1, 1, 0, 1, 0, 1, 0 ) \right>.\cr}$$

\title4 Fundamentales

Componiendo estos cuatro autómatas definimos otros que nos servirán para
construir cualquier autómata.
 $$\eqalign{
 \NOP &= \srl \PROB \SINK\cr
 \NOT &= \srl \FORK \NAND\cr
 \ID  &= \srl \NOT \NOT\cr
 \AND &= \srl \NAND \NOT\cr
 \OR  &= \srl \prl \NOT \NOT \NAND\cr
 \ONE &= \srl \srl \srl \PROB \FORK \prl \NOT \ID \NAND\cr
 \ZERO &= \srl \ONE \NOT\cr
 \XOR &= \srl \srl \srl \prl \FORK \FORK
         \prl \prl \ID \srl \NAND \FORK \ID
 \cr\noalign{\vskip-3pt}&\qquad % manual splitting
         \prl \NAND \NAND \NAND\cr
 \PERM &= \srl \srl \prl \FORK \FORK \prl \prl \ID
          \srl \XOR \FORK \ID \prl \XOR \XOR\cr
 \SEL &= \srl \srl \prl \srl \PERM \prl \ID \FORK \ID
 \cr\noalign{\vskip-3pt}&\qquad % manual splitting
         \prl \NAND \srl \prl \NOT \ID \NAND \NAND\cr}$$

\newpage

\title4 Funciones

También necesitamos algunas funciones de números naturales
 ($m,n \in \N$) en autómatas, todos ellos con $\no S = 0$ y $\no P = 0$.

\function$ \OR_n =
 \cases{\srl \prl \OR_{n-1} \ID \OR& \siif\ $n>2$\cr
  \OR& \siif\ $n=2$\cr
  \ID& \siif\ $n=1$\cr
  \ZERO& \siif\ $n=0$}\fan[n,1,0,0]$

\function$ \K_n =
 \cases{\ONE& \siif\ $n>0$\cr
        \ZERO& \siif\ $n=0$\cr} \fan[0,1,0,0] $

\MTbeginchar(40pt,20pt,1pc);
 \MT: pickup thin_pen;
 \MT: x1 = 1/2w; y1 = 1/2h; rectangle(1)(1/2w,h);
 \MT: pickup med_pen;
 \MT: x2o = x3o = x4o = 0; x2d = x3d = x4d = w;
 \MT: y2o = y2d = 1/4[y1b,y1t];
 \MT: y3o = y3d = 2/4[y1b,y1t];
 \MT: y4o = y4d = 3/4[y1b,y1t];
 \MT: arrow(2o,2d); arrow(3o,3d); arrow(4o,4d);
 \MT: z0 = (1/2w,-10pt);
 \MTlabel(0)bc"$\ID_3$"; % ID_3
\MTendchar;
\setbox0=\box\MTbox

\MTbeginchar(40pt,20pt,1pc);
 \MT: pickup thin_pen;
 \MT: x1 = 1/2w; y1 = 1/2h; rectangle(1)(1/2w,h);
 \MT: pickup med_pen;
 \MT: x2o = x3o = x4o = 0; x2d = x3d = x4d = 1/2w;
 \MT: y2o = y2d = 3/4[y1b,y1t];
 \MT: y3o = y3d = 2/4[y1b,y1t];
 \MT: y4o = y4d = 1/4[y1b,y1t];
 \MT: arrow(2o,2d); arrow(3o,3d); arrow(4o,4d);
 \MT: point(2d); point(3d); point(4d);
 \MT: z0 = (1/2w,-10pt);
 \MTlabel(0)bc"$\SINK_3$"; % SINK_3
\MTendchar;
\setbox2=\box\MTbox

\MTbeginchar(40pt,20pt,1pc);
 \MT: pickup thin_pen;
 \MT: x1 = 1/2w; y1 = 1/2h; rectangle(1)(1/2w,h);
 \MT: pickup med_pen;
 \MT: x2o = x3o = x4o = 0; x2d = x3d = x4d = x1l;
 \MT: y2o = y2d = 3/4[y1b,y1t];
 \MT: y3o = y3d = 2/4[y1b,y1t];
 \MT: y4o = y4d = 1/4[y1b,y1t];
 \MT: draw z2o .. z2d; draw z3o .. z3d; draw z4o .. z4d;
 \MT: x5o = x6o = x7o = x1r; x5d = x6d = x7d = w;
 \MT: y5o = y5d = 3/4[y1b,y1t];
 \MT: y6o = y6d = 2/4[y1b,y1t];
 \MT: y7o = y7d = 1/4[y1b,y1t];
 \MT: arrow(5o,5d); arrow(6o,6d); arrow(7o,7d);
 \MT: draw z2d{1,0} .. z6o{1,0};
 \MT: draw z3d{1,0} .. z7o{1,0};
 \MT: draw z4d{1,0} .. z5o{1,0};
 \MT: z0 = (1/2w,-10pt);
 \MTlabel(0)bc"$\op{Csr}_3$"; % Csr_3
\MTendchar;
\setbox4=\box\MTbox

\MTbeginchar(40pt,20pt,1pc);
 \MT: pickup thin_pen;
 \MT: x1 = 1/2w; y1 = 1/2h; rectangle(1)(1/2w,h);
 \MT: pickup med_pen;
 \MT: x2o = x3o = x4o = 0; x2d = x3d = x4d = x1l;
 \MT: y2o = y2d = 3/4[y1b,y1t];
 \MT: y3o = y3d = 2/4[y1b,y1t];
 \MT: y4o = y4d = 1/4[y1b,y1t];
 \MT: draw z2o .. z2d; draw z3o .. z3d; draw z4o .. z4d;
 \MT: x5o = x6o = x7o = x1r; x5d = x6d = x7d = w;
 \MT: y5o = y5d = 3/4[y1b,y1t];
 \MT: y6o = y6d = 2/4[y1b,y1t];
 \MT: y7o = y7d = 1/4[y1b,y1t];
 \MT: arrow(5o,5d); arrow(6o,6d); arrow(7o,7d);
 \MT: draw z2d{1,0} .. z7o{1,0};
 \MT: draw z3d{1,0} .. z5o{1,0};
 \MT: draw z4d{1,0} .. z6o{1,0};
 \MT: z0 = (1/2w,-10pt);
 \MTlabel(0)bc"$\op{Csl}_3$"; % Csl_3
\MTendchar;
\setbox6=\box\MTbox

\bigbreak
\line{\hfil\box0\hfil\box2\hfil\box4\hfil\box6\hfil
 \tocfig{\aut I\string|Autómata identidad}%
 \tocfig{SINK\string|Autómata sumidero}%
 \tocfig{Csr\string|Autómata de rotación a la derecha}%
 \tocfig{Csl\string|Autómata de rotación a la izquierda}}

\function$ \ID_n = \cases{
 \prl \ID_{n-1} \ID& \siif\ $n>1$\cr
 \ID& \siif\ $n=1$\cr \NOP& \siif\ $n=0$} \fan[n,n,0,0] $

\function$ \SINK_n = \cases{
 \prl \SINK_{n-1} \SINK& \siif\ $n>1$\cr
 \SINK& \siif\ $n=1$\cr \NOP& \siif\ $n=0$} \fan[n,0,0,0] $

\function$ \op{Csr}_n = \cases{
 \srl \prl \ID \op{Csr}_{n-1} \prl \PERM \ID_{n-2}& \siif\ $n>2$\cr
 \PERM&\siif\ $n=2$\cr
 \ID&\siif\ $n=1$\cr
 \NOP&\siif\ $n=0$} \fan[n,n,0,0] $

\function$ \op{Csl}_n = \cases{
 \srl \prl \op{Csl}_{n-1} \ID \prl \ID_{n-2} \PERM& \siif\ $n>2$\cr
 \PERM&\siif\ $n=2$\cr
 \ID&\siif\ $n=1$\cr
 \NOP&\siif\ $n=0$} \fan[n,n,0,0] $

\MTbeginchar(50pt,40pt,1pc);
 \MT: pickup thin_pen;
 \MT: x1 = 1/2w; y1 = 1/2h; rectangle(1)(3/5w,h);
 \MT: pickup med_pen;
 \MT: x4bl = x4ml = x4tl = 0;
 \MT: y4bl = 3/8h; y4ml = 4/8h; y4tl = 5/8h;
 \MT: x4br = x4mr = x4tr = 2/5w;
 \MT: y4br = y4bl; y4mr = y4ml; y4tr = y4tl;
 \MT: draw z4bl -- z4br; draw z4ml -- z4mr; draw z4tl -- z4tr;
 \MT: z5bl = z4br + (1/5w,2/8h); z6bl = z4br + (1/5w,-2/8h);
 \MT: z5ml = z4mr + (1/5w,2/8h); z6ml = z4mr + (1/5w,-2/8h);
 \MT: z5tl = z4tr + (1/5w,2/8h); z6tl = z4tr + (1/5w,-2/8h);
 \MT: z5br = z5bl + (2/5w,0); z6br = z6bl + (2/5w,0);
 \MT: z5mr = z5ml + (2/5w,0); z6mr = z6ml + (2/5w,0);
 \MT: z5tr = z5tl + (2/5w,0); z6tr = z6tl + (2/5w,0);
 \MT: fork(4br,5bl,5br); fork(4mr,5ml,5mr); fork(4tr,5tl,5tr);
 \MT: fork(4br,6bl,6br); fork(4mr,6ml,6mr); fork(4tr,6tl,6tr);
 \MT: z8 = (1/2w,-10pt);
 \MTlabel(8)bc"$\FORK_3$";
\MTendchar;
\setbox0=\box\MTbox

\MTbeginchar(50pt,40pt,1pc);
 \MT: pickup thin_pen;
 \MT: x1 = 1/2w; y1 = 1/2h; rectangle(1)(3/5w,h);
 \MT: pickup med_pen;
 \MT: x2o = x3o = 0; x2d = x3d = 2/5w;
 \MT: y2o = y2d = 5/9h; y3o = y3d = 4/9h;
 \MT: draw z2o -- z2d; draw z3o -- z3d;
 \MT: x4o = x5o = x6o = x7o = x8o = x9o = x2d;
 \MT: x4m = x5m = x6m = x7m = x8m = x9m = x2d + 1/5w;
 \MT: x4d = x5d = x6d = x7d = x8d = x9d = w;
 \MT: y4o = y6o = y8o = y2d; y5o = y7o = y9o = y3d;
 \MT: y4m = y4d = 8/9h; y5m = y5d = 7/9h;
 \MT: y6m = y6d = 5/9h; y7m = y7d = 4/9h;
 \MT: y8m = y8d = 2/9h; y9m = y9d = 1/9h;
 \MT: fork(4o,4m,4d); fork(5o,5m,5d); fork(6o,6m,6d);
 \MT: fork(7o,7m,7d); fork(8o,8m,8d); fork(9o,9m,9d);
 \MT: z0 = (1/2w,-10pt);
 \MTlabel(0)bc"$\op{Mult}_{3,2}$";
\MTendchar;
\setbox2=\box\MTbox

\bigbreak
\line{\hfil\box0\hfil\box2\hfil
 \tocfig{FORK\string|Autómata de bifurcación}%
 \tocfig{Mult\string|Autómata de multiplicación}}

\function$ \FORK_n = \cases{
 \srl \prl \FORK_{n-1} \FORK \prl \prl \ID_{n-1} \op{Csr}_n \ID&
 \siif\ $n>1$\cr \FORK& \siif\ $n=1$\cr
 \NOP& \siif\ $n=0$} \fan[n,2n,0,0] $

\function$ \op{Mult}_{m,n} = \cases{
 \srl \op{Mult}_{m-1,n} \prl \FORK_n \ID_{(m-2)n}
 \quad \hbox{ \siif\ } (m>2)\land(n>0)\hidewidth&\cr
 \FORK_n& \siif\ $(m=2)\land(n>0)$\cr
 \ID_n& \siif\ $(m=1)\land(n>0)$\cr
 \SINK_n& \siif\ $(m=0)\land(n>0)$\cr
 \NOP& \siif\ $n=0$} \fan[n,mn,0,0] $

\MTbeginchar(60pt,60pt,1pc);
 \MT: pickup thin_pen;
 \MT: x1 = 1/2w; y1 = 1/2h; rectangle(1)(2/3w,h);
 \MT: pickup med_pen;
 \MT: for k := 2 upto 10: x[k]o = 0; x[k]d = x1l; endfor
 \MT: y2o = y2d = 13/14h; y3o = y3d = 12/14h; y4o = y4d = 11/14h;
 \MT: y5o = y5d = 8/14h; y6o = y6d = 7/14h;
 \MT: y7o = y7d = 5/14h; y8o = y8d = 4/14h;
 \MT: y9o = y9d = 2/14h; y10o = y10d = 1/14h;
 \MT: for k := 2 upto 10: draw z[k]o .. z[k]d; endfor
 \MT: for k := 11 upto 19: x[k]o = x1r; x[k]d = w; endfor
 \MT: y11o = y11d = 13/14h; y12o = y12d = 12/14h; y13o = y13d = 11/14h;
 \MT: y14o = y14d = 8/14h; y15o = y15d = 7/14h; y16o = y16d = 6/14h;
 \MT: y17o = y17d = 3/14h; y18o = y18d = 2/14h; y19o = y19d = 1/14h;
 \MT: for k := 11 upto 19: arrow([k]o, [k]d); endfor
 \MT: draw z2d{1,0} .. z11o{1,0};
 \MT: draw z3d{1,0} .. z14o{1,0};
 \MT: draw z4d{1,0} .. z17o{1,0};
 \MT: draw z5d{1,0} .. z12o{1,0};
 \MT: draw z6d{1,0} .. z13o{1,0};
 \MT: draw z7d{1,0} .. z15o{1,0};
 \MT: draw z8d{1,0} .. z16o{1,0};
 \MT: draw z9d{1,0} .. z18o{1,0};
 \MT: draw z10d{1,0} .. z19o{1,0};
 \MT: z0 = (1/2w,-10pt);
 \MTlabel(0)bc"$\op{Join}_{3,3}$"; % Join_3,3
\MTendchar;
\setbox0=\box\MTbox

\MTbeginchar(60pt,60pt,1pc);
 \MT: pickup thin_pen;
 \MT: x1 = 1/2w; y1 = 1/2h; rectangle(1)(2/3w,h);
 \MT: pickup med_pen;
 \MT: for k := 2 upto 10: x[k]o = 0; x[k]d = x1l; endfor
 \MT: y2o = y2d = 13/14h; y3o = y3d = 12/14h; y4o = y4d = 11/14h;
 \MT: y5o = y5d = 8/14h; y6o = y6d = 7/14h; y7o = y7d = 6/14h;
 \MT: y8o = y8d = 3/14h; y9o = y9d = 2/14h; y10o = y10d = 1/14h;
 \MT: for k := 2 upto 10: draw z[k]o .. z[k]d; endfor
 \MT: for k := 11 upto 19: x[k]o = x1r; x[k]d = w; endfor
 \MT: y11o = y11d = 13/14h; y12o = y12d = 12/14h; y13o = y13d = 11/14h;
 \MT: y14o = y14d = 8/14h; y15o = y15d = 7/14h;
 \MT: y16o = y16d = 5/14h; y17o = y17d = 4/14h;
 \MT: y18o = y18d = 2/14h; y19o = y19d = 1/14h;
 \MT: for k := 11 upto 19: arrow([k]o, [k]d); endfor
 \MT: draw z2d{1,0} .. z11o{1,0};
 \MT: draw z3d{1,0} .. z14o{1,0};
 \MT: draw z4d{1,0} .. z15o{1,0};
 \MT: draw z5d{1,0} .. z12o{1,0};
 \MT: draw z6d{1,0} .. z16o{1,0};
 \MT: draw z7d{1,0} .. z17o{1,0};
 \MT: draw z8d{1,0} .. z13o{1,0};
 \MT: draw z9d{1,0} .. z18o{1,0};
 \MT: draw z10d{1,0} .. z19o{1,0};
 \MT: z0 = (1/2w,-10pt);
 \MTlabel(0)bc"$\op{Disjoin}_{3,3}$";
\MTendchar;
\setbox2=\box\MTbox

\MTbeginchar(50pt,40pt,1pc);
 \MT: pickup thin_pen;
 \MT: x1 = 1/2w; y1 = 1/2h; rectangle(1)(3/5w,h);
 \MT: pickup med_pen;
 \MT: x2o = x3o = x4o = x5o = x6o = x7o = 0;
 \MT: x2d = x3d = x4d = x5d = x6d = x7d = x1l;
 \MT: y2o = y2d = 8/9h; y3o = y3d = 7/9h; y4o = y4d = 6/9h;
 \MT: y5o = y5d = 3/9h; y6o = y6d = 2/9h; y7o = y7d = 1/9h;
 \MT: for k := 2 upto 7: draw z[k]o .. z[k]d; endfor
 \MT: x12o = x13o = x14o = x15o = x16o = x17o = x1r;
 \MT: x12d = x13d = x14d = x15d = x16d = x17d = w;
 \MT: y12o = y12d = 8/9h; y13o = y13d = 7/9h; y14o = y14d = 5/9h;
 \MT: y15o = y15d = 4/9h; y16o = y16d = 2/9h; y17o = y17d = 1/9h;
 \MT: for k := 12 upto 17: arrow([k].o,[k].d); endfor
 \MT: draw z2d{1,0} .. z12o{1,0}; draw z3d{1,0} .. z14o{1,0};
 \MT: draw z4d{1,0} .. z16o{1,0}; draw z5d{1,0} .. z13o{1,0};
 \MT: draw z6d{1,0} .. z15o{1,0}; draw z7d{1,0} .. z17o{1,0};
 \MT: z0 = (1/2w,-10pt);
 \MTlabel(0)bc"$\op{Rea}_{2,3}$";
\MTendchar;
\setbox4=\box\MTbox

\bigbreak
\line{\hfil\box0\hfil\box2\hfil\box4\hfil
 \tocfig{Join\string|Autómata de unión}%
 \tocfig{Disjoin\string|Autómata de separación}%
 \tocfig{Rea\string|Autómata de reagrupación}}

\function$ \op{Join}_{m,n} = \cases{
 \srl \prl \prl \ID_{m-1} \op{Csl}_{(m-1)(n-1)+1}
 \ID_{n-1} \prl \op{Join}_{m-1,n} \ID_n
 \hidewidth \cr& \siif\ $(m>2)\land(n>1)$\cr
 \prl \prl \ID \op{Csl}_n \ID_{n-1}& \siif\ $(m=2)\land(n>1)$\cr
 \ID_n& \siif\ $(m=1)\land(n>0)$\cr
 \ID_m& \siif\ $(m>0)\land(n=1)$\cr
 \NOP& \siif\ $(m=0)\lor(n=0)$} \fan[mn,mn,0,0] $ % \no I = m+m(n-1)

\function$ \op{Disjoin}_{m,n} = \cases{
 \srl \prl \op{Disjoin}_{m-1,n} \ID_n \prl \prl \ID_{m-1}
 \op{Csr}_{(m-1)(n-1)+1} \ID_{n-1}
 \hidewidth \cr& \siif\ $(m>2)\land(n>1)$\cr
 \prl \prl \ID \op{Csr}_n \ID_{n-1}& \siif\ $(m=2)\land(n>1)$\cr
 \ID_n& \siif\ $(m=1)\land(n>0)$\cr
 \ID_m& \siif\ $(m>0)\land(n=1)$\cr
 \NOP& \siif\ $(m=0)\lor(n=0)$} \fan[mn,mn,0,0] $ % \no O=m+m(n-1)

\function$ \op{Rea}_{m,n} = \cases{ % from m of n bit to n of m bit
 \srl \op{Disjoin}_{m,n} \prl \ID_m \op{Rea}_{m,n-1}
 \quad \hbox{ \siif\ } (m>0)\land(n>2)\hidewidth&\cr
 \op{Disjoin}_{m,2}& \siif\ $(m>0)\land(n=2)$\cr
 \ID_m& \siif\ $(m>0)\land(n=1)$\cr
 \NOP& \siif\ $(m=0)\lor(n=0)$} \fan[mn,nm,0,0] $

\bigskip

\function$ \SEL_n = \cases{
 \srl \srl \prl \op{Mult}_{n,1} \op{Rea}_{2,n} \op{Join}_{n,3}
 \prl^n \SEL& \siif\ $n>1$\cr
 \SEL& \siif\ $n=1$\cr \SINK& \siif\ $n=0$} \fan[1+2n,n,0,0] $

\function$ \op{Cand}_n = \cases{
 \srl \srl \prl \op{Mult}_{n,1} \ID_n \op{Rea}_{2,n} \prl^n \AND&
  \siif\ $n>1$\cr \AND& \siif\ $n=1$\cr
 \SINK& \siif\ $n=0$} \fan[1+n,n,0,0] $

\function$ \op{Coder}_n = \cases{
 \srl \srl \srl \prl \FORK_{2^{n-1}} \ID_{2^{n-1}}
 \prl \prl \OR_{2^{n-1}} \op{Coder}_{n-1} \op{Coder}_{n-1}
\hidewidth&\cr\quad % manual spliting
 \prl \FORK \ID_{2n-2} \prl \ID \SEL_{n-1} & \siif\ $n>1$\cr
 \prl \ID \SINK& \siif\ $n=1$\cr
 \SINK& \siif\ $n=0$} \fan[2^n,n,0,0] $

\function$ \op{Switch}_n = \cases{
 \srl \srl \FORK_{2n+1} \prl \NOT \ID_{4n+1}
&\cr\quad % manual spliting
 \prl \SEL_n \SEL_n& \siif\ $n>0$\cr
 \SINK & \siif\ $n=0$} \fan[1+2n,2n,0,0] $

\function$ \op{Decoder}_n = \cases{
 \srl \prl \prl \ID \prl^{2^{n-1}} \ZERO \op{Decoder}_{n-1}
 &\cr\quad % splitted
 \op{Switch}_{2^{n-1}}& \siif\ $n>1$\cr
 \srl \FORK \prl \ID \NOT& \siif\ $n=1$\cr
 \ONE& \siif\ $n=0$} \fan[n,2^n,0,0] $


\break

\title4 La forma canónica

Por fin la expresión $\op{Canon}_{|A}$, definida a partir de la forma
unidimensional de
 $|A = \bigl< \no I,\no O,\no S,\no P; (|A_r) \bigr>$,
que es importante porque puede desarrollarse hasta que sólo aparezcan las
siete palabras
 $\{\srl$, $\prl$, $\fdb$, \SINK, \FORK, \PROB, \NAND $\}$,
y porque,
 $\forallA$, $\op{Canon}_{|A} = |A$:
 $$\eqalign{\op{Canon}_{|A} = \fdb^{\no S} \srl \srl \srl \srl \srl
  &\op{Decoder}_{\no S+\no I}\cr
  &\op{Shape}_{\!\!|A}\cr
  &\prl^{2^{\no S+\no I}} \op{Cand}_{2^{\no S+\no O}}\cr
  &\op{Rea}_{2^{\no S+\no I},2^{\no S+\no O}}\cr
  &\prl^{2^{\no S+\no O}} \OR_{2^{\no S+\no I}}\cr
  &\op{Coder}_{\no S+\no O}}$$
donde $\op{Shape}_{\!\!|A}$,\vadjust{\smallskip}\hfil\break
 con $\inp{\rm Shape} = 2^{\no S+\no I}$,
     $\out{\rm\, Shape} = 2^{\no S+\no I}.(1+2^{\no S+\no O})$,
     $\st{\rm Shape} = 0$, $\pr{\rm Shape} =\no P$,
es:
\function$ \op{Shape}_{\!\!|A} = \cases{
  \srl \prl \ID_{2^{\no S+\no I}}
  \srl \srl \srl \prl^{\no P} \PROB
  \op{Decoder}_{\no P} \op{Mult}_{2^{\no S+\no I},2^{\no P}}
\hidewidth&\cr\qquad % manual splitting
  \prli_{r \,\through\, (\no S+\no I+\no S+\no O)} \OR_{|A_r}
  \op{Join}_{2^{\no S+\no I},1+2^{\no S+\no O}}& \siif\ $\no P>0$\cr
 \noalign{\vskip3pt}
  \srl \prl \ID_{2^{\no S+\no I}}
  \prli_{r \,\through\, (\no S+\no I+\no S+\no O)} \K_{|A_r}
  \op{Join}_{2^{\no S+\no I},1+2^{\no S+\no O}}& \siif\ $\no P=0$.}$

\medskip
Decimos que $\op{Canon}_{|A}$ es una \definition{forma canónica} de
expresar el autómata |A.  Para expresar que cualquier autómata |A puede
ser escrito utilizando, únicamente, las siete palabras, escribiremos:
 $$ \{ |A \} = \Re(\srl, \prl, \fdb, \SINK, \FORK, \PROB, \NAND ).$$

\title4 Comentarios

Los autómatas que pertenecen al conjunto Minimum son independientes, es
decir, no es posible construir uno de ellos por composición de los otros
tres.  Son, pues, una base del álgebra automática.

Pero el conjunto Minimum no es el único capaz de generar cualquier
autómata.  Otros conjuntos de estas características son:
 $$\eqalign{
  \op{Minimum}_1 &=
   \{\, \SINK, \FORK, \PROB, \op{NOR} = \srl \OR \NOT \,\}\cr
  \op{Minimum}_2 &=
   \{\, \SINK, \ID, \prl^n \PROB, \prl \FORK \NAND \,\}
   \;\; \forall n>0.\cr}$$
Sólo he encontrado bases de cuatro miembros.


\title3 La probabilidad binaria

Si se define la \definition{probabilidad binaria} como aquella probabilidad
que describe la ocurrencia de cualquiera de las combinaciones booleanas de
los eventos generados por $n$ dispositivos binarios de máxima entropía,
entonces las probabilidades manejadas por el álgebra automática son
probabilidades binarias y sus autómatas son doblemente binarios.


\title3 Otra construcción

\title4 Otra forma

También se pueden escribir autómatas rellenando el array, en vez de con
probabilidades, con los valores de estado y de salida.  Por ejemplo:
 $$\op{AND} = \bigl< 2, 1, 0, 0; \left[\matrix{ [1]\cr [0]\cr [0]\cr
 [0]\cr}\right] \bigr>$$
donde en cada fila hay un solo string porque
$2^{\pr{\op{AND}}}=1$ y cada string sólo tiene un bit porque
$\st{\op{AND}}+\out{\op{AND}}=1$.

Otro ejemplo es:
 $$|H = \bigl< 1, 1, 1, 1;
 \pmatrix{ 2&0&0&0\cr 1&0&0&1\cr 1&0&0&1\cr 0&0&0&2\cr} \bigr> =
 \bigl< 1, 1, 1, 1; \left[\matrix{
 [1\,1]&[1\,1]\cr [1\,1]&[0\,0]\cr [1\,1]&[0\,0]\cr [0\,0]&[0\,0]\cr}\right]
 \bigr>.$$

\label{forma A star}
Notaremos $[|A^*_{pu}]$ la matriz de esta forma de representar autómatas.
Luego, si $|A_{pq} = n$, entonces en la fila $|A^*_p$ aparecerá $n$ veces el
string correspondiente al índice $q$.  Es decir, $u \through \no P$ y
$|A^*_{pu}$ es un string de tamaño $\no S + \no O$. Ambas notaciones
contienen la misma información.

Linealizando $[|A^*_{pu}]$ y empalmando los
 $2^{\no S+\no I} . 2^{\no P}$
strings que resultan, obtenemos un string binario $[|A^*_z]$ de longitud
 $\no L = (\no S + \no O) . 2^{\no S+\no I+\no P}$.
Por ejemplo, la forma linealizada de $\AND$ es:
 $\AND = \bigl< 2, 1, 0, 0; \left[1\, 0\, 0\, 0 \right] \bigr>$.


\title4 La construcción

Usando esta manera de representar autómatas, cuya forma linealizada es
 $|A = \bigl< \no I,\no O,\no S, \no P; \left[|A^*_z\right] \bigr>$,
podemos utilizar la construcción más sencilla $\op{Kanon}$, que puede
ser desarrollada hasta que sólo aparezcan las mismas siete palabras y tal
que,
\forallA, $\op{Kanon}_{|A} = |A$:
 $$\op{Kanon}_{|A} = \srl \prl \prli_{z=\no L-1}^0 \K_{|A^*_z} \ID_{\no I}
   \op{Universal}_{\no I,\no O,\no S,\no P}$$
 donde $\no L=(\no S+\no O).2^{\no S+\no I+\no P}$. \hfil\break
 O de manera más compacta, usando la notación automática del string
 $[|A^*_z]$:
 $$\op{Kanon}_{|A} = \srl \prl \opd{[|A^*_z]} \ID_{\no I}
      \op{Universal}_{\no I,\no O,\no S,\no P} .$$

\goodpage

\title4 Universal

Usa un autómata universal:

\bigskip\noindent$\op{Universal}_{i,o,s,p} = \cases{
 \srl \op{Cross}_{(s+o)2^{s+i+p},i} \fdb^s \srl \prl \prl^p \PROB
\cr\qquad % manual spliting
  \op{Selector}_{s+i,(s+o)2^p} \op{Selector}_{p,s+o}& \siif\ $p>0$
\cr\noalign{\vskip3pt}
 \srl \op{Cross}_{(s+o)2^{s+i},i}
    \fdb^s \op{Selector}_{s+i,s+o}& \siif\ $p=0$.}$

\medskip
El autómata $\op{Universal}_{i,o,s,p}$
 (con $\no I = (s+o).2^{s+i+p}+i$, $\no O = o$, $\no S = s$ y $\no P = p$)
es un \definition{autómata universal}, y de aquí su nombre, porque es
una ampliación de cualquier autómata |A con $\inpA = i$, $\outA = o$,
$\stA = s$ y $\prA = p$.


\title4 Funciones

\label{Cross}
Y también usa dos funciones adicionales.

\function$ \op{Cross}_{m,n} = \cases{
 \srl \prl \op{Cross}_{m,n-1} \ID \prl \ID_{n-1} \op{Csr}_{m+1}
\hidewidth \cr & \siif\ $(m>0)\land(n>1)$\cr % manual adjustment
 \op{Csr}_{m+1}& \siif\ $(m>0)\land(n=1)$\cr
 \ID_m& \siif\ $(m>0)\land(n=0)$\cr
 \ID_n& \siif\ $m=0$} \fan[m+n,n+m,0,0] $

\function$ \op{Selector}_{m,n} = \cases{
 \srl \srl \prl \op{Csl}_m \ID_{n2^m} \prl \ID_{m-1}
 \SEL_{n2^{m-1}} \op{Selector}_{m-1,n}
 \hidewidth \cr & \siif\ $(m>1)\land(n>0)$\cr % manual adjustment
 \SEL_n& \siif\ $(m=1)\land(n>0)$\cr
 \ID_n& \siif\ $(m=0)\land(n>0)$\cr
 \SINK_m& \siif\ $n=0$} \fan[m+n2^m,n,0,0] $

\MTbeginchar(50pt,40pt,1pc);
 \MT: pickup thin_pen;
 \MT: x1 = 1/2w; y1 = 1/2h; rectangle(1)(3/5w,h);
 \MT: pickup med_pen;
 \MT: x2o = x3o = x4o = x5o = x6o = 0;
 \MT: x2d = x3d = x4d = x5d = x6d = x1l;
 \MT: y2o = y2d = 7/8h; y3o = y3d = 6/8h;
 \MT: y4o = y4d = 3/8h; y5o = y5d = 2/8h; y6o = y6d = 1/8h;
 \MT: for k := 2 upto 6: draw z[k]o .. z[k]d; endfor
 \MT: x12o = x13o = x14o = x15o = x16o = x1r;
 \MT: x12d = x13d = x14d = x15d = x16d = w;
 \MT: y12o = y12d = 7/8h; y13o = y13d = 6/8h; y14o = y14d = 5/8h;
 \MT: y15o = y15d = 2/8h; y16o = y16d = 1/8h;
 \MT: for k := 12 upto 16: arrow([k].o,[k].d); endfor
 \MT: draw z2d{1,0} .. z15o{1,0}; draw z3d{1,0} .. z16o{1,0};
 \MT: draw z4d{1,0} .. z12o{1,0}; draw z5d{1,0} .. z13o{1,0};
 \MT: draw z6d{1,0} .. z14o{1,0};
 \MT: z0 = (1/2w,-10pt);
 \MTlabel(0)bc"$\op{Cross}_{2,3}$";
\MTendchar;
\vskip1pc
\noindent
$$\box\MTbox\tocfig{Cross\string|Autómata de cruce}$$


% EPI3.TEX (RMCG19980818)

\chapter ¿Qué vale como explicación?

Si llueve, entonces se moja el suelo. Por esto decimos que la lluvia es una
condición suficiente para que el suelo esté mojado. Pero hay otras: si se
riega, también se moja el suelo. Es necesario, pero no suficiente, que no
llueva para que el suelo esté seco. También es necesario que no se haya
regado.

Con estos datos elaboramos el siguiente algoritmo de resolución. Cuando
vamos a salir de casa, observamos el suelo de la calle para determinar si
debemos tomar el paraguas o no. Éste es el problema. Si el suelo está seco,
entonces sabemos que no llueve y que, por lo tanto, no es necesario el
paraguas. Si está mojado, entonces hemos de confirmar si llueve, para tomar
en este caso el paraguas, o si acaban de regar para no tomarlo.

Las causas son las condiciones suficientes para que ocurra el efecto. La
explicación causal es aquélla que usa causas y efectos, de tal suerte que
las causas sirven como explicación de los efectos. Luego, si queremos saber
por qué está mojado el suelo, y descubrimos que está lloviendo, explicamos
que el suelo está mojado porque está lloviendo. Esta explicación causal
pide, a su vez, otra explicación, por qué llueve, de modo que la lluvia no
es la explicación final del suelo mojado. Pero, en este caso, como en la
totalidad de los casos prácticos, no se busca la explicación final. Y es que
si el problema consiste en determinar si debemos, o no, tomar el paraguas,
saber que llueve es suficiente para tomarlo, según el algoritmo que acabamos
de ver.

\chapter La explicación final

Aunque en la práctica es suficiente aquella explicación que nos permite
solucionar el problema que se nos plantea, el hecho de que las explicaciones
sean artefactos simbólicos implica que podemos preguntarnos el porqué de
cada explicación sin encontrar nunca una explicación final, o causa primera,
que no admita una explicación ulterior, que no tenga una causa previa.

Hay quien dice que tal secuencia inacabable de explicaciones no es posible,
que consiguientemente ha de haber una explicación final, y que tal causa
primera es Dios, cuya existencia queda así probada. Pero el argumento es
falaz. En un diccionario no existen palabras que no puedan ser explicadas
con otras palabras.

\goodpage

Y, sin embargo, ya que existe un problema primero del que se derivan todos
los demás problemas, sí que podría existir una explicación final. La
explicación final sería aquélla que sirviera para solucionar el problema
primero, o sea, el problema aparente de la supervivencia. Lamentablemente
los problemas aparentes no tienen solución.

Así que lo último no es una explicación, sino un problema. O lo primero,
según se mire.

\chapter La existencia

Caballo es un nombre común, como piedra. Esto quiere decir que caballo es el
nombre que reciben las soluciones de cierto problema que fija las
condiciones que debe cumplir una cosa para ser denominada caballo. Los
simbolismos permiten asignar nombres a cualquier construcción simbólica y
permiten combinar cualesquiera condiciones. Así, añadiendo la condición de
tener un único cuerno en el centro de su frente a la definición de caballo,
tenemos la definición del unicornio.

Decimos que los unicornios no existen porque no hay ningún animal que cumpla
tal definición. Que haya nombres sin referente no es una limitación del
simbolismo, sino su virtud. Recordemos que para representar la libertad de
un problema se usan nombres sin referente. Así, tiene interés conocer si un
determinado nombre tiene, o no tiene, referente. Y por esta razón decimos
que existe aquello que tiene referente, y que no existe aquello que no lo
tiene.

¿Qué pruebas de la existencia son suficientes? Aquéllas que nos permitan
solucionar el problema al que nos estemos enfrentando en ese preciso
momento. Si quisiéramos colocarnos en un plano absoluto, olvidándonos del
problema, entonces nunca encontraríamos pruebas suficientes, dada la
naturaleza simbólica de las pruebas.

\vfil
\centerline{\it Je pense donc je suis}
\centerline{\it Cogito ergo sum\cite{Descartes1641}}
\vfil
\break

\chapter ¿Existe este libro?

En general damos por hecho que los objetos materiales, como este libro,
existen. Aceptamos como prueba, en condiciones normales, la proporcionada
por nuestros sentidos. Esto es así porque los objetos físicos suelen ser
útiles o herramientas o, por el contrario, impedimentos o condicionantes en
nuestros problemas. Sin embargo, \person{Descartes} argumentó que nuestros
sentidos pueden sufrir ilusiones, por lo que no son siempre fiables, y
concluyó que lo único indubitable era la existencia del yo.



\chapter ¿Existo yo?

\person{Descartes} estaba en lo cierto. Para el sujeto el yo es
la solución del problema de la supervivencia. Negar el yo sería negar el
problema primero. Que finalmente muramos, o sea, que el problema de la
supervivencia no tenga solución, no impide que seamos resolutores del
problema de la supervivencia, es decir, no evita que busquemos la solución.
Por el contrario, si el problema de la supervivencia tuviera solución,
entonces no seríamos resolutores, sino soluciones.

\chapter ¿Existe España?

Siempre que le solucione algún problema a alguien o, por el contrario,
mientras sea un impedimento o condicionante a alguno, existirá España.
Nótese que Mesopotamia también existe, según estos principios, para los
arqueólogos.

\chapter ¿Existe Dios?

Este asunto se deja como ejercicio al lector.

\vfil
\MTbeginfigure(80,80);
 \MT: save u; u = w/6;
 \MT: y2 = y3 = h - y1 = u;
 \MT: x1 = w/2;
 \MT: z1 - z2 = (z3 - z2) rotated 60;
 \MT: for i := 1 upto 3:
 \MT:  fill fullcircle scaled (2u) shifted (z[i]);
 \MT: endfor
 \MT: unfill z1 -- z2 -- z3 -- cycle;
\MTendfigure"No hay triángulo"Figura adaptada de \person{Resnikoff};
\vfil

\break


\chapter El conocimiento

Sabemos que el nivel jerárquico superior de control en las personas, que
denominamos consciencia, es simbólico. Llamamos conocimiento o saber a la
información consciente, que es simbólica.

No toda la información que trata y maneja nuestro cerebro es conocimiento.
Según un ejemplo ya visto, el control de la frecuencia de latido del
corazón, efectuado por el cerebro, no es consciente y, por consiguiente, no
es conocimiento según nuestra definición.


\chapter El absolutismo

Es difícil fijar los límites del conocimiento, ya que, por definición, no es
posible considerar conscientemente aquello que no se puede saber. Por esta
razón parece que todo puede ser sabido. Llamaremos absolutismo a esta
creencia en la posibilidad de un conocimiento completo.

Por ejemplo, y contra lo dicho antes, los mecanismos de control de la
frecuencia de latido del corazón pueden ser conocidos, y lo son, por los
estudiosos de la medicina. Pero no es lo mismo conocer el mecanismo de
control que ejercer efectivamente un control consciente sobre la frecuencia
del latido. Los absolutistas pueden alegar, no obstante, que hay quien
puede, incluso, llegar a tener ese control consciente del corazón.

Otro ejemplo es el diccionario, que parece ser capaz de definir todas y cada
una de las palabras de la lengua.

\chapter Hawking

\person{Hawking}\cite{Hawking1988} es absolutista. Cree posible enunciar una
teoría unificada que dé cuenta de todo cuanto acontece. Opina, además, que
seguramente sólo existe una teoría que salve los hechos y sea coherente. De
ser así, concluye, Dios no habría tenido ninguna libertad en la creación del
universo. Ni tan siquiera para establecer las condiciones iniciales.


\vfil
\MTbeginfigure(105,30);
 \MT: pickup thick_pen;
 \MT: save u, v; u = w/7; v = w/7;
 \MT: y1 = y2 = y3 = h/2; x1 = 0; x3 + v = w; x2 = (x1 + x3)/2;
 \MT: x1u = x1d = x1 + v; y1u = y1 + v; y1d = y1 - v;
 \MT: x2u = x2d = x2 - v; y2u = y2 + v; y2d = y2 - v;
 \MT: x3u = x3d = x3 + v; y3u = y3 + v; y3d = y3 - v;
 \MT: draw z1 .. z1u; draw z1 .. z1d;
 \MT: draw z2 .. z2u; draw z2 .. z2d;
 \MT: draw z3 .. z3u; draw z3 .. z3d;
\MTendfigure"Los vértices laterales\cr son equidistantes del central"%
             Figura adaptada de \person{Resnikoff};
\vfil
\break


\chapter Penrose

\person{Penrose}\cite{Penrose1989}, además de coincidir con \person{Hawking},
expone que el razonamiento deductivo formal no puede ser completo, según el
teorema de indecidibilidad de \person{G\"odel}, pero que es notorio que
existen máquinas capaces de ir más lejos, a saber, cualquier cerebro que
siga el teorema de \person{G\"odel}, por lo que infiere que existe una
lógica más poderosa que las matemáticas que, aventura, podría ser aquélla
que se esconde tras las paradojas cuánticas.

Es decir, \person{Penrose} reconoce que es imposible alcanzar el
conocimiento completo en un simbolismo y, en vez de aceptar esta limitación,
la aporta como prueba de la existencia de un metasimbolismo. Si este
metasimbolismo cuántico fuera completo, entonces podría justificarse el
absolutismo.

\chapter Refutación del absolutismo

¿Cómo rebatir el absolutismo con toda esta evidencia a su favor?

Lo primero es admitir que no hay cosa alguna que no pueda ser explicada.
Todo puede ser, efectivamente, explicado. Sabemos que los simbolismos son
sistemas cerrados, y por lo tanto, ilimitados, ya que es posible permanecer
en ellos cuanto se quiera. La cuestión es que no tiene interés permanecer
indefinidamente en ellos. No tiene sentido quedarse por siempre en el
simbolismo, porque el simbolismo es un medio, una herramienta, y no un fin
en sí mismo.

Un pianista de extraordinaria técnica puede preferir tocar una pieza de
enorme dificultad, para mostrar así toda su habilidad, a interpretar otra
más sencilla, aunque ésta última tenga un mayor valor musical. Esto es una
perversión musical, ya que la técnica es un medio y no un fin, en la que el
oyente sale perdiendo. Las exhibiciones circenses pueden ser muy
entretenidas, pero no debe confundirse el tocino con la velocidad.

El simbolismo es una herramienta para resolver problemas. Las herramientas
sólo deben ser usadas para realizar aquellas tareas para las que son útiles.
Y, sobre todo, una vez concluida la tarea para la cual la herramienta ayuda,
debe guardarse la herramienta, porque, a partir de entonces, sólo estorba.
En el caso del simbolismo, quedarse por siempre en el simbolismo es lo que
ocurre con una paradoja como `esta frase es falsa' que si es verdadera,
entonces es falsa, y entonces es verdadera, y entonces es falsa, etc.

Dicho de otro modo. Que podamos vivir sin que las paradojas nos causen
dificultades no prueba la existencia de un metasimbolismo ignoto, sino que
el conocimiento es un medio y que no es absoluto. El simbolismo es sólo un
medio y, en consecuencia, el conocimiento es un medio, no un fin.

Las expresiones simbólicas pueden no tener significado. Estas explicaciones
vacías de significado, paradójicas, son falso conocimiento, ya que no
aportan información, aunque lo parezca. Por lo tanto, todo puede ser
explicado, pero no toda explicación contiene conocimiento.


\chapter El escepticismo

Así como el absolutismo es optimista con respecto a la posibilidad del
conocimiento, el escepticismo es pesimista.

\person{Hume}\cite{Hume1748} enuncia con claridad el problema. La regla de
la inferencia postula que, si algo ha ocurrido cada día hasta hoy, entonces
también ocurrirá mañana. Merced a ella, podemos hacer inferencias
racionales, por ejemplo, predecir que mañana saldrá el sol porque todas las
mañanas, hasta hoy, ha salido el sol por el levante. Sin embargo, la propia
operación de inferir no es racional, o mejor dicho, no puede ser fundada
racionalmente. No puede fundarse en hechos, como que el sol sale todos los
días, porque incurriríamos en un círculo vicioso.

\chapter Popper

\person{Popper}\cite{Popper1972} actualizó el pensamiento de \person{Hume}
afirmando que una teoría puede ser refutada por un hecho que la falsee, pero
que una teoría nunca, por más evidencia que acumule, puede ser verificada
por los hechos. Es decir, que, si un día no saliera el sol, entonces la
teoría de que todos los días sale el sol quedaría refutada, pero que haya
salido el sol cada día no asegura que saldrá mañana.

\person{Popper} deja abierta la posibilidad de un absolutismo asintótico, ya
que una teoría puede acumular cada vez más evidencia, hasta hacerse casi
completamente cierta.

Estamos usando el término {\em teoría} como una red de conceptos
relacionados. Los conceptos son los elementos de la teoría. Tanto los
conceptos como las teorías son construcciones simbólicas y, dada la
naturaleza de los simbolismos, tanto una teoría como un concepto pueden ser
designados por una palabra que le sirve de etiqueta. Un hecho sería un
concepto independiente de cualquier teoría, es decir, absoluto.


\chapter Kuhn

\person{Kuhn}\cite{Kuhn1970} distingue, en el progreso de la ciencia,
períodos normales en los que se desarrollan los aspectos secundarios de las
teorías, y períodos revolucionarios en los que sobreviene un cambio de
teoría, o de paradigma si utilizamos su terminología. Observa que ninguna
teoría es en la práctica refutada por hechos que la falsean, esto es, por
contrajemplos, sino que es refutada solamente cuando aparece otra teoría que
es capaz de explicar estos contraejemplos.

\chapter Feyerabend

Para \person{Feyerabend}\cite{Feyerabend1988} la diferencia entre hechos y
teoría no es tan nítida como parece, e incluso piensa que no hay tal
diferencia. Por ejemplo, la teoría de que el sol sale cada día da por
supuesto que el día y la salida del sol son dos hechos puros, no teñidos de
teoría, independientes. Pero no es así. Se puede decir tanto que un día no
salió el sol como que un día duró cuarenta y ocho horas. La primera manera
de decir refuta la teoría de que el sol sale cada día, pero no la segunda.

Tras descubrir que no hay hechos puros, esto es, hechos independientes de
las teorías, \person{Feyerabend} concluye que no existe una manera racional
de determinar qué teoría es mejor. Las teorías son inconmesurables ya que
cada una explica sus propios hechos.

\chapter Refutación del escepticismo

Es cierto que no existen los hechos como conceptos absolutos, tal como han
descubierto \person{Kuhn} y, sobre todo, \person{Feyerabend}, que llega a
afirmar que todo vale para conseguir que triunfe una teoría, repitiendo así
a los sofistas. Pero esta última posición escéptica ya no se sostiene.

A pesar de que la postura escéptica se oponga a la absolutista, su defecto
es básicamente el mismo. La teoría científica no es más que una explicación
simbólica, y como tal no es un fin es sí misma, sino un medio para resolver
problemas y, en último término, el problema de la supervivencia.

Teóricamente se puede conceder prioridad a la mecánica relativista sobre la
mecánica clásica, porque la mecánica relativista es capaz de fijar los
límites dentro de los cuales es seguro utilizar la mecánica clásica, pero no
a la inversa. Pero la mecánica clásica de \person{Newton} sigue siendo
utilizada. Sus matemáticas son más sencillas que las de la mecánica
relativista de \person{Einstein}, de modo que, cuando en el problema a
resolver las velocidades son mucho menores que la velocidad de la luz, es
más útil la mecánica clásica que la relativista. Luego, en la resolución de
problemas prácticos, la mecánica newtoniana sigue viva, aunque haya de
recurrirse a la mecánica einsteniana para certificar su validez.

No se trata de determinar si el concepto de simultaneidad o de masa de
\person{Newton} es erróneo o no, sino de dilucidar si tales conceptos
solucionan o no el problema que se nos plantea. Sólo cuando al concepto de
masa, o a cualquier otra construcción simbólica, se le da un valor final,
una validez absoluta, se puede plantear la imposibilidad de comparar, o
inconmensurabilidad, de los conceptos de masa de ambas mecánicas. Siendo un
medio, siempre es posible determinar cuál permite alcanzar el fin con más
facilidad y la precisión requerida.

Luego, no puede haber entidades simbólicas absolutas, finales, con
significado propio, cuando todo el simbolismo, completo, es sólo una
herramienta, un medio. Pero por esa misma razón, tampoco es necesario que
existan tales entidades simbólicas absolutas para que, desde éstas, el
significado se propague a las demás entidades simbólicas. El significado
viene de fuera del simbolismo, en concreto viene del problema que el
simbolismo ayuda a resolver.


\chapter La comunicación

Tanto el absolutismo como el escepticismo tienen un concepto erróneo del
significado.

Es cierto que, si se pretende, todo puede ser explicado. Pero, en tal caso,
las explicaciones son necesariamente circulares. La explicación es un medio,
no un fin, es una herramienta para resolver el problema de la supervivencia.

Es cierto que, al final, no se pueden distinguir los hechos empíricos de las
teorías hipotéticas, porque ambos son construcciones simbólicas. Pero no hay
inconmensurabilidad de teorías, ya que al final está el común problema de la
supervivencia.

Si los conceptos y teorías manejados por los distintos individuos fueran
inconmensurables, entonces no habría posibilidad de que se comunicaran. Las
palabras son convenciones para designar problemas, resoluciones y
soluciones, en forma de preguntas, procedimientos o comportamientos, de tal
suerte que si no hubiera un punto fijo de apoyo en donde anclar todo el
sistema simbólico de convenciones, éste quedaría limitado al individuo.

Así como la existencia no traumática de las paradojas es una prueba en
contra del absolutismo, la existencia de comunicación entre individuos es
una prueba en contra del escepticismo.


\chapter Lenguas muertas

Dado que los lenguajes simbólicos son meras convenciones, descifrar el
significado de un escrito en una lengua muerta y olvidada tendría que ser
imposible. No lo es porque, aunque no lo parezca, se tienen muchos datos de
partida. Por ejemplo, se sabe que fue escrito por personas, y las personas
nos enfrentamos, en último término, al mismo problema y con los mismos
medios básicos. Dicho de otro modo, los asuntos sobre los que interesa
escribir no son muchos y los modos que tenemos de emitir sonidos tampoco. Al
parecer, los primeros escritos tenían un propósito fiscal, a saber, recordar
quienes habían pagado sus tributos. Y los signos escritos siempre son
ideográficos o fonéticos.

No me es posible determinar si tendría sentido que desarrollase un lenguaje
simbólico una cultura que, al contrario que las nuestras, no tuviese su raiz
en el problema de la supervivencia. Pero, aun suponiendo que tuviera
sentido, sin ninguna comunalidad con tal cultura, nos sería imposible hallar
una base sobre la que construir los significados. No interpretaríamos su
lenguaje como un lenguaje. Aunque las piedras se quisieran comunicar con
nosotros, y las regularidades cristalográficas fueran un producto de la
sintaxis de su lengua, no podríamos entenderlas.


\chapter El lenguaje y la libertad

Un organismo no simbólico puede resolver problemas, incluso de varios modos.
Lo que no puede es expresar problemas y resoluciones. Al no poder expresar
problemas, no tiene una manera general de crear nuevos problemas o
resoluciones. Sí que puede tener mecanismos para crear ciertos tipos
concretos de problemas o de resoluciones, pero no de un modo general y sin
limitaciones.

Una de las maneras no simbólicas de resolver consiste en probar una serie de
comportamientos hasta dar con uno que solucione el problema. Es el
procedimiento de tanteo, o de prueba y error, que denominamos adaptación.

Otra manera no simbólica de resolver consiste en modelar lo externo, esto
es, interiorizar el comportamiento de lo externo de manera que, por ejemplo,
se pueda hacer el tanteo internamente, y así no tener que sufrir los errores
cometidos al tantear. A este método lo denominamos aprendizaje.

Incluso se pueden combinar varias maneras de resolver, como por ejemplo la
adaptación y el aprendizaje, para aplicar el aprendizaje cuando se tiene un
buen modelo del exterior y, en los otros casos, la adaptación.

Es seguro que algunas especies animales son capaces de aumentar los
contenidos de sus nombres comunes, esto es, pueden incorporar a su
repertorio de alimentos, o de predadores, nuevas especies. Lo que no pueden
hacer las especies no simbólicas es crear nuevos nombres comunes, o tipos de
cosas completamente nuevas. No pueden crear artefactos ni dibujar. Y no
pueden referirse a lo que no es, de modo que no pueden hacer planes ni tener
proyectos. No pueden diseñar herramientas.

Sólo un simbolismo permite referirse a lo que no es, y esto es necesario
para referirse a la libertad, a la incógnita, de un problema. Por esto es
acertado decir que el hombre, que es el único animal con lenguaje simbólico,
es el único animal libre. Por esto el lenguaje simbólico y la libertad
aparecen simultáneamente en el {\it homo sapiens}.


\chapter Shannon

\person{Shannon}\cite{Shannon1948} explica, en el segundo párrafo del
artículo que funda la teoría matemática de la comunicación, el alcance de la
misma. Éstas son sus palabras:

\beginquotation
El problema fundamental de la comunicación es el de reproducir en un punto,
o bien exactamente, o bien aproximadamente, un mensaje seleccionado en otro
punto. Frecuentemente los mensajes tienen {\em significado}; esto es, se
refieren a ciertas entidades físicas o conceptuales, o están relacionados en
base a cierto sistema con ellas. Estos aspectos semánticos de la
comunicación son irrelevantes al problema ingenieril. Lo que importa es que
el mensaje es uno {\em seleccionado de entre un conjunto} de mensajes
posibles. El sistema debe ser diseñado para operar con cualquier posible
selección, y no únicamente con aquélla efectivamente elegida, ya que esto es
desconocido en el momento del diseño.
\endquotation

La simplificación propuesta por \person{Shannon}, que consiste en evitar los
aspectos semánticos, se mostró enormemente fructífera porque permitió el
análisis matemático del problema. Y, sin embargo, sólo si el mensaje tiene
algún significado interesa su transmisión. Por esto confunde el adverbio que
\person{Shannon} añade cuando escribe que ``frecuentemente los mensajes
tienen {\em significado}'', al sugerir que en ocasiones tiene sentido, o
comunica algo, enviar un mensaje sin significado.

Incluso desde un punto de vista completamente ingenieril tiene interés el
aspecto semántico de la comunicación, porque la información semántica puede
utilizarse para resolver las ambigüedades sintácticas del mensaje. Dicho de
otro modo, un sistema de comunicación capaz de utilizar la información
semántica será más fiable que otro equivalente que no la utilice.

La ironía, que consiste en dar a entender lo contrario de lo que se dice, es
un buen ejemplo del valor de la información semántica del mensaje. En este
caso, el sistema de comunicación semántico, al recibir un mensaje irónico,
deberá ser capaz de deducir su verdadero significado, y esto es lo que
habitualmente hacen las personas.

Luego la decisión de \person{Shannon} de desdeñar la semántica no se pudo
deber a su ``irrelevancia''.  Se debió, por un lado, a que el modelo
simplificado permite su análisis matemático y, sobre todo, a que
\person{Shannon} no disponía de ninguna teoría semántica que pudiera ser
formalizada matemáticamente.


\vfil
\MTbeginfigure(120,80);
 \MT: pickup thick_pen;
 \MT: z1o = (0,1/3h); z1d = (w,1/3h);
 \MT: draw z1o .. z1d;
 \MT: z2o = (0,2/3h); z2d = (w,2/3h);
 \MT: draw z2o .. z2d;
 \MT: pickup thin_pen;
 \MT: for i := 0 upto 24:
 \MT:  x[i+11] = i*w/24;
 \MT:  draw (w/2,0){up} .. (x[i+11],h/2) .. {up}(w/2,h);
 \MT: endfor
\MTendfigure"Las líneas horizontales\cr son rectas paralelas"%
             Se trata de la ilusión de la dirección de \person{Wundt},
             adaptada de \person{Resnikoff};
\vfil\break


\chapter ¿Resolución o lenguaje?

Varias son las utilidades del simbolismo. El simbolismo permite tanto la
resolución de problemas y la elaboración de planes como el lenguaje
simbólico con sintaxis. El simbolismo proporciona tanto libre albedrío y
voluntad como consciencia de uno mismo y de lo otro.

Se puede explicar que una cualquiera de ellas supone una ventaja evolutiva y
que las otras no son más que felices consecuencias. Pero esto sería
engañoso, porque no es posible tener una de ellas sin tener las otras.

¿Resolución o lenguaje? Ambos.

¿Libertad o consciencia? Los cuatro.

Que estas características, que son las que tenemos por más humanas, sean
diferentes manifestaciones de un mismo fenómeno, a saber, la simbolización,
permite explicar la aparición del {\it homo sapiens\/} como el resultado de
un único salto evolutivo.
$$\hbox{\rm Simbolismo}
 \left\lbrace\vcenter{\hbox{Resolución de problemas}
         \hbox{Consciencia de uno mismo y de lo demás}
         \hbox{Lenguaje simbólico con sintaxis}
         \hbox{Libre albedrío y voluntad}}\right.$$


\chapter Los límites del conocimiento

Hemos rechazado los dos extremos, que todo puede ser conocido y que nada
puede ser conocido. Pero, ¿pueden fijarse con más precisión los límites del
conocimiento?

El conocimiento, como todo el aparato simbólico, es una herramienta que
sirve para la resolución del problema de la supervivencia y todos los demás
problemas, que se derivan, en último término, de aquél. Las herramientas
sólo deben ser usadas para realizar aquellas tareas para las que son útiles.
Y, sobre todo, una vez concluida la tarea para la cual la herramienta ayuda,
debe guardarse la herramienta, porque, a partir de entonces, sólo estorba.

Luego el conocimiento es útil hasta que se llega al problema de la
supervivencia, pero la causa de la vida queda fuera del alcance del
conocimiento. No tiene sentido preguntarse por la causa de la vida o, en
todo caso, la explicación de la causa de la vida no puede contener
conocimiento. Pueden conocerse, sin embargo, las causas de muchos otros
sucesos. Como ya hemos visto, puede conocerse la causa por la cual la calle
está mojada. Pero, en cualquier caso, nunca se conocen las causas últimas,
sino aquéllas que nos permiten solucionar los problemas que nos encontramos.


\chapter El gran plan

Partimos de un hecho primero, que no explicamos, la vida. Si nos interesa el
aspecto epistemológico de la vida, entonces la vida debe entenderse como un
problema aparente. La vida dispone de mecanismos de reproducción sobre los
que opera un proceso de selección merced al cual la vida se diversifica y
adapta al medio, como explicó \person{Darwin}. Al diversificarse, cada
individuo vivo busca, ya por sí mismo ya estableciendo grupos, su propia
supervivencia o la de su estirpe, y de este modo se sostiene la vida toda.

Este proceso de evolución darwiniana ha mostrado una notable capacidad para
adaptarse a condiciones diversas y cambiantes. Se puede mostrar que,
supuesto que la vida es un problema aparente y que se cumplen ciertas
condiciones, que evidentemente se han cumplido, los individuos capaces de
varios comportamientos, que llamamos adaptadores, tienen una capacidad para
sobrevivir y reproducirse mayor que aquellos individuos que sólo ejecutan
mecánicamente un único comportamiento. También se puede mostrar, que dadas
otras condiciones también verificadas, los aprendices, individuos capaces de
representarse el comportamiento de su entorno y, a partir de tal modelo,
capaces de predecir internamente el resultado de sus acciones sobre el
medio, tienen una ventaja evolutiva sobre los adaptadores y, en
consecuencia, pueden sobrevivir a costa, o en lugar, de éstos. Y supuestas
otras condiciones, también encontradas, los conocedores, que son capaces de
varios modos de resolver, aventajan a los mecanismos, a los adaptadores y a
los aprendices, que son capaces de un único modo de resolver.

Por fin, y en otras condiciones igualmente cumplidas, como atestigua la mera
existencia de este libro, es ventajoso disponer de un simbolismo. Sólo
simbólicamente se pueden representar los problemas, las resoluciones y las
soluciones, de modo que este individuo simbólico puede tener una visión del
gran plan, que es la vida, que es un problema. Como él mismo es parte de
este gran plan, es necesario disponer de una lógica simbólica para ser un
sujeto, para tener un yo.

\chapter Gaia

Considerar la vida toda como un único organismo, denominado %llamado
Gaia\cite{Lovelock1979}, coincide en aspectos fundamentales con los
argumentos aquí presentados.

\chapter Saber es un medio, el fin es vivir

Saber es solamente un medio, el fin es vivir.


\chapter Yo quiero vivir, pero moriré

Yo quiero vivir, pero moriré. La frase anterior resume nuestra situación.
Enuncia el problema de la supervivencia y declara que no tiene solución.

Pero hay otros puntos de vista.

\chapter Negar la muerte

Yo quiero vivir, y no moriré. Ésta es la esperanza a la que muchos se
aferran. Las variantes son muchas. Lo más arduo de esta posición es pasar
por alto la observación de que todos los seres vivos terminan por morir.

El credo dualista divide a la persona en cuerpo y alma. El cuerpo muere,
pero no el alma. El alma queda fuera del alcance de los sentidos, lo que
explica que solamente se pueda verificar la muerte del cuerpo. Sobre lo que
ocurre al alma una vez que el cuerpo muere, hay dos posibilidades: que el
alma pase a otro cuerpo, o que no pase.

La creencia dualista puede tener su fundamento en la diferencia que existe
entre la realidad y el yo. El yo es la solución del problema del sujeto,
está en donde está la libertad del problema. La realidad, por otro lado, es
la condición del problema del sujeto. El cuerpo del sujeto no es su yo, sino
un condicionante que el sujeto debe tener en cuenta para procurarse su
supervivencia. Así que el cuerpo es parte de la realidad, pero no el yo.
Luego el yo cumple varios de los requisitos exigidos al alma.

\chapter Negar el deseo

Yo no quiero vivir, y moriré. Es la postura del pesimista, más vale no
desear nada y de este modo no es posible la decepción. La dificultad de esta
posición es que niega la propia vida, por lo que parece difícil que quien no
desea vivir pueda evitar el suicidio.

\goodpage

\chapter Negar la libertad

No hay libertad, de manera que aunque parece que yo quiero vivir, esto es
una ilusión. No puedo querer, no puede haber deseo, ya que no hay elección
que hacer. El fatalismo dice que todo está escrito. Nos parece que tenemos
libertad para hacer, pero no es así. Sus dos variantes más extendidas son el
fatalismo religioso y el materialismo.

Para el fatalismo religioso Dios es todopoderoso. Si hubiera un ser libre
distinto de Dios, entonces este ser libre podría actuar en contra de la
voluntad divina, pero esto significaría que Dios no es todopoderoso. Luego,
el único ser libre es Dios todopoderoso. A este respecto, la teología
musulmana es coherente y la cristiana no.

El materialismo supone la existencia de sustancias elementales cuyas
relaciones están regidas por leyes naturales. Todo cuanto existe resulta de
la combinación de estos elementos, que obedece a las leyes de la naturaleza.
De modo que para los materialistas son las leyes naturales las que
determinan todo cuanto sucede, sin dejar espacio alguno a la libertad, que
simplemente es una ilusión.

La otra gran dificultad del fatalismo, siendo la primera la de tener que
negar la evidencia de la libertad, es su ética. El fatalismo reduce a la
persona a una marioneta o a un mecanismo. En ninguno de los dos casos tiene
la persona la posibilidad de actuar de otro modo, de manera que tampoco es
posible prescribir qué debe hacer. No hay posibilidad de ser virtuoso, ni de
ser perverso, simplemente no hay posibilidad.


\vfil
\MTbeginfigure(160,120);
 \MT: pickup thick_pen;
 \MT: save u, v, pm, alpha, betha;
 \MT: u = w/4; v = 6pt; pm = 4/5*(u-v); alpha = -2; betha = 282;
 \MT: z1 = (0,h/3);
 \MT: z11 = z1 + u*(right rotated (alpha+0));
 \MT: z12 = z1 + 2u*(right rotated (alpha+80));
 \MT: z13 = 1.03*(z11 + z12 - z1);
 \MT: draw z1 -- z11 -- z13 -- z12 -- cycle;
 \MT: z1d = z1 + (0,-v); z11d = z11 + (0,-v);
 \MT: draw z1 -- z1d -- z11d -- z11;
 \MT: x1o = x1 + v/2; z1o = whatever[z1d,z11d];
 \MT: z1b = z1o + (0,-pm); z1r = z1b + (v,0);
 \MT: x1x = x1r; z1x = whatever[z1d,z11d];
 \MT: draw z1o -- z1b -- z1r -- z1x -- cycle;
 \MT: x11o = x11d - v/2; z11o = whatever[z11d,z1d];
 \MT: z11b = z11o + (0,-pm); z11r = z11b + (-v,0);
 \MT: x11x = x11r; z11x = whatever[z1d,z11d];
 \MT: draw z11o -- z11b -- z11r -- z11x -- cycle;
 \MT: x13o = x13 - v/4; z13o = whatever[z13,z11];
 \MT: z13b = z13o + (0,-4pm/5); z13r = z13b + (-v,0);
 \MT: y13x = y13b; z13x = whatever[z13,z11];
 \MT: draw z13o -- z13b -- z13x;
 \MT: z23 = (w,h/2);
 \MT: z23 = 1.03*(z21 + z22 - z2);
 \MT: z22 = z2 + 2u*(right rotated (betha+80));
 \MT: z21 = z2 + u*(right rotated (betha+0));
 \MT: draw z2 -- z21 -- z23 -- z22 -- cycle;
 \MT: z21d = z21 + (0,-v); z23d = z23 + (0,-v);
 \MT: draw z21 -- z21d -- z23d -- z23;
 \MT: x2o = x2 + v/4; z2o = whatever[z2,z21];
 \MT: z2b = z2o + (0,-4pm/5);
 \MT: y2x = y2b; z2x = whatever[z2,z21];
 \MT: draw z2o -- z2b -- z2x;
 \MT: x21o = x21d + v/2; z21o = whatever[z21d,z23d];
 \MT: z21b = z21o + (0,-pm); z21r = z21b + (v,0);
 \MT: x21x = x21r; z21x = whatever[z21d,z23d];
 \MT: draw z21o -- z21b -- z21r -- z21x -- cycle;
 \MT: x23o = x23d - v/2; z23o = whatever[z23d,z21d];
 \MT: z23b = z23o + (0,-pm); z23r = z23b + (-v,0);
 \MT: x23x = x23r; z23x = whatever[z23d,z21d];
 \MT: draw z23o -- z23b -- z23r -- z23x -- cycle;
\MTendfigure"Los paralelogramos\cr son iguales"%
             Figura adaptada de \person{Gazzaniga};
\vfil
\break


\chapter Minsky

\person{Minsky}\cite{Minsky1985} piensa que ``las mentes son simplemente lo
que los cerebros hacen'', con lo que resuelve a la manera materialista el
problema del dualismo cartesiano entre el cuerpo y la mente. El lector que
así lo prefiera puede leer alma donde está escrito mente.

Al no conocer todas las causas adoptamos otras maneras de hablar. Así, sobre
todo al referirnos a personas, usamos conceptos como voluntad, libertad o
propósito que, opina \person{Minsky}, sólo son maneras de hablar útiles
porque ignoramos las causas. Y concluye:
\beginquotation
No importa que el mundo físico no deje espacio alguno a la libre voluntad:
ese concepto es esencial en nuestros modelos de lo mental. En él está basada
demasiada de nuestra psicología como para prescindir alguna vez de él.
Estamos virtualmente forzados a mantener esa creencia, aun cuando sabemos
que es falsa [\dots].
\endquotation

El materialismo, que con \person{Monod}\cite{Monod1970} ha retomado la
antigua máxima de \person{Demócrito} ``todo es azar y necesidad'', confunde
la explicación con el fenómeno. El azar y la necesidad son conceptos
simbólicos muy útiles, pero sólo eso, útiles. Y la libertad es también un
concepto útil.

En la resolución del problema aparente hay una parte que consiste en modelar
lo externo, y en esta modelación lo que interesa es encontrar un mecanismo,
esto es, un modelo sin ningún grado de libertad, que se comporte exactamente
como el exterior. De este modo se puede utilizar el modelo para prever con
exactitud lo que pasaría en cada caso. Así es como procuramos modelar el
mundo físico, y por esto el conocimiento científico, a nivel social, y la
realidad, a nivel personal, excluyen la libertad.

Pero esta modelación sólo sirve para traer lo externo al interior, para
fijar el cambio, para convertir el fenómeno en símbolo. No puede olvidarse
que, por importante que sea, la modelación no es un fin en sí misma, sino
que tiene el propósito de ayudar en la resolución del problema aparente. Y
en el problema aparente, como en todo problema, hay libertad porque sin
libertad no hay problema. En el caso del problema aparente de la
supervivencia tenemos libertad para actuar y la condición de sobrevivir.

\goodpage

\chapter Yo soy libre

Un simbolismo es un sistema extensible de convenciones que sirve para
resolver problemas, ya que permite la expresión de problemas, resoluciones y
soluciones. Para resolver simbólicamente es preciso, primero, fijar
sintácticamente el problema para que, una vez fijo, pueda procesarse
algorítmicamente y de este modo obtener, finalmente, una solución simbólica
que tenga significado, esto es, que se refiera a un comportamiento que
solucione efectivamente el problema.

El significado de los símbolos no es interno al simbolismo, que por sí mismo
está vacío de significados, sino que viene dado por el papel que interpretan
estos símbolos en la resolución del problema. Así, la incógnita ha de ser un
símbolo libre de significado, un puro artefacto sintáctico, mientras que la
solución ha de tener significado.

Siendo la vida un problema en continua resolución, solamente un individuo
que disponga de una lógica simbólica puede representarse la vida como
problema y a sí mismo como resolutor, o sea, solamente él puede ser un
sujeto. Como este sujeto es un resolutor, su propósito es solucionar el
problema vital, y se ve a sí mismo como la solución. En definitiva, el yo es
la solución del problema del sujeto, y si la solución del problema del
sujeto fuera $\Omega$, entonces podríamos usar $\Omega$ para referirnos al
yo del sujeto. Pero como el problema del sujeto no alcanza nunca una
solución, ésta no puede ser expresada directamente. El problema sí. Luego la
única manera de referirse al yo, consiste en referirse indirectamente a él
como la solución del problema del sujeto. La solución de un problema no
solucionado es la incógnita, la libertad del problema, de donde se concluye
que el yo es libertad, aunque eso sí, condicionada.

Además, ¿hay algo más real que mi sensación de ser libre?


\endinput

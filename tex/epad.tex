% EPAD.TEX (RMCG19970304)

\newpage

\title2 La dinámica

\title3 La ecuación dinámica

Si denotamos
 $|S_s \!/ 2^{\pr S}$, con
 $|S_s, \pr S \in \N$,
la probabilidad de que el estado actual de $|A$ sea $s$, de modo que
 $s \through \no S$, entonces
podemos representar el vector de estado de |A a la manera probabilística:
 $|S = \left< \no S, \pr S; (|S_s) \right>$.

Si denotamos
 $|J_{\imath} \!/ 2^{\pr{\!J}}$, con
 $|J_{\imath}, \pr{\!J} \in \N$,
la probabilidad de que la entrada actual a $|A$ sea $\imath$, de modo que
 $\imath \through \no I$, entonces
podemos representar el vector de entrada a |A a la manera probabilística:
 $|J = \left< \no I, \pr{\!J}; (|J_{\imath}) \right>$.

Si las variables del vector de estado actual son probabilísticamente
independientes de las variables del vector de entrada, entonces
 $$ { |S_s \over 2^{\pr S} } \cdot
    { |J_{\imath} \over 2^{\pr{\!J}} }  =
    { |P_{\!s\imath} \over 2^{\pr S+\pr{\!J}} } $$
es la probabilidad de que el estado actual de $|A$ sea $s$ y que la entrada
actual a $|A$ sea $\imath$. Linealizando la matriz $(|P_{\!s\imath})$, para
lo que hacemos $p = s.2^{\no I}+\imath$, obtenemos el vector de
estado+entrada
 $$|P = \left< \no S+\no I, \pr S+\pr{\!J}; (|P_p) \right>.$$
Nótese que, usando la descripción automática de vectores, podemos
resumir los cálculos anteriores así:
 $$|P = \prl |S |J.$$

Puesto que las probabilidades que definen al autómata |A,
 $|A_{pq} \!/ 2^{\no P}$,
son probabilidades de los valores de estado siguiente y de salida actual
condicionadas a los valores de $|P$, dadas estas probabilidades podemos
calcular aquéllas.  En concreto,
 $$\sum_p { |P_p \over 2^{\pr S+\pr{\!J}} } \cdot
 { |A_{pq} \over 2^{\no P} } =
 { |Q_q \over 2^{\pr S+\pr{\!J}+\no P} } $$
es la probabilidad de que el vector estado-siguiente+salida $|Q$ tome el
valor correspondiente al índice $q = n.2^{\no O}+o$.  Es decir,
 $|Q_q = \sum_p |P_p \cdot |A_{pq}$ y
 $$|Q = \bigl< \no S+\no O, \pr S+\pr{\!J}+\no P;
   (\textstyle{\sum_p} |P_p \cdot |A_{pq}) \bigr> .$$
Estos cálculos pueden resumirse usando, de nuevo, la descripción
automática de vectores, para obtener la \definition{ecuación
dinámica}:
 $$|Q = \srl |P \FA = \srl \prl |S |J \FA$$
donde {\FA} es la función característica de |A (véase la
\S\refsc{La función característica}).

A partir de este resultado es también posible describir separada y
probabilísticamente los vectores de estado siguiente $|N$ y de salida
$|O$.  Usando $|Q_{no}$,
 $|Q_{no} = \sum_s \sum_\imath |P_{\!s\imath} \cdot \ALFA$,
que es la forma matricial de $|Q_q$, queda:
 $$\eqalign{
  |N &= \left< \no S, \pr S+\pr{\!J}+\no P;
   (\textstyle{\sum_o} |Q_{no}) \right> \cr
  |O &= \left< \no O, \pr S+\pr{\!J}+\no P;
   (\textstyle{\sum_n} |Q_{no}) \right> .\cr}$$
Es decir, hemos proyectado la matriz $(|Q_{no})$ en dos partes.  Debe
tenerse presente que, en general, las variables del vector de estado
siguiente no serán probabilísticamente independientes de las variables
del vector de salida.

\title4 Ejemplo

Sea el autómata |H del ejemplo que tenía un 1 subrayado (en la página
{\refpg{channel}}), y sea el string de estado+entrada
 $|P^{|H} = \left< 2, 0; (0, 1, 0, 0) \right>$,
que significa que es seguro que el autómata
está en el estado $1$ y que la variable de entrada vale $0$.  Calculamos
$|Q^{|H}$ de acuerdo a la ecuación dinámica y obtenemos
 $\left< 2, 1; (1, 0, 0, 1) \right>$.
Esto significa que es tan probable que el estado siguiente sea $1$ y la
salida actual $1$ como que el estado siguiente sea $0$ y la salida actual
$0$.  Nótese que la probabilidad de que el estado siguiente sea $1$ es $1/2$
y que la probabilidad de que la salida valga $0$ es también $1/2$, pero que
es imposible que el estado siguiente sea diferente de la salida.


\title3 La dinámica determinada

La ecuación dinámica es determinística, ya que el vector $|Q$ queda
completamente determinado por $|S$, $|J$ y \FA, aunque describe un proceso
probabilístico. El autómata, aun siendo probabilístico, usa valores
determinados de las variables; concretamente y por ser binario, las
variables toman, en todo instante, el valor 1 o el valor 0.

Mientras que al partir un vector descrito probabilísticamente se
perderá información si las variables que quedan en las distintas partes
no son probabilísticamente independientes, se puede partir el string que
toma como valor un vector sin pérdida alguna de información.

Sea $S = [b_{\ell-1}\,b_{\ell-2}\, \ldots\, b_2\,b_1\,b_0]$ un string,
$S\in\B^\ell$, cualquiera. Si notamos
 $$\abovedisplayshortskip=-3pt\abovedisplayskip=-3pt\relax
 \displaylines{
  S_{m\times} = [b_{\ell-1}\,b_{\ell-2}\, \ldots\, b_{\ell-m}] =
    \prli_{i=\ell-1}^{\ell-m} \K_{b_i} \cr
  S_{\times n} = [b_{n-1}\, \ldots\, b_2\,b_1\,b_0] =
    \prli_{i=n-1}^0 \K_{b_i} \cr}$$
los strings cabeza $S_{m\times}$, de longitud $m \leq \ell$, y cola
$S_{\times n}$, de longitud $n \leq \ell$, en que queda partido $S$, de
longitud $\ell = m + n$, entonces es inmediato que:
 $$S = \prl S_{m\times} \, S_{\times n} \quad (\ell = m + n),$$
lo que demuestra que no se pierde información al partir un string.

Luego si $\opd S$ es el string de estado actual del autómata |A e $\opd I$
es el string de entrada, notamos $\opd P = \prl \opd S \opd I$, y la
ecuación dinámica queda:
 $$|Q = \bigl< \no S+\no O, \no P; (|A_{Pq}) \bigr>
   \quad\hbox{o bien}\quad
  |Q = \srl \opd P \FA = \srl \prl \opd S \opd I \FA $$
donde |Q es un vector probabilístico si
 $\no P > 0$ y un string si
 $\no P = 0$.  En cualquier caso, escribiremos
 $Q \leapsfrom |Q$
si el vector |Q \definition{sabemos que toma como valor} el string $Q$, y
entonces el string de estado siguiente $N$ y el string de salida $O$ valen:
 $$N = Q_{\no S\times} \qquad O = Q_{\times\no O} .$$

\title3 El tiempo

Ahora describiremos el paso de un instante al siguiente.  Recordemos que
denominamos string al valor que toma un vector y stream a la secuencia de
strings ordenada creciendo desde el instante inicial.

Utilizaremos números naturales para referirnos a los instantes del
tiempo.  El instante inicial utilizará el número $0$, y cada instante
utilizará el número siguiente al utilizado por el instante anterior.
Denominaremos tiempo, \T, al conjunto de todos los instantes,
 $\T = \{ 0, 1, 2, 3, \ldots \}$.
Es decir, que al vector |V en el instante $t \in \T$ lo denotaremos $|V(t)$
y a su valor, si lo conocemos, $V(t)$. Si conocemos su valor en el instante
siguiente, lo anotaremos $V(t+1) \leapsfrom |V(t+1)$.

Ya podemos establecer la \definition{ecuación temporal} de los
autómatas, a saber, que el estado siguiente del instante actual, $|N(t)$,
es el estado actual del instante siguiente, $|S(t+1)$. Es decir:
 $$ |N(t) = |S(t+1) .$$

\title3 La función de transferencia

Si aplicamos el stream de entrada $I(0), I(1), \ldots, I(\tau)$, de duración
$\delta$, $\delta=\tau+1$, al autómata |A en el estado inicial $S_0$,
$|S(0)= S_0 \in \B^{\no S}$, obtendremos en el instante final $\tau$ un
string de salida del que, {\it a priori}, sólo podemos conocer su vector de
probabilidad $|O(\tau)$. Para calcular $|O(\tau)$ hemos de aplicar
reiteradamente la ecuación dinámica. Notando con sombrero las
probabilidades, $|A_{pq} \!/ 2^{\no P} = \P|A_{pq}$, y teniendo en cuenta
que, por la ecuación temporal, $n_t$ y $s_{t+1}$ son dos sinónimos del mismo
índice, queda:
 $$\displaylines{
  \P|O_{o_\tau}\!\bigl(S_0; I(0), I(1), \ldots, I(\tau)\bigr)
  = \hfill\cr =
   \sum_{s_\tau n_\tau} \P|A_{s_\tau I(\tau) n_\tau o_\tau} \cdot
   \sum_{\scriptstyle s_{\tau-1} \atop \scriptstyle o_{\tau-1}}
    \P|A_{s_{\tau-1} I(\tau-1) n_{\tau-1} o_{\tau-1}}
  \cdots \cr \noalign{\vskip-3pt} \hfill \cdots
   \sum_{s_1 o_1} \P|A_{s_1 I(1) n_1 o_1} \cdot
   \sum_{o_0} \P|A_{S_0 I(0) n_0 o_0} .}$$

Llamamos \definition{función de transferencia} del autómata |A desde su
estado $S_0$, $S_0 \in \B^{\no S}$, y la notamos $Z^{|A}_{S_0}$, a la
función que hace corresponder, cada uno de los posibles streams de entrada a
|A, con el vector probabilístico que describe la salida que se obtendría en
el instante final si se aplicara dicho stream de entrada al autómata |A
desde el estado $S_0$. Es decir:
 $$ Z^{|A}_{S_0}\!\bigl(I(0), I(1), \ldots, I(\tau)\bigr) = |O(\tau) ,$$
que se calcula como hemos visto arriba.

\goodpage

\title4 Extensión probabilística

La extensión probabilística de la función de transferencia, notada
$|Z^{|A}_{S_0}$, hace la misma correspondencia que $Z^{|A}_{S_0}$ cuando,
en vez de conocer el stream de entrada, tenemos su descripción
probabilística
 $|J(0), |J(1), \ldots, |J(\tau)$.
O sea,
 $$\displaylines{
  \P|O_{o_\tau}\!\bigl(S_0; |J(0), |J(1), \ldots, |J(\tau)\bigr) =
 \hfill\cr\hfill =
   \sum_{\imath_\tau s_\tau n_\tau}
    \P|J(\tau)_{\imath_\tau}
    \P|A_{s_\tau \imath_\tau n_\tau o_\tau} \cdot
   \sum_{\scriptstyle\imath_{\tau-1} \atop
         {\scriptstyle s_{\tau-1} \atop
          \scriptstyle o_{\tau-1}}}
    \P|J(\tau-1)_{\imath_{\tau-1}}
    \P|A_{s_{\tau-1} \imath_{\tau-1} n_{\tau-1} o_{\tau-1}}
 \cdots \cr \hfill \cdots
   \sum_{\imath_1 s_1 o_1}
    \P|J(1)_{\imath_1} \P|A_{s_1 \imath_1 n_1 o_1} \cdot
   \sum_{\imath_0 o_0}
    \P|J(0)_{\imath_0} \P|A_{S_0 \imath_0 n_0 o_0} .\cr}$$

\title4 Teorema

Se tiene que:
 $$Z^{|A}_{S_0} = Z^{|A'}_{S'_0} \implies
  |Z^{|A}_{S_0} = |Z^{|A'}_{S'_0} .$$

\title4 Demostración

 $$\displaylines{
   Z^{|A}_{S_0} = Z^{|A'}_{S'_0} \implies
     \forall I(0), I(1), \ldots, I(\tau) :
   \hfill \cr
   \P|O^{|A}_{o_\tau}\!\bigl(S_0; I(0), I(1), \ldots, I(\tau)\bigr) =
   \P|O^{|A'}_{o_\tau}\!\bigl(S'_0; I(0), I(1),\ldots, I(\tau)\bigr) .\cr}$$
Y, por lo tanto,
 $$\displaylines{
  \P|O^{|A}_{o_\tau}\!\bigl(S_0; |J(0), |J(1), \ldots, |J(\tau)\bigr) =
 \hfill\cr\qquad\qquad =
   \sum_{\imath_\tau \imath_{\tau-1} \ldots \imath_1 \imath_0}
      \P|J(\tau)_{\imath_\tau}
      \P|J(\tau-1)_{\imath_{\tau-1}} \ldots
      \P|J(1)_{\imath_1} \P|J(0)_{\imath_0} \cdot
 \hfill\cr\hfill \cdot
   \sum_{s_\tau n_\tau}
      \P|A_{s_\tau \imath_\tau n_\tau o_\tau} \cdot
   \sum_{\scriptstyle s_{\tau-1} \atop
         \scriptstyle o_{\tau-1}}
      \P|A_{s_{\tau-1} \imath_{\tau-1} n_{\tau-1} o_{\tau-1}}
 \cdots \qquad\qquad \cr \hfill \cdots
   \sum_{s_1 o_1} \P|A_{s_1 \imath_1 n_1 o_1} \cdot
   \sum_{o_0} \P|A_{S_0 \imath_0 n_0 o_0} =
 \cr\qquad\qquad =
   \sum_{\imath_\tau \imath_{\tau-1} \ldots \imath_1 \imath_0}
      \P|J(\tau)_{\imath_\tau}
      \P|J(\tau-1)_{\imath_{\tau-1}} \ldots
      \P|J(1)_{\imath_1} \P|J(0)_{\imath_0} \cdot
 \hfill\cr\hfill \cdot
   \sum_{s'_\tau n'_\tau}
      \P|A'_{s'_\tau \imath_\tau n'_\tau o_\tau} \cdot
   \sum_{\scriptstyle s'_{\tau-1} \atop
         \scriptstyle o_{\tau-1}}
      \P|A'_{s'_{\tau-1} \imath_{\tau-1} n'_{\tau-1} o_{\tau-1}}
 \cdots \qquad\qquad \cr \hfill \cdots
   \sum_{s'_1 o_1} \P|A'_{s'_1 \imath_1 n'_1 o_1} \cdot
   \sum_{o_0} \P|A'_{S'_0 \imath_0 n'_0 o_0} =
 \cr \qquad\qquad =
  \P|O^{|A'}_{o_\tau}\!\bigl(S'_0; |J(0), |J(1), \ldots, |J(\tau)\bigr)
   . \hfill\cr}$$
Con lo que se demuestra que:
 $Z^{|A}_{S_0} = Z^{|A'}_{S'_0} \implies |Z^{|A}_{S_0} = |Z^{|A'}_{S'_0}$.

\goodpage

\title4 Extensión temporal

La extensión en el tiempo de la función de transferencia, notada
$\vec{Z}^{|A}_{S_0}$, es la función que devuelve, en vez del valor final
$|O(\tau)$ como $Z^{|A}_{S_0}$, toda la secuencia
 $|O(0), |O(1), \ldots, |O(\tau)$.

\labeled\title3 El comportamiento

Las \definition{variables externas} son las de entrada al autómata y las
de salida del autómata.  Al par de streams desarrollado por estos dos
vectores lo denominamos \definition{apariencia del autómata}.

\title4 Teorema

Si  $Z^{|A}_{S_0} = Z^{|A'}_{S'_0}$ entonces no se puede distinguir, por su
apariencia, a |A de $|A'$.

\title4 Demostración

Si  $Z^{|A}_{S_0} = Z^{|A'}_{S'_0}$ entonces, para
cualquier stream de entrada $I(0), I(1), \ldots I(\tau)$, se tiene que:
 $$\displaylines{
   |O^{|A}(0) = Z^{|A}_{S_0}\!\bigl(I(0)\bigr) =
     Z^{|A'}_{S'_0}\!\bigl(I(0)\bigr) = |O^{|A'}\!(0) \cr
   |O^{|A}(1) = Z^{|A}_{S_0}\!\bigl(I(0), I(1)\bigr) =
     Z^{|A'}_{S'_0}\!\bigl(I(0), I(1)\bigr) = |O^{|A'}\!(1) \cr
   \!\vdots \cr
   |O^{|A}(\tau) = Z^{|A}_{S_0}\!\bigl(I(0), I(1),  \ldots, I(\tau)\bigr) =
   Z^{|A'}_{S'_0}\!\bigl(I(0), I(1), \ldots, I(\tau)\bigr) =
   |O^{|A'}\!(\tau)
  \cr}$$
lo que significa que ni siquiera probabilísticamente es posible distinguir
por su apariencia a |A de $|A'$.

\title4 Corolario

Eso también prueba, en relación a la extensión en el
tiempo de la función de transferencia, $\vec{Z}^{|A}_{S_0}$, que:
 $$Z^{|A}_{S_0} = Z^{|A'}_{S'_0} \implies
   \vec{Z}^{|A}_{S_0} = \vec{Z}^{|A'}_{S'_0} .$$

\title4 Definiciones

De aquí se deriva la definición de \definition{indistinguibilidad}:
decimos que dos autómatas, |A y $|A'$, son indistinguibles, y lo notamos
 $|A \nodis |A'$, si y sólo si, por definición,
 $\inpA = \inp{\!|A'}$, y $\outA = \out{\!|A'}$, y se cumple que:
 $$(\exists S_0 \in \B^{\stA}) \land
   (\exists S'_0 \in \B^{\st{\!\!A\!'}}) : \;
   Z^{|A}_{S_0} = Z^{|A'}_{S'_0} .$$
También decimos, en este caso, que $S_0$ y $S'_0$ son
\definition{estados equivalentes}.

La indistinguibilidad depende del estado inicial, pero si cada estado de |A
tuviera algún estado equivalente en $|A'$, y {\it vice versa}, entonces
|A y $|A'$ serían indistinguibles independientemente del estado inicial.

Esto nos sugiere la definición de \definition{equivalencia}:  decimos que
dos autómatas, |A y $|A'$, son equivalentes, y lo notamos
 $|A \equiv |A'$, si y sólo si, por definición,
 $\inpA = \inp{\!A'}$, y $\outA = \out{\!A'}$, y se cumple que:
 $$ \{ Z^{|A}_{S_0} : S_0 \in \B^{\stA} \} =
    \{ Z^{|A'}_{S'_0} : S'_0 \in \B^{\st{\!\!A\!'}} \} .$$

La indistinguibilidad es más débil que la equivalencia, y ésta es
más débil que la igualdad:
 $$ |A = |A' \implies |A \equiv |A' \implies |A \nodis |A'. $$

La equivalencia de autómatas es una relación de equivalencia, y de
ahí su nombre:
 $$\eqalignrc{\forall |A, |B, |C:
  & |A \equiv |A \cr
  & |A \equiv |B \implies |B \equiv |A \cr
  & (|A \equiv |B) \land (|B \equiv |C) \implies |A \equiv |C .\cr}$$

Denominamos \definition{comportamiento} a cada elemento del conjunto
cociente definido por la equivalencia sobre el conjunto de los autómatas,
 $$\{ \hbox{Comportamiento} \} = \{ |A \} / \!\equiv.$$
Es decir, que si dos autómatas son equivalentes, $|B \equiv |C$, entonces
decimos que tienen el mismo comportamiento.

La indistinguibilidad no es una relación de equivalencia. Por ejemplo:
 $$ (\srl \prl \fdb \FORK \ID \XOR \nodis \ID) \land
    (\srl \prl \fdb \FORK \ID \XOR \nodis \NOT) \land
    (\ID \not\nodis \NOT) .$$
Si el estado inicial del autómata $\srl \prl \fdb \FORK \ID \XOR$ es $1$,
entonces se distingue del autómata $\ID$.

\title4 Ampliabilidad

\label{eqamp'}
Cuando interesa más el comportamiento que la igualdad, se pueden utilizar
las siguientes definiciones alternativas de ampliación:
 $$\displaylines{
  |B \mathrel{\tilde{\amp}} \!|A \iff \bigl(\inp B > \inpA\bigr) \land
    \bigl(\exists c: \Syntax(c,\inp B-\inpA,|B) \nodis |A\bigr) ,\cr
  |B \mathrel{\bar{\amp}} \!|A \iff \bigl(\inp B > \inpA\bigr) \land
    \bigl(\exists c: \Syntax(c,\inp B-\inpA,|B) \equiv |A\bigr) .\cr }$$


\title3 La dinámica compuesta

El cálculo de la dinámica de un autómata compuesto puede hacerse
calculando primero el autómata compuesto, utilizando para ello las
definiciones de las composiciones, y aplicando entonces la ecuación
dinámica, o puede hacerse aplicando la ecuación dinámica a cada parte
y componiendo los resultados como se propone a continuación.

Nótese que si {\em conocemos} los valores determinados $Q$ que toman los
vectores $|Q$ ($Q \leapsfrom |Q$), entonces pueden ser utilizados los
cálculos desacoplados, por partes, o paso a paso. La ecuación temporal toma,
en este caso, la forma: $N(t) = S(t+1)$.


\title4 Composición en serie

El nombre de la composición en serie tiene sentido porque
 $$Z^{\srl\!|B\!|C}_{\prl\! S_{|B} S_{|C}} =
   \vec{Z}^{|B}_{S_{|B}} \circ |Z^{|C}_{S_{|C}} $$
donde $\circ$ denota la composición de funciones,
 $[g \circ f](x) = f\bigl(g(x)\bigr)$, y |Z y
 $\vec{\opd{Z}}$ son las extensiones, probabilística y temporal,
de la función de transferencia~$\opd Z$.

\title4 Demostración

Notamos $|A=\srl|B|C$ y, por tratarse de strings,
 $$\displaylines{
   \dot{S}_0 = S^{|A}(0)_{\st B \times} \qquad
   \ddot{S}_0 = S^{|A}(0)_{\times \st C} .\cr}$$
Por la definición de la composición en serie,
 $I^{|B}(t) = I^{|A}(t)$, que notamos abreviadamente $I_t$.

La disposición de los índices se muestra en la figura, de la cual
también se deduce que:
 $$ \F_{\srl\!|B\!|C} = \srl \srl \srl
    \prl \op{Cross}_{\st B,\st C} \ID_{\inp B}
    \prl \ID_{\st C} \F_B
    \prl \op{Cross}_{\st C,\st B} \ID_{\inp C}
    \prl \ID_{\st B} \F_C .$$
% Composición en serie
\MTbeginfigure(150,60);
 \MT: pickup thin_pen;
 \MT: x1 = 1/2w; y1b = 0;
 \MT: rectangle(1)(w-40u,35v); % 1 is F_{+ B C}
 \MT: pickup thick_pen;
 \MT: x2l = x1l + 30u; y2b = y1b + 5v;
 \MT: rectangle(2)(20u,20v); % 2 is F_B
 \MT: x3r = x1r - 5u; y3b = y1b + 5v;
 \MT: rectangle(3)(20u,20v); % 3 is F_C
 \MT: z2lbl = z2; z3lbl = z3;
 \MTlabel(2lbl)"\F_B";
 \MTlabel(3lbl)"\F_C";
 \MT: pickup med_pen;
 \MT: for i := 11 upto 20:
 \MT:  y[i]b = y1b + 10v;
 \MT:  y[i]m = y1b + 20v;
 \MT:  y[i]t = y1b + 30v;
 \MT: endfor
 \MT: x11b = x11m = x11t = x1l;
 \MT: x12b = x12m = x12t = 1/6[x1l,x2l];
 \MT: x13b = x13m = x13t = 5/6[x1l,x2l];
 \MT: x14b = x14m = x14t = x2l;
 \MT: x15b = x15m = x15t = x2r;
 \MT: x16b = x16m = x16t = 1/6[x2r,x3l];
 \MT: x17b = x17m = x17t = 5/6[x2r,x3l];
 \MT: x18b = x18m = x18t = x3l;
 \MT: x19b = x19m = x19t = x3r;
 \MT: x20b = x20m = x20t = x1r;
 \MT: draw z11b .. z12b; draw z11m .. z12m; draw z11t .. z12t;
 \MT: draw z12b .. z13b;
 \MT: draw z12m{1,0} .. z13t{1,0};
 \MT: draw z12t{1,0} .. z13m{1,0};
 \MT: arrow(13b,14b); arrow(13m,14m);
 \MT: draw z13t .. z16t;
 \MT: draw z15b .. z16b; draw z15m .. z16m;
 \MT: draw z16b .. z17b;
 \MT: draw z16m{1,0} .. z17t{1,0};
 \MT: draw z16t{1,0} .. z17m{1,0};
 \MT: arrow(17b,18b); arrow(17m,18m);
 \MT: draw z17t .. z20t;
 \MT: draw z19b .. z20b; draw z19m .. z20m;
 \MT: z10 = (0,y11b); draw z10 .. z11b;
 \MT: z21 = (w,y20b); arrow(20b,21);
 \MT: arrowlessfeedback(32,20m,11m)(h,10u,10u)
 \MT: arrowlessfeedback(31,20t,11t)(h-12u,10u,5u);
\MTendfigure"Composición\cr en serie"Función característica;

\bigbreak
Primero calculamos, dentro de la composición $\srl|B|C$, el vector
estado-siguiente+salida de |B, que notamos ${}^{|B}\!\!|Q$. Para ello
sustituimos
 $\P|A_{\dot{s}_t \ddot{s}_t i_t \dot{n}_t \ddot{n}_t \ddot{o}_t}$
por
 $\sum_{\dot{o}_t} \P|B_{\dot{s}_t i_t \dot{n}_t \dot{o}_t} \cdot\,
  \P|C_{\ddot{s}_t \dot{o}_t \ddot{n}_t \ddot{o}_t}$
en la ecuación dinámica aplicada reiteradamente y, en el instante final
$\sigma$, dejamos libres los índices $\dot{n}_\sigma$ y $\dot{o}_\sigma$.
 $$\displaylines{
 {}^{|B}\!\!\P|Q_{\dot{n}_\sigma \dot{o}_\sigma} =
   \sum_{\dot{s}_\sigma \ddot{s}_\sigma \ddot{n}_\sigma \ddot{o}_\sigma}
    \P|B_{\dot{s}_\sigma I_\sigma \dot{n}_\sigma \dot{o}_\sigma}
    \P|C_{\ddot{s}_\sigma \dot{o}_\sigma \ddot{n}_\sigma \ddot{o}_\sigma}
  \cdot \hfill \cr \qquad\qquad\qquad \cdot
   \sum_{\scriptstyle \dot{s}_{\sigma-1} \ddot{s}_{\sigma-1}
    \atop \scriptstyle \dot{o}_{\sigma-1} \ddot{o}_{\sigma-1}}
    \P|B_{\dot{s}_{\sigma-1} I_{\sigma-1}
          \dot{n}_{\sigma-1} \dot{o}_{\sigma-1}}
    \P|C_{\ddot{s}_{\sigma-1} \dot{o}_{\sigma-1}
          \ddot{n}_{\sigma-1} \ddot{o}_{\sigma-1}}
  \cdots \hfill \cr \noalign{\vskip-3pt} \hfill \cdots
   \sum_{\scriptstyle \dot{s}_1 \ddot{s}_1 \atop
         \scriptstyle \dot{o}_1 \ddot{o}_1}
    \P|B_{\dot{s}_1 I_1 \dot{n}_1 \dot{o}_1}
    \P|C_{\ddot{s}_1 \dot{o}_1 \ddot{n}_1 \ddot{o}_1}
  \cdot
   \sum_{\dot{o}_0 \ddot{o}_0}
    \P|B_{\dot{S}_0 I_0 \dot{n}_0 \dot{o}_0}
    \P|C_{\ddot{S}_0 \dot{o}_0 \ddot{n}_0 \ddot{o}_0} .\cr}$$
Como resulta que
 $$\forall s, \imath : \;
   \sum_{\ddot{n} \ddot{o}} \P|C_{s \imath \ddot{n} \ddot{o}} = 1 ,$$
y recordando que, merced a la ecuación temporal,
 $\ddot{s}_{t}$
también puede escribirse como
 $\ddot{n}_{t-1}$,
pueden eliminarse todas las apariciones de
 $\P|C_{s \imath \ddot{n} \ddot{o}}$,
con lo que queda:
 $$\displaylines{
  {}^{|B}\!\!\P|Q_{\dot{n}_\sigma \dot{o}_\sigma} =
   \sum_{\dot{s}_\sigma}
    \P|B_{\dot{s}_\sigma I_\sigma \dot{n}_\sigma \dot{o}_\sigma} \cdot
   \sum_{\scriptstyle\dot{s}_{\sigma-1} \atop
         \scriptstyle \dot{o}_{\sigma-1}}
    \P|B_{\dot{s}_{\sigma-1} I_{\sigma-1}
          \dot{n}_{\sigma-1} \dot{o}_{\sigma-1}}
  \cdots \hfill \cr \hfill \cdots
   \sum_{\dot{s}_1 \dot{o}_1}
    \P|B_{\dot{s}_1 I_1 \dot{n}_1 \dot{o}_1}
  \cdot
   \sum_{\dot{o}_0}
    \P|B_{\dot{S}_0 I_0 \dot{n}_0 \dot{o}_0} =
  \P|Q^{|B}_{\dot{n}_\sigma \dot{o}_\sigma}\!
     (\dot{S}_0; I_0, I_1, \ldots, I_\sigma) .\cr}$$
Es decir, que |B funciona dentro de $\srl|B|C$ del mismo modo que solo y,
lo que es más importante,
 ${}^{|B}\!\!\P|Q_{\dot{n}_\sigma\dot{o}_\sigma}$
únicamente depende del estado inicial $\dot{S}_0$ y  del stream de
entrada $I_0, I_1, \ldots, I_\sigma$. Este resultado será útil a
continuación.

Ahora sustituimos
 $\P|A_{\dot{s}_t \ddot{s}_t i_t \dot{n}_t \ddot{n}_t \ddot{o}_t}$
por
 $\sum_{\dot{o}_t} \P|B_{\dot{s}_t i_t \dot{n}_t \dot{o}_t} \cdot\,
  \P|C_{\ddot{s}_t \dot{o}_t \ddot{n}_t \ddot{o}_t}$
en la expresión que calcula $|O^{|A}$, y tenemos que:
 $$\displaylines{
  \P|O^{|A}_{\ddot{o}_\tau}\!\bigl(S^{|A}(0); I_0, I_1, \ldots, I_\tau\bigr)
  = \hfill\cr \qquad =
   \sum_{\dot{s}_\tau \ddot{s}_\tau \dot{n}_\tau \ddot{n}_\tau}
   \Bigl( \sum_{\dot{o}_\tau}
    \P|B_{\dot{s}_\tau I_\tau \dot{n}_\tau \dot{o}_\tau}
    \P|C_{\ddot{s}_\tau \dot{o}_\tau \ddot{n}_\tau \ddot{o}_\tau}
   \Bigr)
  \cdot \hfill \cr \qquad\qquad\qquad \cdot
   \sum_{\scriptstyle \dot{s}_{\tau-1} \ddot{s}_{\tau-1}
    \atop \scriptstyle \ddot{o}_{\tau-1}}
   \Bigl( \sum_{\dot{o}_{\tau-1}}
    \P|B_{\dot{s}_{\tau-1} I_{\tau-1}
          \dot{n}_{\tau-1} \dot{o}_{\tau-1}}
    \P|C_{\ddot{s}_{\tau-1} \dot{o}_{\tau-1}
          \ddot{n}_{\tau-1} \ddot{o}_{\tau-1}}
   \Bigr)
  \cdots \hfill \cr\noalign{\vskip-6pt}\hfill \cdots
   \sum_{\scriptstyle \dot{s}_1 \ddot{s}_1
    \atop \scriptstyle \ddot{o}_1}
   \Bigl( \sum_{\dot{o}_1}
    \P|B_{\dot{s}_1 I_1 \dot{n}_1 \dot{o}_1}
    \P|C_{\ddot{s}_1 \dot{o}_1 \ddot{n}_1 \ddot{o}_1}
   \Bigr)
  \cdot \qquad\qquad\qquad \cr \noalign{\vskip-6pt}\hfill \cdot
   \sum_{\ddot{o}_0}
   \Bigl( \sum_{\dot{o}_0}
    \P|B_{\dot{S}_0 I_0 \dot{n}_0 \dot{o}_0}
    \P|C_{\ddot{S}_0 \dot{o}_0 \ddot{n}_0 \ddot{o}_0}
   \Bigr)
 = \cr \qquad =
   \sum_{\dot{o}_\tau \ddot{s}_\tau \dot{n}_\tau \ddot{n}_\tau}
    {}^{|B}\!\!\P|Q_{\dot{n}_\tau \dot{o}_\tau}
    \P|C_{\ddot{s}_\tau \dot{o}_\tau \ddot{n}_\tau \ddot{o}_\tau}
  \cdot \hfill \cr \qquad\qquad\qquad \cdot
   \sum_{\scriptstyle \dot{s}_\tau \ddot{s}_{\tau-1}
    \atop \scriptstyle \dot{o}_{\tau-1} \ddot{o}_{\tau-1}}
    {}^{|B}\!\!\P|Q_{\dot{n}_{\tau-1} \dot{o}_{\tau-1}}
    \P|C_{\ddot{s}_{\tau-1} \dot{o}_{\tau-1}
          \ddot{n}_{\tau-1} \ddot{o}_{\tau-1}}
  \cdots \hfill \cr \hfill \cdots
   \sum_{\dot{s}_2 \ddot{s}_1 \dot{o}_1 \ddot{o}_1}
    {}^{|B}\!\!\P|Q_{\dot{n}_1 \dot{o}_1}
    \P|C_{\ddot{s}_1 \dot{o}_1 \ddot{n}_1 \ddot{o}_1}
  \cdot
   \sum_{\dot{s}_1 \dot{o}_0 \ddot{o}_0}
    {}^{|B}\!\!\P|Q_{\dot{n}_0 \dot{o}_0}
    \P|C_{\ddot{S}_0 \dot{o}_0 \ddot{n}_0 \ddot{o}_0}
  = \cr \noalign{\vskip3pt}\qquad =
   \P|O^{|C}_{\ddot{o}_\tau}\!\bigl(\ddot{S}_0;
      |O^{|B}(\dot{S}_0; I_0),
      |O^{|B}(\dot{S}_0; I_0, I_1),
    \ldots \hfill\cr\hfill \ldots ,
      |O^{|B}(\dot{S}_0; I_0, I_1,\ldots, I_{\tau-1}),
      |O^{|B}(\dot{S}_0; I_0, I_1, \ldots, I_\tau)
    \bigr)
  . \cr}$$
Con lo que se demuestra que:
 $Z^{\srl\!|B\!|C}_{\prl\!\dot{S}_0 \ddot{S}_0} =
   \vec{Z}^{|B}_{\dot{S}_0} \circ |Z^{|C}_{\ddot{S}_0}$.

\smallskip
Sabiendo los valores determinados $Q$ que toman los vectores de
estado-siguiente+salida, se pueden utilizar los cálculos desacoplados, o
paso a paso, como sigue:
 $$\eqalign{
 Q^{|B} \leapsfrom |Q^{|B} &=
   \srl \prl S^{|A}_{\st B\times} I^{|A} \F_B \cr
 I^{|C} &= O^{|B} = Q^{|B}_{\times\out B}\cr
 Q^{|C} \leapsfrom |Q^{|C} &=
   \srl \prl S^{|A}_{\times\st C} I^{|C} \F_C \cr
 N^{|A} &= \prl N^{|B} N^{|C} =
   \prl Q^{|B}_{\st B\times} Q^{|C}_{\st C\times} \cr
 O^{|A} &= O^{|C} = Q^{|C}_{\times\out C} .\cr}$$

\bigskip
\title4 Composición en paralelo

El nombre de la composición en paralelo tiene sentido porque
 $$Z^{\prl\!|B\!|C}_{\prl\! S_{|B} S_{|C}} =
  Z^{|B}_{S_{|B}} \times Z^{|C}_{S_{|C}} $$
notando $\times$ el producto cartesiano de funciones,
 $[f \times g](x,y) = \bigl(f(x),g(y)\bigr)$.

\goodpage
\title4 Demostración

Notamos $|A=\prl|B|C$ y, por tratarse de strings,
 $$\displaylines{
   \dot{I}_t = I^{|A}(t)_{\inp B \times} \qquad
   \ddot{I}_t = I^{|A}(t)_{\times \inp C} \cr
   \dot{S}_0 = S^{|A}(0)_{\st B \times} \qquad
   \ddot{S}_0 = S^{|A}(0)_{\times \st C} .\cr}$$

La disposición de los índices se muestra en la figura, de la cual
también se deduce que:
 $$ \F_{\prl\!|B\!|C} = \srl \srl
    \prl \prl \ID_{\st B} \op{Cross}_{\st C,\inp B} \ID_{\inp C}
    \prl \F_B \F_C
    \prl \prl \ID_{\st B} \op{Cross}_{\out B,\st C} \ID_{\out C} .$$
% Composición en paralelo
\MTbeginfigure(120,80);
 \MT: pickup thin_pen;
 \MT: x1 = 1/2w; y1b = 0;
 \MT: rectangle(1)(w-40u,55/80h); % 1 is F_{*BC}
 \MT: pickup thick_pen;
 \MT: x2 = x3 = w/2;
 \MT: y2b = y1b + 5v;
 \MT: rectangle(2)(20u,20v); % 2 is F_C
 \MT: y3t = y1t - 5v;
 \MT: rectangle(3)(20u,20v); % 3 is F_B
 \MT: z2lbl = z2; z3lbl = z3;
 \MTlabel(2lbl)"\F_C";
 \MTlabel(3lbl)"\F_B";
 \MT: pickup med_pen;
 \MT: for i := 11 upto 18:
 \MT:  y[i]b = 1/6[y1b,y1t];
 \MT:  y[i]n = 2/6[y1b,y1t];
 \MT:  y[i]m = 4/6[y1b,y1t];
 \MT:  y[i]t = 5/6[y1b,y1t];
 \MT: endfor
 \MT: x11b = x11n = x11m = x11t = x1l;
 \MT: x12b = x12n = x12m = x12t = 1/6[x1l,x2l];
 \MT: x13b = x13n = x13m = x13t = 5/6[x1l,x2l];
 \MT: x14b = x14n = x14m = x14t = x2l;
 \MT: x15b = x15n = x15m = x15t = x2r;
 \MT: x16b = x16n = x16m = x16t = 1/6[x2r,x1r];
 \MT: x17b = x17n = x17m = x17t = 5/6[x2r,x1r];
 \MT: x18b = x18n = x18m = x18t = x1r;
 \MT: draw z11b .. z12b; draw z11n .. z12n;
 \MT: draw z11m .. z12m; draw z11t .. z12t;
 \MT: draw z12b .. z13b; draw z12t .. z13t;
 \MT: draw z12m{1,0} .. z13n{1,0};
 \MT: draw z12n{1,0} .. z13m{1,0};
 \MT: arrow(13b,14b); arrow(13n,14n); arrow(13m,14m); arrow(13t,14t);
 \MT: draw z15b .. z16b; draw z15n .. z16n;
 \MT: draw z15m .. z16m; draw z15t .. z16t;
 \MT: draw z16b .. z17b; draw z16t .. z17t;
 \MT: draw z16m{1,0} .. z17n{1,0};
 \MT: draw z16n{1,0} .. z17m{1,0};
 \MT: draw z17b .. z18b; draw z17n .. z18n;
 \MT: draw z17m .. z18m; draw z17t .. z18t;
 \MT: z10b = (0,y11b); draw z10b .. z11b;
 \MT: z10n = (0,y11n); draw z10n .. z11n;
 \MT: z19b = (w,y18b); arrow(18b,19b);
 \MT: z19n = (w,y18n); arrow(18n,19n);
 \MT: arrowlessfeedback(32,18m,11m)(h,10u,10u)
 \MT: arrowlessfeedback(31,18t,11t)(h-12u,10u,5u);
\MTendfigure"Composición\cr en paralelo"Función característica;

\bigbreak
Sustituyendo
 $\P|A_{\dot{s}_t \ddot{s}_t i_t \ddot{\imath}_t
        \dot{n}_t \ddot{n}_t \dot{o}_t \ddot{o}_t}$
por
 $\P|B_{\dot{s}_t i_t \dot{n}_t \dot{o}_t} \cdot\,
  \P|C_{\ddot{s}_t \ddot{\imath}_t \ddot{n}_t \ddot{o}_t}$
en la expresión que calcula $|O^{|A}$,  tenemos que:
 $$\displaylines{
  \P|O^{|A}_{o_\tau}\!\bigl(S^{|A}(0);
       I^{|A}(0), I^{|A}(1), \ldots, I^{|A}(\tau)\bigr)
  = \hfill\cr \qquad\qquad =
   \sum_{\dot{s}_\tau \ddot{s}_\tau \dot{n}_\tau \ddot{n}_\tau}
    \P|B_{\dot{s}_\tau \dot{I}_\tau \dot{n}_\tau \dot{o}_\tau}
    \P|C_{\ddot{s}_\tau \ddot{I}_\tau \ddot{n}_\tau \ddot{o}_\tau}
  \cdot \hfill \cr \qquad\qquad\qquad \cdot
   \sum_{\scriptstyle \dot{s}_{\tau-1} \ddot{s}_{\tau-1}
    \atop \scriptstyle \dot{o}_{\tau-1} \ddot{o}_{\tau-1}}
    \P|B_{\dot{s}_{\tau-1} \dot{I}_{\tau-1}
          \dot{n}_{\tau-1} \dot{o}_{\tau-1}}
    \P|C_{\ddot{s}_{\tau-1} \ddot{I}_{\tau-1}
          \ddot{n}_{\tau-1} \ddot{o}_{\tau-1}}
  \cdots \hfill \cr \noalign{\vskip-3pt} \hfill \cdots
   \sum_{\scriptstyle \dot{s}_1 \ddot{s}_1 \atop
         \scriptstyle \dot{o}_1 \ddot{o}_1}
    \P|B_{\dot{s}_1 \dot{I}_1 \dot{n}_1 \dot{o}_1}
    \P|C_{\ddot{s}_1 \ddot{I}_1 \ddot{n}_1 \ddot{o}_1}
  \cdot
   \sum_{\dot{o}_0 \ddot{o}_0}
    \P|B_{\dot{S}_0 \dot{I}_0 \dot{n}_0 \dot{o}_0}
    \P|C_{\ddot{S}_0 \ddot{I}_0 \ddot{n}_0 \ddot{o}_0}
 = \cr \qquad\qquad =
  \biggl(
   \sum_{\dot{s}_\tau \dot{n}_\tau}
    \P|B_{\dot{s}_\tau \dot{I}_\tau \dot{n}_\tau \dot{o}_\tau} \cdot
   \sum_{\scriptstyle \dot{s}_{\tau-1} \atop
         \scriptstyle \dot{o}_{\tau-1}}
    \P|B_{\dot{s}_{\tau-1} \dot{I}_{\tau-1}
          \dot{n}_{\tau-1} \dot{o}_{\tau-1}}
  \cdots \hfill \cr \noalign{\vskip-3\jot} \hfill \cdots
   \sum_{\dot{s}_1 \dot{o}_1}
    \P|B_{\dot{s}_1 \dot{I}_1 \dot{n}_1 \dot{o}_1} \cdot
   \sum_{\dot{o}_0} \P|B_{\dot{S}_0 \dot{I}_0 \dot{n}_0 \dot{o}_0}
  \biggr)
  \cdot \qquad\qquad \cr \qquad\qquad\qquad\qquad \cdot
  \biggl(
   \sum_{\ddot{s}_\tau \ddot{n}_\tau}
    \P|C_{\ddot{s}_\tau \ddot{I}_\tau \ddot{n}_\tau \ddot{o}_\tau} \cdot
   \sum_{\scriptstyle \ddot{s}_{\tau-1} \atop
         \scriptstyle \ddot{o}_{\tau-1}}
    \P|C_{\ddot{s}_{\tau-1} \ddot{I}_{\tau-1}
          \ddot{n}_{\tau-1} \ddot{o}_{\tau-1}}
    \cdots \hfill \cr \noalign{\vskip-3\jot} \hfill \cdots
   \sum_{\ddot{s}_1 \ddot{o}_1}
    \P|C_{\ddot{s}_1 \ddot{I}_1 \ddot{n}_1 \ddot{o}_1} \cdot
   \sum_{\ddot{o}_0} \P|C_{\ddot{S}_0 \ddot{I}_0 \ddot{n}_0 \ddot{o}_0}
    \biggr)
  = \cr \noalign{\vskip3pt}\qquad\qquad =
   \P|O^{|B}_{\dot{o}_\tau}\!(\dot{S}_0;
               \dot{I}_0, \dot{I}_1, \ldots, \dot{I}_\tau) \cdot
   \P|O^{|C}_{\ddot{o}_\tau}\!(\ddot{S}_0;
               \ddot{I}_0, \ddot{I}_1, \ldots, \ddot{I}_\tau)
  . \hfill\cr}$$
De modo que $|O^{\prl\!|B\!|C} = \prl |O^{|B} |O^{|C}$ y, en definitiva:
 $Z^{\prl\!|B\!|C}_{\prl\!\dot{S}_0 \ddot{S}_0} =
   Z^{|B}_{\dot{S}_0} \times Z^{|C}_{\ddot{S}_0} $.

\goodpage

Conociendo los valores determinados que toman los vectores |Q, de
estado-siguiente+salida, podemos aplicar los cálculos paso a paso o
desacoplados:
 $$\eqalign{
 Q^{|B} \leapsfrom |Q^{|B} &=
  \srl \prl S^{|A}_{\st B\times} I^{|A}_{\inp B\times} \F_B \cr
 Q^{|C} \leapsfrom |Q^{|C} &=
  \srl \prl S^{|A}_{\times\st C} I^{|A}_{\times\inp C} \F_C \cr
 N^{|A} &= \prl N^{|B} N^{|C} =
  \prl Q^{|B}_{\st B\times} Q^{|C}_{\st C\times} \cr
 O^{|A} &= \prl O^{|B} O^{|C} =
  \prl Q^{|B}_{\times\out B} Q^{|C}_{\times\out C} .\cr }$$


\title4 Composición de realimentación

Véase que:
 $$\fdb |B = \fdb \fdb^{\st B} \F_B = \fdb^{\st B+1} \F_B
   \implies \F_{\fdb\!|B} = \F_B .$$
La figura muestra la situación.
% Composición de realimentación
\MTbeginfigure(90,60);
 \MT: pickup thick_pen;
 \MT: x1 = w/2; y1b = 5v; z1lbl = z1;
 \MT: rectangle(1)(20u,20v); % 1 is F_B
 \MTlabel(1lbl)"\F_B";
 \MT: pickup med_pen;
 \MT: x2o = 0; x2d = x1l; y2o = y2d = 1/4[y1b,y1t];
 \MT: arrow(2o,2d);
 \MT: x3o = x1r; x3d = w; y3o = y3d = 1/4[y1b,y1t];
 \MT: arrow(3o,3d);
 \MT: z4o = (x1r,3/4[y1b,y1t]);
 \MT: z4d = (x1l,3/4[y1b,y1t]);
 \MT: feedback(4,4o,4d)(h-20v,10u,5u);
 \MT: pickup thin_pen;
 \MT: x5 = 1/2w; y5b = 0;
 \MT: rectangle(5)(w-40u,h-15v); % 5 is A = & B
 \MT: pickup med_pen;
 \MT: x6o = x5l; x6d = x1l; y6o = y6d = 2/4[y1b,y1t];
 \MT: % arrow(6o,6d);
 \MT: x7o = x1r; x7d = x5r; y7o = y7d = 2/4[y1b,y1t];
 \MT: % draw z7o .. z7d;
 \MT: % arrowlessfeedback(8,7d,6o)(h,10u,10u);
 \MT: feedback(8,7o,6d)(h,10u,10u);
\MTendfigure"Composición\cr de realimentación"Función característica;

Los cálculos desacoplados son, en este caso, sencillos:
 $$Q \leapsfrom |Q^{\fdb\!|B} =
   \srl \prl S^{\fdb\!|B} I^{\fdb\!|B} \F_B
   \belowdisplayskip=0pt plus 3pt .$$
Y, por lo tanto:
 $$\abovedisplayskip=\abovedisplayshortskip
   \belowdisplayskip=\belowdisplayshortskip
   N^{\fdb\!|B} = Q_{\st B+1\times} \qquad
   O^{\fdb\!|B} = Q_{\times\out B-1} .$$

\bigskip
\title4 Teorema

\label{demostración}
Si $|F(|X)$ es una \definition{función automática} cualquiera, es decir,
una expresión en la que si sustituimos |X por un autómata resulta un
autómata, entonces
 $\forall\! |A, |A': \; |A \nodis |A' \implies |F(|A) \nodis |F(|A')$.

\title4 Demostración

Si $|B \nodis |B'$ entonces
 $Z^{|B}_{\dot{S}_0} = Z^{|B'}_{\dot{S}'_0}$ y, por tanto,
 $\vec{Z}^{|B}_{\dot{S}_0} = \vec{Z}^{|B'}_{\dot{S}'_0}$, y si
 $|C \nodis |C'$ entonces
 $Z^{|C}_{\ddot{S}_0} = Z^{|C'}_{\ddot{S}'_0}$ y, por consiguiente,
 $|Z^{|C}_{\ddot{S}_0} = |Z^{|C'}_{\ddot{S}'_0}$. De aquí se sigue que
 $\vec{Z}^{|B}_{\dot{S}_0} \circ |Z^{|C}_{\ddot{S}_0} =
  \vec{Z}^{|B'}_{\dot{S}'_0} \circ |Z^{|C'}_{\ddot{S}'_0}$,
lo cual prueba que:
 $$ (|B \nodis |B') \land (|C \nodis |C') \implies
     \srl |B |C \nodis \srl |B' |C' .$$

Si $|B \nodis |B'$ entonces
 $Z^{|B}_{\dot{S}_0} = Z^{|B'}_{\dot{S}'_0}$,
y si $|C \nodis |C'$ entonces
 $Z^{|C}_{\ddot{S}_0} =Z^{|C'}_{\ddot{S}'_0}$.
De aquí se sigue que
 $Z^{|B}_{\dot{S}_0} \times Z^{|C}_{\ddot{S}_0} =
  Z^{|B'}_{\dot{S}'_0} \times Z^{|C'}_{\ddot{S}'_0}$,
lo cual prueba que:
 $$ (|B \nodis |B') \land (|C \nodis |C') \implies
     \prl |B |C \nodis \prl |B' |C' .$$

\goodpage

Si $|B \nodis |B'$ entonces
 $Z^{|B}_{\dot{S}_0} = Z^{|B'}_{\dot{S}'_0}$ y, por tanto,
 $|Z^{|B}_{\dot{S}_0} = |Z^{|B'}_{\dot{S}'_0}$,
lo que significa que cualquier stream de vectores de entrada
 $|J(0), |J(1), \ldots, |J(\tau)$ aplicado a uno de los autómatas, |B,
produce el mismo stream de vectores de salida
 $|O(0), |O(1),\ldots, |O(\tau)$ que aplicado al otro, $|B'$.
Y puesto que sucede con cualquiera, también ocurrirá con aquellos
streams de vectores de entrada cuya primera variable repite la primera
variable de salida. Esto prueba que:
 $$|B \nodis |B' \implies \fdb |B \nodis \fdb |B' .$$

La conjunción de estas tres pruebas, con
 $|B = |B' \implies |B \nodis |B'$, permite concluir que:
 $$|A \nodis |A' \implies |F(|A) \nodis |F(|A') .$$

\title4 Teorema

De modo análogo se prueba que:
 $$|A \equiv |A' \implies |F(|A) \equiv |F(|A') .$$

\title2 Subálgebras

\title3 El determinismo

Decimos que $\opd A$ es un \definition{autómata determinístico} si se cumple
que:  $$\no{P}_{\!\!\!\opd A} = 0.$$

Cualquier autómata determinístico $\opd A$ puede ser escrito utilizando,
únicamente, los constructores:  $\srl$, $\prl$, $\fdb$, $\SINK$, $\FORK$,
$\ONE$, $\NAND$.  Resumiendo:
 $$\{ \opd A \} = \Re(\srl, \prl, \fdb, \SINK, \FORK, \ONE, \NAND).$$

La forma canónica de los autómatas determinísticos es análoga a la
general.  Obsérvese que si bien $\ONE$ puede ser descompuesto en los
autómatas $\SINK$, $\FORK$, $\PROB$ y $\NAND$, en cambio $\PROB$ no puede
ser descompuesto en los autómatas $\SINK$, $\FORK$, $\ONE$ y $\NAND$.

Los autómatas determinísticos admiten dos descomposiciones, a saber, la
de \person{Mealy} y la de \person{Moore}.

La descomposición de \person{Mealy}, o en paralelo, del autómata
 $\opd A$ queda:
 $$\opd A = \fdb^{\no{S}_{\!\opd A}}
   \srl \FORK_{\no{S}_{\!\opd A} + \no{I}_{\!\opd A}}
   \prl \opd{F}^N  \opd{F}^O$$
donde $\opd{F}^N$ es la función determinística (véase la
\S\refsc{encaminamiento}) de estado siguiente y $\opd{F}^O$ la función
determinística de salida.

La descomposición de \person{Moore}, o en serie, del autómata $\opd A$
tiene la forma:
 $$\opd A \equiv  \srl \fdb^{\no{S}_{\!M}}
   \srl \opd{F}^S \FORK_{\no{S}_{\!M}} \opd{F}^{O'}$$
donde $\opd{F}^S$ es la función determinística de estado y $\opd{F}^{O'}\!$
otra función determinística de salida.  Se cumple que: $\no{S}_{\!M} \geq
\no{S}_{\!\opd A}$.

\title3 Funciones

Una \definition{función} es cualquier autómata |F tal que:
$\st F = 0$.

Se tiene que:  $\{ |F \} = \Re(\srl, \prl, \SINK, \FORK, \PROB, \NAND)$.

Nótese que ésta es, en general, una función probabilística.  Así
que una \definition{función determinística} $\opd F$ ha de cumplir:
 $\no{S}_{\!\opd F} = 0$, $\no{P}_{\!\!\opd F} = 0$.

Se tiene que:
 $\{ \opd{F} \} = \Re(\srl, \prl, \SINK, \FORK, \ONE, \NAND)$.

Las funciones determinísticas $\opd F$ son funciones, en el sentido
ordinario, del string de entrada en el string de salida,
 $\opd{F} : \B^{\no{I}_{\opd F}} \! \to \B^{\no{O}_{\!\opd F}}$.

\label{encaminamiento}
Una \definition{función de encaminamiento} es una función
determinística en la que cada variable de salida se limita a repetir el
valor de alguna de las variables de entrada. En la \S\refsc{La
construcción} y la \S\refsc{Cross} las figuras muestran las funciones de
encaminamiento.

La composición en serie de funciones $\srl |F |G$  se corresponde con la
multiplicación de matrices, $(|F_{\imath v}) \times (|G_{v o})$.

\title3 Vectores y strings

Es un vector todo aquel autómata |V que cumple:
 $\inp V = 0$, $\st V = 0$.

Los strings son los vectores determinísticos, esto es, aquellos
autómatas $S$ que cumplen:
 $\no{I}_{\opd S} = 0$, $\no{S}_{\!\opd S} = 0$, $\no{P}_{\!\!\opd S} = 0$.

Todos los strings se pueden construir componiendo los autómatas $\ONE$ y
$\ZERO$ en paralelo
 (véase la \S\refsc{Notación automática de strings}):
 $\{ S \} = \Re(\prl, \ONE, \ZERO)$.

\title3 El álgebra de Boole

El \definition{álgebra de {\sc Boole}}\ndx{Boole}{1} estudia las funciones
determinísticas unarias, es decir, los autómatas $\opd B$ tales que:
 $$\no{O}_{\!\opd B} = 1,\quad
   \no{S}_{\!\opd B} = 0,\quad
   \no{P}_{\!\!\opd B} = 0.$$

Puede hacerse la generalización al álgebra automática de una de las
construcciones típicas del álgebra de \person{Boole}.
 $$\op{DeMorgan}(|A) = \srl \srl \prl^{\inpA}\NOT |A \prl^{\outA}\NOT.$$
Se tiene que DeMorgan(OR) = AND, DeMorgan(AND) = OR.  Pero además,
saliéndonos del álgebra de \person{Boole},
 $\op{DeMorgan}(\DEL) \equiv \DEL$,
con la particularidad de que los estados equivalentes son los opuestos.

Cualquier función determinística $\opd F$ puede descomponerse en un paralelo
de $\no{O}_{\!\opd F}$ funciones booleanas, ${\opd B}_o$:
 $$ {\opd F} = \srl \op{Mult}_{\no{O}_{\!\opd F},\no{I}_{\opd F}}
    \bigotimes_o {\opd B}_o$$
donde $\op{Mult}_{m,n}$ es una función de encaminamiento ya definida (en
\S\refsc{La construcción}).

\title3 Cadenas de Markov

Una \definition{cadena de {\sc Markov}}\ndx{Markov}{1} (homogénea) es
cualquier autómata |M tal que:  $$\inp M = 0,\quad \out M = 0.$$
 De otra manera:  $|M = \fdb^n |F_{\square}$, siendo $|F_{\square}$
una función cuadrada, o sea:
 $$|M = \fdb^n |F_{\square} \; : \quad
   \st{|F_{\square}} = 0 \, , \;
   \inp{|F_{\square}} = n = \out{|F_{\square}} \, .$$
Un modelo markoviano $|M_M$ adopta la forma:
 $|M_M = \fdb^n \srl  |F_{\square} \FORK_n $.
Y un modelo markoviano oculto $|M_O$ es:
 $|M_O = \srl  \fdb^n \srl |F_{\square} \FORK_n |F $,
siendo |F una función cualquiera pero con $\inp F = n$.

\title2 Conclusión

El álgebra automática permite una fácil formalización de la semántica de
cualquier entidad que admita ser descrita como un autómata finito binario,
por ejemplo una computadora, un programa o un lenguaje.

\subtitle Agradecimientos

Un trabajo que está en esta misma línea, pero tomando la ruta de
\person{Church} (el cálculo~$\lambda$) en vez de la de \person{Turing}
(los autómatas finitos), y que ha servido de inspiración al presente
Anexo es \person{Delgado}\cite{Delgado1987}. El resto de la inspiración se
tomó del libro de texto de \person{Fernández} y \person{Sáez
Vacas}\cite{Fernández1987}.

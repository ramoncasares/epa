% EPA3.TEX (RMCG19950910)

\title1 El sujeto

\title2 Una advertencia

El conocedor simbólico marca el final de la evolución física por ser el
asiento de toda la posterior evolución cognitiva.  No describiremos la
evolución cognitiva con el mismo nivel de precisión y detalle con el que
describimos la evolución física, porque todavía es un terreno poco
explorado.  Más que explicar, esbozaremos una explicación, de manera que no
será posible implementar un \definition{sujeto}, que así llamaremos al
conocedor simbólico mejorado, a partir de las ideas aquí esbozadas. Del
mismo modo, es posible que el desarrollo de la teoría llegue a demostrar que
desde el conocedor simbólico hasta el sujeto son precisos pasos intermedios,
pero es más probable que no.  Es incluso posible que no tenga sentido hablar
de un conocedor simbólico que no sea un sujeto.

\title2 La complejidad

Supondremos que el principal problema que ha de resolver un conocedor
simbólico es el de manejar un enorme caudal de percepciones.  Esta
suposición, cuya bondad sólo podrá ser evaluada cuando se complete la
teoría, se apoya en una intuición, a saber, que la gran complejidad de
un motor sintáctico sólo puede justificarse si el problema a resolver
es igualmente complejo.  En un problema aparente la complejidad sólo se
puede medir por el número de variables aparentes, $\no n$, $\no v$ y
$\no m$, y su frecuencia de cambio.

Como ya vimos al tratar sobre las lógicas analíticas, la manera de
disminuir la complejidad de un problema consiste en hacer partes, o sea, en
hacer una traslación de la que resulte un cierto número de problemas
trasladados más sencillos.  Esto puede hacerse también con problemas
aparentes, dividiendo en grupos las variables aparentes, o sea, partiendo
la apariencia.

Trataremos primero la partición del problema aparente y después la del
problema simbólico.  Si la complejidad del problema de la supervivencia para
el sujeto es tan grande como suponemos, es muy posible que ambas sean
necesarias, aunque la partición del problema aparente suponga una etapa de
preprocesamiento no simbólico que, teóricamente, podríamos omitir. No la
omitiremos porque ayuda a entender y explica algunas peculiaridades.

\goodpage

\labeled\title2 La partición aparente

\title3 La modelación analítica

Se puede suponer que el universo es un todo compuesto de varias partes. Para
representar más claramente la situación tomaremos el álgebra automática como
base de estos razonamientos.  En este caso se puede suponer que el
autómata~|R, que es la realidad, o sea, el reflejo interno del universo
exterior, se compone de una serie de autómatas $|R_i$ conectados de una
determinada manera.  La modelación puede entonces reconstruirse en la forma
de dos problemas relacionados.  Uno, el
\definition{problema de la partición}, consistirá en determinar cómo
están conectadas las partes $|R_i$, esto es, qué variables de salida de
cada uno son qué variables de entrada de cada otro.  El otro problema, la
\definition{particularización}, consistirá en modelar cada parte, esto
es, determinar cómo es cada uno de los autómatas $|R_i$.

La efectividad de la modelación analítica será mayor si el
conexionado es tal que la modelación de una parte puede ser hecha
independientemente de la modelación de la mayoría de las otras partes.
Esta condición se cumpliría completamente si ninguna variable de salida
de una parte $|R_j$ fuera una variable de entrada a otra parte $|R_k$; esto
es, si la realidad~|R fuera, salvo reorganizaciones o encaminamientos de
las variables, un paralelo de sus partes, $|R = \prl^n |R_i$.

Otra condición que se impone a la modelación analítica es que el
problema del aprendiz resultante, o sea, el problema de la supervivencia en
el que la realidad~|R ahora compuesta de varias partes $|R_i$ sustituye al
universo~|U, pueda ser descompuesto en varios problemas independientes.
Esta condición se cumpliría completamente en las mismas circunstancias
que la condición anterior, o sea, si $|R = \prl^n |R_i$.

En general el requisito de independencia luchará contra la condición de
modelación que requiere que puedan predecirse las reacciones del
universo.  Cuando prevalece el requisito de independencia sobre la
condición de modelación ocurre una \definition{ilusión} que es, pues,
una distorsión que se impone a la realidad para simplificarla.

\title3 La resolución analítica

Hemos tratado los universos partidos, y ahora trataremos las soluciones
partidas, porque también puede suponerse, si ello proporciona algún
beneficio, que la solución es un todo partido.

La resolución de un problema por partes, o \definition{resolución
analítica}, consiste en determinar la disposición y el contenido de las
partes, de modo que la solución sea un todo adecuadamente partido.  El
número de partes puede ser uno,~1.

La resolución analítica de un problema, como la modelación analítica, puede
verse como la resolución de dos subproblemas:  el problema de la partición,
que consiste en determinar la mejor partición del todo, y la
particularización, que, en este caso, consiste en determinar el contenido de
cada parte de manera que el todo solucione el problema.

También en este caso, como en el de la modelación analítica, para
poder sacar partido de la partición, conviene que la particularización
de cada parte pueda ser hecha con independencia de la de las demás.

\title3 La partición y la particularización

Dada la naturaleza de las lógicas analíticas, la resolución del
problema de la partición puede ser hecha gradual o reiteradamente.  Es
decir, que una vez partido el todo, la particularización de una parte
puede recomenzar el ciclo, esto es, la parte puede ser partida.

El problema de la partición y la particularización están
relacionados.  Si vale suponer que la bondad de una partición es igual a
la bondad de la mejor de sus particularizaciones, entonces es posible una
resolución a dos velocidades, una lenta en la que se busca la mejor
partición y otra, más rápida, en la que se busca, para cada
partición propuesta, la mejor particularización.  Pero también son
posibles otras modalidades en las que la interacción entre ambos
problemas es menos jerárquica.  Por ejemplo, puede ser interesante
modificar una de las partes de la partición, digamos uniéndola a alguna
otra, cuando se descubre que su particularización es difícil.

\labeled\title3 La atención

A la facultad de resolver una de las partes del problema, y no las otras
partes, la denominamos \definition{atención}.  Determinar qué partes
atender, y cuáles no, es resolver un problema que denominamos el
\definition{problema de la atención}.

La atención permite concentrar el uso de los recursos en una de las
partes en las que ha quedado partido un problema, lo que, dada la
inevitable finitud de los recursos disponibles, justifica la necesidad de
una resolución analítica del problema de la supervivencia.

La atención presupone la independencia de las partes.  La atención
convierte a la parte en \definition{objeto}.

\goodpage

\title3 Las ilusiones

Hemos visto que algunas ilusiones se deben a que la resolución del
problema de la partición en la modelación es demasiado rígida,
tendiendo a hacer objetos donde sería mejor no hacerlos.  Otras ilusiones
se deberán, seguramente, a fallos en la resolución del problema de la
atención en la modelación, que provocarán que no se atienda lo que
efectivamente importa.

\title3 El preproceso

Podemos imaginar un sujeto que efectúe una modelación analítica,
resolviendo o no, analíticamente o no, algunos de los problemas parciales.
Los problemas solucionados en esta etapa previa de preprocesamiento no
necesariamente simbólico, pero sí analítico, no alcanzarán el
simbolismo.

En el caso de este sujeto con preprocesamiento de la apariencia, lo que
alcanza la parte simbólica no es la apariencia bruta sino una apariencia
de objetos.  Esto no es fundamental para la teoría del sujeto, pero puede
explicar algunas peculiaridades, como las ilusiones, que afectan a las
personas.


\title2 La partición simbólica

\labeled\title3 Un planteamiento general

La partición simbólica se produce al aplicar a la expresión que
representa un determinado problema, $\problem$, una traslación
$\traslation$ que la transforma en otra expresión que integra a un cierto
número de expresiones, cada una de las cuales representa un subproblema, o
sea, un nuevo problema.  La manera de integrar estas expresiones debe
indicar los condicionantes que cada subproblema impone sobre el problema
total, por ejemplo, si solucionar un subproblema concreto es indispensable
para solucionar el problema total o si con solucionar uno de los
subproblemas es suficiente.  Podemos suponer que los subproblemas quedan
integrados en una expresión booleana que indica estas condiciones, y lo
representaremos así:
 $$\traslation(\problem) = \land_i \lor_j \lnot_k \; \problem_{ijk} .$$

Este planteamiento es demasiado general y, sobre todo, es demasiado
rígido, de modo que es posible que no se pueda alcanzar una solución que
cumpla todas las condiciones, en cuyo caso todo el costoso proceso
resolutivo sería inútil.

\labeled\title3 La consciencia

A continuación, no siendo capaces de diseñar completamente un sujeto,
nos limitaremos a apuntar algunas posibilidades.

Para posibilitar la resolución de un gran número de problemas es necesario
resolver muchos de ellos simultáneamente, en paralelo, y para asegurar la
coherencia y la armonía de todos se puede erigir una estructura jerárquica
que controle el proceso.  Sea la \definition{consciencia} el nivel superior
de esta jerarquía, cuyo objetivo es asegurar la unidad del sujeto y que para
ello sólo tratará un único problema, en cada momento el más difícil de
reconciliar con los ya solucionados.  Esto significa que el sujeto tendrá un
pensamiento básicamente paralelo pero sólo tendrá consciencia de una
secuencia de pensamiento.

La herramienta de trabajo de la consciencia es el yo.  El yo es la solución
encontrada hasta el momento, de modo que cuando un nuevo problema alcanza la
consciencia lo primero es comprobar si el yo, tal como está definido en ese
momento, soluciona el problema.  Para ello lo más probable es que haya de
reducirse el yo total a los aspectos, en forma de valores y creencias, que
son de interés en el problema planteado (véase la
\S\refsc{creencias}).

La consciencia reconstruye un único problema del cual el yo es la solución
única. Llamaremos \definition{problema del sujeto} a este problema, que es
una versión simbólica y resumida del problema interiorizado de la
supervivencia. Se efectúan tres resúmenes: primero la apariencia bruta se
simplifica como una apariencia de objetos, después, de entre todos los
problemas simbólicos a resolver en cada momento, sólo uno alcanza la
consciencia donde, por último, este problema se unifica con todos los
problemas que ya la alcanzaron con anterioridad.

La consciencia debe disponer de canales de comunicación con los niveles
inferiores.  La utilidad principal de estos canales descendentes será la de
trasladar información relativa al filtrado de los problemas ascendentes. Por
ejemplo, un problema mecanizado no debe alcanzar la consciencia; y
recuérdese (en la \S\refsc{mecanización}) que el conocedor simple ya era
capaz de mecanizar problemas.

Como consecuencia de esta organización del sujeto, si fuera precisa la
comunicación entre sujetos, entonces lo más eficiente sería comunicar
los problemas, soluciones y resoluciones que alcanzan la consciencia,
porque son los más importantes para el sujeto, y utilizando la misma
sintaxis que utiliza la mente del sujeto, porque al no requerir
traducción exigiría el mínimo esfuerzo de computación.

\goodpage

\labeled\title3 El yo

El \definition{yo} es la solución del problema del sujeto, por lo que
sustituye a la incógnita, o sea, ocupa en el problema el lugar de la
libertad, y por esto el sujeto experimenta el yo como el lugar en donde
está el \definition{libre albedrío}.  Pero el propósito del yo, como
el de la consciencia, es dar unidad al sujeto, por lo que el yo ha de ser la
solución única de todos los problemas que van llegando a la consciencia.
Así pues, la tarea del yo consiste en facilitar la interacción entre
distintos problemas, trasladando condiciones desde los ya pasados hasta el
presente.

Por consiguiente, el yo tiene que ser una solución abierta al futuro, y
no una solución definitiva, o cerrada. El yo no puede ser un
comportamiento, ya que si el yo, que compendia el pasado del sujeto, fuera
un comportamiento, entonces sería imposible determinar, a la llegada de
cada nuevo problema a la consciencia, las consecuencias que acarrearía
sobre los problemas previamente solucionados, pero ya olvidados, modificar
tal comportamiento. La consciencia necesita retener más información que
la contenida en un comportamiento para resolver su difícil problema.

Por no hacer el problema de la consciencia aún más arduo es por lo que los
problemas mecanizados, que son aquéllos a los que ya se les ha encontrado
una solución definitiva, no deben alcanzar la consciencia. Incluso aquellos
otros problemas para los que el sujeto no ha encontrado el comportamiento
que los reduce, o sea, no mecanizados, pero de los que conoce un modo de
resolución que los soluciona definitivamente, también deben ser apartados de
la consciencia.

Eliminados de la consciencia aquellos problemas para los que el sujeto tiene
una solución o, al menos, una resolución definitiva, sólo quedan en
ella los problemas abiertos.  Por esto el sujeto, en vez de codificar el yo
como una solución, cerrada, codifica el yo como un problema, abierto.  La
tarea de la consciencia es así factible, ya que si el nuevo problema es
idéntico al anterior, excepto porque incluye una condición adicional,
entonces la solución del nuevo problema también es solución del
anterior.  Además, puesto que el sujeto es un muy potente resolutor de
problemas simbólicos, la transición desde el problema hasta su
solución le resulta fácil, y el sujeto puede usar, y usa, el problema
para referirse perifrásticamente a su solución.  El yo se presenta, a la
vez, como el problema del sujeto y su solución.

\label{creencias}
La tarea de la consciencia consiste en reformular el nuevo problema que la
alcanza como una condición adicional del problema anterior. Denominamos
\definition{creencias} a las condiciones que definen el yo. Las creencias
son acumulativas.

La acumulación de condiciones vale hasta que el problema se queda sin
solución.  Este yo paradójico, que es problema pero no solución (véase la
\S\refsc{Una digresión paradójica}), es inviable. Para paliar la situación,
el sujeto limita el ámbito de aplicación de las creencias conflictivas
estableciendo qué creencias son pertinentes en cada circunstancia. Llamamos
\definition{valores} a las condiciones que rigen sobre las creencias, o
sobre otros valores, siendo condiciones de condiciones.  A diferencia de las
creencias, los valores no son acumulativos, ya que son jerárquicos.

Cuando aparecen contradicciones entre los propios valores tiene lugar una
crisis de valores.  Entonces el sujeto puede ignorar, momentáneamente, la
nueva creencia, o alguna de las creencias previas que ésta contradice. A
la larga, sin embargo, esta negación de la realidad amenaza la coherencia
del sujeto.  Otra posibilidad, más compleja, consiste en reorganizar el
sistema de valores. Y si la consciencia no fuera capaz de superar la crisis
de valores, entonces la reestructuración, que sería aún más
complicada, habría de alcanzar los dominios del inconsciente, cuyo estudio
inició \person{Freud}\cite{Freud1900}.

Esas peligrosas situaciones paradójicas en las cuales el problema del
sujeto se queda sin solución han de ser excepcionales o, dicho de otro
modo, el problema del sujeto tendrá normalmente muchas soluciones. Luego
en circunstancias normales, los valores y creencias que delimitan el yo son
insuficientes para determinar el comportamento consciente del sujeto, y de
aquí la necesidad de la voluntad.  La \definition{voluntad} es el dato
libre, o sea, no condicionado por el problema de la supervivencia, que
utiliza el sujeto para determinar su comportamiento consciente, una vez
considerados sus valores y sus creencias. Podemos llamar \definition{gustos}
a las pautas que organizan la voluntad.

Mientras el aprendiz solamente modelaba la apariencia, el sujeto modela
todo el problema de la supervivencia.  Así que conviene definir la
realidad como aquella parte de la condición del problema que sustituye
convenientemente a la apariencia, porque de este modo vale tanto para el
aprendiz como para el sujeto.  Definida de esta manera, la realidad del
sujeto es gradual, ya que los valores y las creencias son tanto más
reales cuanto más útiles resultan en la predicción de la futura
apariencia.

No iremos más allá.  Lo importante es que conceptos tan escurridizos
como consciencia y yo pueden corresponder a entidades operativas dentro de
la teoría presentada.

\goodpage

\labeled sujeto\title2 Conclusión

Un \definition{sujeto} es un conocedor simbólico capaz de enfrentarse a un
problema de la supervivencia complejo.  Para ello dispone de una estructura
de resolución jerárquica a cuya cúspide denominamos consciencia.  El
objetivo de la consciencia es dar unidad al sujeto.  La herramienta de la
consciencia es el yo.

El yo es la solución del problema que, al resolverse, determina el
comportamiento consciente del sujeto, pero no cristaliza como
comportamiento, sino que permanece definido como problema. Por esto, el yo
es libertad condicionada, siendo los valores y las creencias las
condiciones que lo definen por delimitación.

El hilo de la consciencia en donde habita el yo del sujeto es el resumen de
la trama de corrientes inconscientes que se urden con la apariencia de
objetos resultante del preprocesamiento no simbólico de la apariencia
bruta.

\wverb\auxf"\iverb\tocf|\goodpage|" % \goodpage in the toc

\labeled La solución\title1 La soluci^^f3n

\title2 Segunda advertencia

Hemos descrito una cierta secuencia resolutiva del problema aparente.  La
secuencia tiene algún interés en sí misma, pero sería mucho más
interesante si tuviese alguna relación con la evolución
darwiniana\cite{Darwin1859} de la que es resultado la persona.

A continuación trataremos de deducir algunos rasgos del sujeto.  Si alguna
persona piensa que tales rasgos son imposibles en una persona, entonces o
bien estamos deduciendo incorrectamente la filosofía del sujeto o bien la
secuencia resolutiva mostrada sólo tiene el interés de ser una de las
resoluciones teóricas del problema aparente.

Debe advertirse que el contenido de este capítulo es filosófico,
metafísico según veremos, por lo que sus aseveraciones no pueden ser
verificadas objetivamente como las de los capítulos precedentes.  Por
esta razón pasaré a ser beligerante y utilizaré la primera persona
para expresar mis opiniones.


\title2 El subjetivismo

El sujeto no sabe del sujeto, sino que el sujeto se ve como su yo.  Yo soy
un yo sabiéndome yo.

El yo es libertad ejercitada, o sea, es libertad condicionada por la
biografía y la necesidad de vivir.  Yo soy libre, dispongo de libre
albedrío.

El sujeto trata, sin descanso, de no fallar en la resolución del problema
aparente de la supervivencia.  Yo quiero vivir.

El sujeto tiene una lógica simbólica, con sintaxis y semántica.  La
sintaxis es lo que permite pasar, de tratar soluciones, a tratar problemas
y resoluciones.  Hay un problema y yo soy la solución.

Pero algún sujeto, como resolutor general de problemas, puede llegar a
percatarse de que si el problema de la supervivencia tuviera una
solución, entonces no le haría falta ser un resolutor general de
problemas y le bastaría con ser un ejecutor de la solución concreta.
Yo moriré.

El problema que llega a la consciencia del sujeto, es decir, el problema
del sujeto, es un resumen extremado del problema de la supervivencia.  Es
un problema interior, simbólico o sintáctico, que resulta de la
traslación del problema exterior de la supervivencia.  Yo estoy en un
mundo sintáctico.

El sujeto no puede representarse algo que no sea representable en su
lógica.  Por esto el sujeto considera que su lógica puede explicarlo
todo, es decir, considera que es completa, comprensiva.  Yo puedo saberlo
todo.

La sintaxis exige una gran cantidad de recursos de computación y tiene
autonomía, aunque carece de valor en cuanto pierde la referencia
semántica que, en última instancia, proviene del problema de la
supervivencia.  Yo puedo saber absurdos, paradojas.

La \definition{filosofía subjetiva} mantiene la primacía de la
apariencia hasta sus últimas consecuencias.  O sea, la filosofía
subjetiva postula que puede prescindirse del problema exterior y de la
lógica externa pero no de la apariencia exterior y, si se mantiene el
problema interior y la lógica interna, nada cambia visto desde el
interior.


\labeled\title2 El mundo

El sujeto modela, no solamente la apariencia, sino el problema de la
supervivencia entero.  En el problema del sujeto, la realidad sustituye a
la apariencia, o sea, al exterior.  El sujeto es libre de hacer su
voluntad, pero si no somete la realidad a la apariencia, entonces errará
sus predicciones y sus cálculos serán equivocados.  Entre la libre
voluntad y la sometida realidad se encuentra la libertad en todos sus
grados.

Queda el todo, que llamamos \definition{mundo}, partido en
dos\cite{Schopenhauer1818}: la realidad, exterior y sometida a las leyes de
la naturaleza; y yo, interior, que residiendo en la libertad me rijo por mi
voluntad. La realidad no lo es todo.  Yo quedo fuera de la realidad.  La
libertad es un concepto ajeno a la realidad.


\title2 El conocimiento

\title3 La descripción

La realidad es la mejor descripción que tenemos de la apariencia externa.
Se puede suponer, si ello tiene algún valor epistemológico, que la
apariencia externa es generada por el \definition{cosmos}.  Si se acepta
esta suposición, entonces se sigue que la realidad y el cosmos deben de
ser, aproximadamente, iguales.

La realidad tiene como objetivo inmediato permitir la solución del
problema del sujeto y como objetivo final permitir la solución del
problema de la supervivencia.  Para cumplir ambos objetivos, la realidad
tiene que describir la apariencia externa, ésta es la
\definition{exigencia descriptiva}.  Interesan las descripciones que
permitan prever lo que pasará, ya que como vimos en el caso del aprendiz
(en la \S\refsc{El modelo}), el beneficio de la lógica interna se realiza
si la realidad adelanta a la apariencia.  En resumen, \definition{conocer}
es describir y el \definition{conocimiento} individual es la realidad de
cada uno.

Una realidad determinística facilita la resolución del problema del
sujeto.  A la exigencia de una realidad determinística la denominamos
\definition{exigencia determinística} y su campeón fue
\person{Einstein}.  Pero la exigencia descriptiva es más fuerte que la
determinística, como arguyó \person{Bohr}, porque la primera atañe al
problema y la segunda a su resolución.

\label{ciencia}
La tarea de definir una realidad interpersonal, teniendo en cuenta la
exigencia descriptiva, es denominada \definition{ciencia}.  El resultado de
la ciencia, denominado conocimiento científico, es una descripción de la
apariencia externa de validez social realizada en cierto lenguaje
simbólico.  Llamaremos \definition{matemáticas} a la rama de la ciencia
interesada en el lenguaje simbólico descriptivo y \definition{física} a
la rama de la ciencia interesada por las descripciones.

La realidad es objetiva.  Esto significa que es imposición, que,
independientemente de la voluntad del yo, se puede comprobar si una
descripción salva la exigencia descriptiva, o no. El conocimiento es
objetivo, por serlo cada realidad.  Surgen
dificultades\cite{Kuhn1970}$^,$\cite{Feyerabend1988} porque la realidad del
sujeto es gradual y su apariencia no es la apariencia bruta, de manera que
resulta difícil distinguir la apariencia (la observación) de la realidad (la
teoría), aunque en la práctica científica no parece imposible valorar el
alcance predictivo de las teorías.

Ni la realidad ni su versión social, el conocimiento científico, son
definitivas.  Esto es consecuencia de que el problema de la modelación es
aparente (véase la \S\refsc{modelación aparente}), y por lo tanto no tiene
una solución definitiva.


\title3 La física

Los fenómenos físicos son, en última instancia, mediciones.  Medir es
comparar contra cierto patrón elegido convencionalmente.  Como no tiene
sentido medir el patrón, resulta que cualquier teoría física completa
ha de ser paradójica.

Convendría tener todo el conocimiento reducido a una única descripción.
Porque, de haber más, cada descripción proporcionaría una previsión y, no
siendo coincidentes, no podría asegurarse lo que ocurriría en el futuro.

\labeled\title2 El saber

\title3 La explicación

No todo el saber es conocimiento.  Llamamos \definition{saber} a las
explicaciones de la filosofía.  Decimos que la \definition{filosofía}
explica porque llega hasta el final, es decir, hasta mi.

Es \definition{explicación} la solución que yo juzgo buena.  El cambio
es peligroso, porque morir es cambiar.  El no cambio es bueno, porque no
cambiar es vivir.  Para hacer bueno lo peligroso ha de convertirse el
cambio en no cambio.  Por esto, \definition{explicar} es convertir el
cambio en no cambio.  El no cambio no ha de ser explicado.

Si algo cambia puedo pensar que no cambia, puedo pensar que toma diferentes
aspectos.  Llamo \definition{ente} a aquello que no cambia sino de aspecto.
Para que un ente no me perturbe debe ser inmutable y sus cambios de aspecto
deben ser reducidos a leyes inmutables.  Una ley inmutable enuncia que un
ente determinado, en determinadas condiciones, siempre sufre un determinado
cambio de aspecto.  Puesto que ocurre siempre así, el cambio no debe ser
considerado cambio.  Empero, para poder despreciar el cambio, debe
concederse una mayor importancia a la ley inmutable que al cambio por ella
reducido.  Niego el cambio inexplicable, o
\definition{caos}, porque me asusta.

La explicación es lo que yo tomo por bueno, y tomo la muerte por mala. Por
esto la urgencia de la explicación es distraer la muerte.  Dentro de la
explicación estoy yo y mi vida.  Si la explicación es completa y lo cubre
todo, entonces he eliminado la muerte.  Si hay distintas explicaciones para
las distintas circunstancias, entonces no hay una explicación completa.  Por
lo tanto, quien consigue una explicación completa y única niega la muerte y
su vida no es absurda; pero mi explicación no es completa.

Sé el pasado porque ya lo he convertido en no cambio.  Ignoro el futuro
porque es imposible eludir la muerte y convertirla en no cambio.  Aunque es
tan seguro que moriré como que he nacido, no es posible explicar la
muerte.

\labeled\title3 La metafísica

La ciencia no sabe las explicaciones últimas, porque no me alcanza; la
ciencia describe.  Una descripción es una explicación, aunque no
concluyente.  Conocer es convertir la cambiante apariencia en símbolos
permanentes.  La explicación abarca todo el problema, incluye tanto la
condición, y en ella la realidad, que es el modelo de la apariencia, como
la libertad, que soy yo.

Denominamos \definition{saber puro} a aquel saber que no es conocimiento.
Por ejemplo, los fundamentos del conocimiento no son conocimiento, son
saber puro.  Denominamos \definition{metafísica} a la parte de la
filosofía que no es ciencia, es decir, la metafísica es el dominio del
saber puro.

La realidad es objetiva, puede ser refinada para que describa mejor la
apariencia externa, pero al abandonar los dominios del conocimiento, y
penetrar en los del saber puro, se pierde toda referencia externa.  Por
consiguiente la metafísica, sin posibilidad de contraste contra el
exterior, no es objetiva, es libre, es subjetiva.  El saber puro se refiere
a mi, y yo soy libertad y voluntad.


\title3 Historia

Desde que \person{Descartes}\cite{Descartes1641} argumentó que el saber
seguro era el saber dentro del yo, el problema de la epistemología ha sido
determinar la posibilidad del conocimiento.  La solución del propio
\person{Descartes} se basa en que yo soy finito pero tengo la idea del
infinito.  Como lo infinito no cabe en lo finito, tiene que existir lo
infinito.  Lo infinito es Dios, y Dios quiere que yo conozca el universo por
Él creado.

Para el empirismo británico el conocimiento es el resultado de aplicar la
operación de inferir a los datos de la experiencia, o sea, a la apariencia.
Pero con \person{Hume}\cite{Hume1748} esta posición llega al escepticismo,
porque ``si hubiera alguna sospecha de que el curso de la naturaleza pudiera
cambiar, y que el pasado pudiera no ser pauta del futuro, toda experiencia
se haría inútil, y no podría dar lugar a inferencia o conclusión alguna.  Es
imposible, por tanto, que argumentos algunos de la experiencia puedan
demostrar esta semejanza del pasado con el futuro; puesto que todos estos
argumentos están fundados sobre la suposición de aquella semejanza''.

\person{Kant}\cite{Kant1787} eliminó el {\it Deus ex machina}, superó
el escepticismo y dio la primera solución al problema de la posibilidad
del conocimiento. Descubrió que es precisa una lógica del yo o, según
la expresión de \person{Kant}, descubrió que existen los juicios
sintéticos {\it a priori}, y entendió que el conocimiento era la
apariencia del universo reducida a la lógica del yo o, según
\person{Kant}, reducida a las categorías.

Pero para \person{Kant} las categorías eran necesarias (véase en su
``Crí\-tica de la razón pura'', edición de 1787, la \S27 de la analítica de
los conceptos de la lógica transcendental).  Esta proposición metafísica
tiene como consecuencia, no sólo la posibilidad del conocimiento, sino
también la posibilidad de un conocimiento absoluto o racional.  La lógica,
definida por sus categorías, ha de ser única si las categorías son
necesariamente como son.  Llamaremos \definition{lógica universal}, o
\definition{razón}, a esta lógica única.

Supuesta la existencia de la lógica universal, si mi conocimiento salva
la apariencia, como se creía que lo hacía la física de
\person{Newton}, entonces la realidad es de la única manera racionalmente
posible.  Mi conocimiento es el conocimiento, no hay otro posible.  La
ciencia queda así fundada racionalmente, esto es, la ciencia no necesita
de la metafísica para dar explicaciones completas y concluyentes.

Pero la necesidad de las categorías es un postulado demasiado exigente, y
sólo es justificable si, por imperativos metafísicos, deseamos tener la
posibilidad de alcanzar un conocimiento absoluto.

\labeled\title3 El límite

La necesidad de las categorías me coloca en una posición inmerecidamente
favorable.  ?`Por qué había yo de tener un conocimiento absoluto? (esto es,
un conocimiento transcendente).  El subjetivismo es pesimista, estima que el
conocimiento no es absoluto y que las explicaciones últimas, que el
conocimiento remite al saber puro, son libres e incompletas.  Que mi
conocimiento no sea absoluto tiene el atractivo de generalizar y congeniar
con dos convicciones anteriores:  que mi posición no es absoluta, convicción
impuesta por \person{Galileo}, y que mi tiempo no es absoluto, convicción
impuesta por \person{Einstein}.

\smallskip
Explicación subjetivista: Lo primero es el problema de la supervivencia, o
sea, yo quiero vivir pero yo moriré.  Puesto que el problema de la
supervivencia se define en apariencias, las mejores soluciones son
aquéllas capaces de resolver cualquier problema.  Llamamos sujeto al
resolutor general que, para poder resolver los problemas, ha de ser capaz de
representarlos en su lógica simbólica.  Un problema es la expresión de
cierta libertad y cierta condición.  Cuando el sujeto se representa el
problema, su explicación es que la libertad soy yo y la condición es la
realidad.  Es entonces cuando el sujeto descubre que lo primero es el
problema de la supervivencia, o sea, yo quiero vivir pero yo moriré.
\smallskip

La explicación subjetivista establece que la muerte es la causa de la
explicación.  La explicación de la muerte sería la explicación de la causa
de la explicación.  Por consiguiente, para el subjetivismo, ni la muerte ni
la vida tienen explicación.  El sujeto es un buscador compulsivo de
explicaciones y, por esto, el sujeto busca explicaciones incluso donde no
las hay.

Los límites del saber quedan establecidos así por el subjetivismo.  Las
descripciones del conocimiento pueden seguirse hasta llegar, finalmente, a
las explicaciones del saber puro.  Esto consiste en ubicar la realidad en el
más amplio marco del problema de la supervivencia en el que estoy yo. Y las
explicaciones del saber puro son libres pero incompletas.

El saber puro es libre porque se refiere a mi, que soy libertad.  El saber
puro es incompleto porque la muerte no puede explicarse.  La muerte es el
cambio que no puede tornarse no cambio.  La muerte es el cambio.

Si defino el \definition{sentido} como algo que me justifica desde afuera,
algo cuyo valor es absoluto, o simplemente me transciende, entonces ni el
mundo exterior ni yo mismo ni mi vida ni mi muerte tienen sentido.

Se explica para vivir, pero vivir no tiene explicación.

Morir tampoco.

\title3 La tolerancia

Para el subjetivismo, el saber puro es libre e incompleto.  En negativo,
esto significa que la metafísica no tiene condiciones y que la
metafísica no es completa.  Esta proposición metafísica, el saber
puro es libre e incompleto, hace al subjetivismo tolerante y resignado.

Si la metafísica es incompleta, entonces no tenemos respuesta a todas las
preguntas.  Explicarlo todo es útil, pero sólo hasta que queremos
explicar la causa de la explicación.  Si la metafísica no tiene
condiciones, entonces cualquier metafísica es posible.  Así que la
metafísica subjetiva no excluye a ninguna otra metafísica.  Todo mi
saber es demostrable y patente; pero también lo es el tuyo aunque difiera
del mío.

Estoy diciendo que si se acepta la metafísica subjetivista, entonces se
sigue que cualquier metafísica es posible.  Aunque lo parezca, esto no es
una contradicción, ya que también la metafísica subjetivista es posible.
Pero es que, además, es un hecho que distintas personas sostienen
metafísicas diferentes, por lo que cualquier metafísica que excluya a las
demás es incapaz de acomodar este hecho.

\goodpage

\title3 El individuo

Puesto que mi lógica no es absoluta ni me transciende, mis explicaciones
de la apariencia no son más valiosas que yo, o más exactamente, mis
descripciones de la apariencia valen menos que mi vida.

Tú, una flor y el sol recibís significado de mi querer vivir.  Es un
significado subjetivo, no esencial.  Si yo no tuviera ansia de vivir,
entonces no necesitaría, ?`para qué?, explicar las apariencias; no
habría problema ni realidad. Mi vida da significado a las descripciones y,
en general, a las explicaciones.

El \definition{individuo}, el yo que aparece con el sujeto, hace que
parezca que la vida, o sea, el problema de la supervivencia, es un problema
del individuo.  No es así.  Que la consciencia de la vida aparezca en
individuos, en yoes indivisibles, no implica que la vida del individuo sea
toda la vida.  Además, ocurre que si yo fuera el único ser vivo,
entonces moriría rápidamente.

Mi vida es más que yo.  Mi vida está constituida por muchos otros yoes
y no yoes.  El yo es intransferible, así que el otro yo es
\definition{él}.  Disponer de una lógica simbólica y de un yo no es
imprescindible para vivir.  Una flor vive, aunque no se percate de ello.

\label{comunicación}
Eres \definition{tú\/} aquel otro yo, o él, con el que me estoy
comunicando.  Tú y yo nos traspasamos significados.  Para ello es
necesario que tú y yo compartamos la misma semántica.  Compartir la
sintaxis es útil, pero lo imprescindible es compartir la semántica, de
manera que es posible comunicarse con animales que no disponen de una
lógica simbólica, pero que comparten nuestra vida.

Porque mi vida no soy únicamente yo y porque yo puedo comunicarme contigo
y con vosotros, es inteligente intentar la solución conjunta del problema
de la supervivencia.  Para que la comunicación sea útil debe eliminar
aquello que sea peculiar de cada individuo.  Denominamos
\definition{principio de intersubjetividad} a la eliminación de lo
peculiar en la comunicación.

Por razones análogas, la ciencia (véase la \S\refsc{ciencia}) también
debe acatar el principio de intersubjetividad.  Por ejemplo,
\person{Einstein} usó explícitamente el principio de intersubjetividad
en su teoría general de la relatividad, al señalar que las leyes físicas
deben tener la misma forma sin importar cómo se mueva el sistema de
referencia espaciotemporal respecto al que se efectúan las medidas.

\goodpage

\title3 Varias soluciones

El problema es que yo quiero vivir pero moriré.  El problema tiene varias
soluciones, o explicaciones, metafísicas.

\title4 Inmortalismo

La primera reacción contra un problema es negarlo.  No hay problema
cuando la condición, la realidad, se somete al deseo.  Yo quiero vivir y
yo no moriré.

El cristianismo y muchas otras religiones son inmortalistas.  Yo no
moriré.  El \definition{alma} no cambia al morir, lo que permite
convertir a la muerte, de cambio, en no cambio.

El \definition{inmortalismo} niega el problema al negar la muerte.

\title4 Neutralismo

También podría suceder que el problema fuese un problema, que el
problema no tuviera solución, pero que el problema pudiera ser
neutralizado.  Para neutralizar un problema basta desestimar su
valoración.  Es decir, la eliminación del deseo de vivir neutraliza el
problema.  Yo no quiero vivir y yo moriré.

El budismo es neutralista.  Culpa del sufrimiento y de la muerte al deseo.
El estoicismo también es neutralista.

Para el \definition{neutralismo} el problema queda neutralizado eliminando
el deseo.

\title4 Fatalismo

Podría ser que el problema no fuera, a pesar de parecerlo, un problema. Y
si ocurre que alguien lo tiene por problema, se puede conceder que es un
pseudoproblema, es decir, un problema ilusorio.  Parece que yo quiero vivir
y que yo moriré, pero esto es sólo una ilusión, es una manera de
hablar.

\label{materialismo}
Según el \definition{materialismo} solamente existe aquello que puede ser
descrito.  Por lo tanto no hay metafísica, y toda la filosofía se
reduce a la ciencia.  La muerte no es un problema, porque incluso al morir
se conserva la energía.  El libre albedrío es ilusión.

Para el \definition{fatalismo} no hay problema, ya que tampoco hay
libertad.

\title4 Transcendentalismo

La explicación transcendente utiliza el siguiente razonamiento:  yo no
puedo entender por qué me moriré porque, aunque mi muerte sí tiene
explicación, ocurre que la explicación de mi muerte queda fuera del
alcance de mi lógica.

Es difícil, o imposible, descubrir los límites de la lógica, porque ello ha
de hacerse desde dentro de la lógica, y la lógica es la totalidad de lo que
me es posible.  Si se sospecha, como lo hace el taoísmo, que las
limitaciones de la lógica son las causantes de nuestros males, y que el bien
está más allá de la lógica, entonces la paradoja, lo que no puede ser
sabido, se convierte en lo bueno.  Dios, que para los cristianos es la
explicación final, también aúna la bondad completa con la omnipotencia y la
justicia absoluta, lo cual resulta paradójico. Pero el cristianismo sólo
recurre a la paradoja para resolver problemas secundarios, por ejemplo para
explicar la razón del sufrimiento injusto, mientras que el problema
principal lo resuelve por el expediente de la inmortalidad.

Para el \definition{transcendentalismo} el problema tiene una solución
desconocida.

\title4 Escapismo

La más común de las soluciones consiste en ignorar o evitar el
problema.  (Yo quiero vivir pero moriré).

Nótese que el transcendentalismo es una forma razonada de escapismo.

Para el \definition{escapismo} el problema no debe ser planteado.

\title4 Subjetivismo

Para el subjetivismo hay conceptos anteriores a la explicación.  Estos
conceptos no pueden ser explicados.  Yo quiero vivir pero moriré.

Para el \definition{subjetivismo} el problema no tiene solución.

\title3 La explicación subjetivista

Yo soy subjetivista.  Mi explicación última, aquella explicación de
la que no doy explicación, es la explicación subjetivista.  La
explicación subjetivista establece que el problema es anterior a la
explicación, de manera que el problema primero no puede ser explicado.
Esto evita, definitivamente, que se pueda intentar explicar la
explicación, es decir, evita la regresión metafísica infinita.  A
cambio, no me preguntes por qué acepto la explicación subjetivista.  Te
lo repito, no explico la explicación subjetivista, la asumo.

Lo primero, el origen y la única fuente de significado, es el problema
aparente, yo quiero vivir pero moriré.  Si un problema aparente es
complejo, entonces la evolución puede discurrir hasta el sujeto, o sea,
puede toparse con un resolutor general organizado jerárquicamente en
forma de conocedor simbólico provisto de consciencia y con un yo.  Esto
tiene varias consecuencias.

El sujeto es un conocedor simbólico que se ve, a sí mismo, como su yo.
El yo es la incógnita del problema, el lugar de la libertad que debe
rellenar a voluntad.  De este modo, las explicaciones últimas, aquéllas
que alcanzan el yo, son libres.  El saber es libre.

La otra parte del problema es la realidad.  Es la parte del problema que
llamábamos condición.  La realidad es objetiva, o sea, ajena a la
libertad e independiente de la voluntad, y por esto decimos que se rige por
leyes inmutables.  El conocimiento es objetivo.

El sujeto es un conocedor simbólico cuyo yo vive en un mundo sintáctico. Así
que, aunque considere que su lógica es completa porque es el conjunto de
todo lo que puede imaginar (``nada ilógico {\em puede} ser pensado''),
también puede saber paradojas, esto es, expresiones sintácticas sin relación
posible con el problema aparente, expresiones sin significado.

\label{inteligencia}
La lógica del sujeto es contingente, un producto de la historia
evolutiva, y no necesaria.  No hay una lógica universal o razón.  Esto
significa que otra \definition{inteligencia}, esto es, otro resolutor
general tratando de resolver otro problema en otra lógica, no será
considerada inteligencia.

El sujeto es un resolutor general, un buscador compulsivo de soluciones, de
explicaciones.  Y el problema más urgente es el primero, eludir la muerte.
Desgraciadamente los problemas aparentes no tienen una solución
definitiva, siendo ésa precisamente la razón por la cual la evolución
puede llegar hasta el sujeto.  Luego el sujeto es una prueba, para el propio
sujeto, de que la muerte no puede ser explicada, aunque esa explicación
sería suficiente.

Y aunque el saber es libre y cualquier metafísica es posible, yo creo que
la explicación tiene límites.  La tentación de explicar la muerte es
fuerte y, por esto, la explicación absoluta tiene tanta urgencia.  Pero
yo soy incapaz de negar la libertad, o el deseo, o la muerte.


\title3 La muerte

``De lo que no se puede hablar hay que callar'',
concluyó \person{Wittgenstein}.  Es triste aceptar el consejo de
\person{Wittgenstein} y continuar elaborando explicaciones.  Porque supone
aceptar, con resignación, que mi imaginación está necesariamente
limitada a los objetos de la lógica en la que trabaja.  Porque supone
entender que las explicaciones no explican.  Que no se debe buscar ni una
explicación ni un sentido a la vida, porque es para vivir para lo que se
buscan las explicaciones.  Que saber es sólo un medio, que el fin es
sobrevivir.  Que la forma que yo le doy a la realidad está al servicio de
mi ansia compulsiva por sobrevivir.  Y es triste porque, a pesar de todo,
yo moriré.

\bigskip

Hasta he dado nombres a la muerte. Cada enfermedad es un nombre de la
muerte. Parece entonces que es la enfermedad quien me mata, pero no es así.
Simplemente moriré.


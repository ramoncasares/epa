% EPAI.TEX (RMCG19950910)

\catcode`\@=11
\prologskip=12pt plus12pt minus6pt


\prolog Lista de figuras (FIGURAS)

 Las figuras del texto principal y del anexo son, todas ellas, originales.
 Las figuras de la introducción han sido adaptadas mayormente de
 \person{Resnikoff}\cite{Resnikoff1989}, y también de
 \person{Gazzaniga}\cite{Gazzaniga1998} y de
 \person{Ernst}\cite{Ernst1978}.
 El número que aparece al final de cada entrada
 indica la página en la que está la figura. Así, por ejemplo,
 27 significa que la figura aparece en la página~27.

\def\tocline#1{\ifnum#1=0\let\next=\dopart\else
 \ifnum#1=9\let\next=\dofigure\else\let\next=\gobblefour\fi\fi
 \next}
\def\dofigure#1#2#3#4{\par
 \def\taketover##1|##2|{\def\1{##1}\def\2{##2}}\taketover#1|%
 \count255=#2\advance\count255by1
 \pdfcode \pdfdest num \number\count255 fitbh\pdfendcode
 \hang\noindent {\let\cr=\space\bf\1}%
 \setbox0=\hbox{\2}\ifdim\wd0=0pt \space\else{\bf:} \2.\fi
 \space\pdfgoto{{#4}}{#2}}
\def\dopart#1#2#3#4{\def\1{#1}\def\2{^^cdndices}\ifx\1\2\else
 \bigskip\leftline{\lmone#1}\nobreak\smallskip\fi}

\input auxiliar.toc


\prolog Bibliografía (BIBLIOGRAFÍA)

Se señalan las ediciones de las obras consultadas. Los números que aparecen
detrás del punto al final de cada reseña indican las páginas en las que la
obra está citada, así 128 significa que la cita está en la página~128.

\newwrite\bibnotes \immediate\openout\bibnotes=auxiliar.bdb
\iverb\bibnotes"% AUXILIAR.BDB"
\iverb\bibnotes"\record{Churchland1986}{BOOK}{"
\iverb\bibnotes" \TITLE{Neurophilosophy: Toward a Unified Science"
\iverb\bibnotes" of the Mind\discretionary{/}{}{/}Brain}}"
\iverb\bibnotes"\record{Turing1950}{INCOLLECTION}{\EDITOR{M.~Boden}}"
\iverb\bibnotes"\record{Searle1980}{INCOLLECTION}{\EDITOR{M.~Boden}}"
\iverb\bibnotes"\record{McCulloch1943}{INCOLLECTION}{\EDITOR{W.~McCulloch}}"
\iverb\bibnotes"\record{Turing1937}{INCOLLECTION}{\EDITOR{Ph.~Laplante}}"
\iverb\bibnotes"\record{Descartes1641}{BOOK}{\YEAR{1637, 41}}"
\iverb\bibnotes"\endinput"
\immediate\closeout\bibnotes

\def\ndxline#1#2{\ifnum#2=2 \let\oldk@y=\newk@y \def\newk@y{#1}%
 \let\next=\bibline \else \let\next\gobblethree \fi \next}

\def\bibbreak{\ifdim\lastskip<\smallskipamount \removelastskip
 \penalty-200 \vskip3pt plus 2pt \relax\fi} %plus 1.5pt minus 1pt \fi}

\def\bibline#1#2#3{\ifx\newk@y\oldk@y, \pdfgoto{#3}{#1}\else
 \bibbreak\cleantoks % \smallbreak\cleantoks
 \def\record##1##2##3{\def\secondk@y{##1}\ifx\newk@y\secondk@y##3\fi}
 \input epa.bdb
 \input auxiliar.bdb
 \hang\noindent\pdflabel{\bf \the\AUTHOR:}%
 {\def\em{\string\em\space}\def~{\string~}\def\it{\string\it\space}%
  \lbl{\newk@y}{\string\rm\space\expandafter\expandafter\the\AUTHOR\space
  \expandafter\expandafter(\the\YEAR):\space
  {\string\sl\space\expandafter\expandafter\the\TITLE}}}%
 {\sl\null{ }\the\TITLE}%
 \setbox0=\hbox{\the\YEAREDITION}\ifdim\wd0=0pt
  \YEAREDITION={\the\YEAR}\else\ (\the\YEAR)\fi
 \setbox0=\hbox{\the\EDITOR}\ifdim\wd0=0pt\else.
   Recopilado por \the\EDITOR\fi
 \setbox0=\hbox{\the\BOOKTITLE}\ifdim\wd0=0pt\else
   \ en ``\the\BOOKTITLE''\fi
 \setbox0=\hbox{\the\EDITION}\ifdim\wd0=0pt\else, \the\EDITION\fi
 \setbox0=\hbox{\the\PUBLISHER}\ifdim\wd0=0pt\else, \the\PUBLISHER\fi
 \setbox0=\hbox{\the\ADDRESS}\ifdim\wd0=0pt\else, \the\ADDRESS\fi
 \setbox0=\hbox{\the\JOURNAL}\ifdim\wd0=0pt\else, \the\JOURNAL\fi
 \setbox0=\hbox{\the\VOLUME}\ifdim\wd0=0pt\else, vol.~\the\VOLUME\fi
 \setbox0=\hbox{\the\MONTH}\ifdim\wd0=0pt
  \setbox0=\hbox{\the\YEAREDITION}\ifdim\wd0=0pt\else, \the\YEAREDITION\fi
  \else, \the\MONTH\ de \the\YEAREDITION\fi
 \setbox0=\hbox{\the\PAGES}\ifdim\wd0=0pt\else, pp.~\the\PAGES\fi
 \setbox0=\hbox{\the\ISBN}\ifdim\wd0=0pt\else, ISBN \the\ISBN\fi.
 \setbox0=\hbox{\the\NOTE}\ifdim\wd0=0pt\else \the\NOTE. \fi
 \pdfgoto{#3}{#1}\fi
 \ignorespaces}

\medskip

\input auxiliar.abc


\prolog Índice de definiciones (DEFINICIONES)

Se indican las páginas en las que aparecen las definiciones técnicas de los
conceptos listados.

\def\ndxline#1#2{\ifnum#2=0 \let\oldk@y=\newk@y \def\newk@y{#1}%
 \let\next=\defline \else \let\next\gobblethree \fi \next}

\def\defline#1#2#3{\ifx\newk@y\oldk@y, \pdfgoto{#3}{#1}\else
 \par\hang\noindent \newk@y,\nobreak\space\space
  \pdfgoto{#3}{#1}\fi\ignorespaces}

\begindoublecolumns
\parindent=12pt \rightskip=0pt plus 4pc \hyphenpenalty=250
\input auxiliar.abc
\enddoublecolumns


\prolog Índice de nombres (NOMBRES)

%Se indican las personas mencionadas y, en letra negrilla, la página en la
%que aparece la mención. Así, {\bf8} significa que la mención está en la
%página~{\bf8}.

\def\ndxline#1#2{\ifnum#2=1 \let\oldk@y=\newk@y \def\newk@y{#1}%
 \let\next=\ndxone \else \let\next\gobblethree \fi \next}

\def\ndxone#1#2#3{\let\oldp@ge=\newp@ge\def\newp@ge{#3}%
 \ifx\newk@y\oldk@y
  \ifx\newp@ge\oldp@ge \else, \pdfgoto{#3}{#1}\fi
 \else \par\noindent\hang{\sc\newk@y},\space\space
  \pdfgoto{#3}{#1}\fi\ignorespaces}

\begindoublecolumns
\parindent=12pt \rightskip=0pt plus 4pc \hyphenpenalty=250
\input auxiliar.abc
\enddoublecolumns

\catcode`\@=12


\tenpoint

\prolog Índice (ÍNDICE)

\def\tocline#1{\ifcase #1\let\next\toclinezero \or
 \let\next\toclineone \or \let\next\toclinetwo \or
 \let\next\toclinethree \or \let\next\toclinefour \else
 \let\next\gobblefour \fi \next}

\def\toclinezero#1#2#3#4{\bigskip\null
 \def\3{#3}\ifx\3\empty\def\1{#1}\else\def\1{#3. #1}\fi
 \centerline{\lmzero\pdfgoto{\1}{#2}}}
\def\toclineone#1#2#3#4{\medbreak
 \def\3{#3}\def\0{0}\def\Z{Z}%
 \ifx\3\0 \let\3=\empty \fi \ifx\3\Z \let\3=\empty \fi
 \ifx\3\empty \def\1{\vbox{\halign{\hfil####\cr#1\crcr}}\tocdotfill}%
  \else\def\1{#3\quad#1\hfill}\def\3{#3 }\fi
 \line{{\bf\1\space}\pdfgoto{#4}{#4}%
 \pdfcode\let\cr=\space%\stringactives
 \pdfoutline goto num #2 {\3#1}\pdfendcode}}
\def\toclinetwo#1#2#3#4{\line{\let\cr\space #3\quad #1\tocdotfill
 \space\pdfgoto{#4}{#2}}}
\def\toclinethree#1#2#3#4{\line{\let\cr\space \qquad #3\quad #1\tocdotfill
 \space\pdfgoto{#4}{#2}}}
\let\toclinefour\gobblefour

\def\tocdotfill{\leaders\hbox to \baselineskip{\bf\hss.\hss}\hfill}

\kern-2pc
\label{Índice}

\input auxiliar.toc

% \newpage \normalbottom

\null\newpage

\footline={\hfil} \headline={\hfil}
\loop \ifnum\pageno<207 \null\newpage \repeat


\null\vfill
{\bf El problema aparente} es una reelaboración en forma de ensayo de mi
prolongada tesis doctoral.  La tesis fue dirigida por el Profesor Fernando
Sáez Vacas, de la Universidad Politécnica de Madrid, y fue defendida el día
11 de mayo de 1993 ante un tribunal compuesto por los Profesores Gregorio
Fernández (Universidad Politécnica de Madrid), Miguel Ángel Quintanilla
(Universidad de Salamanca), José Cuena (Universidad Politécnica de Madrid),
Ángel Rivi\`ere (Universidad Autónoma de Madrid) y Carlos Delgado Kloos
(Universidad Politécnica de Madrid), que le concedieron por unanimidad la
máxima calificación, apto {\it cum laude}.

\null
{\bf El problema aparente} presenta una teoría del conocimiento modelada
matemáticamente, de manera que sus conceptos son precisos y sus
consecuencias verificables. Pretende, por tanto, sentar las bases de una
epistemología científica. La introducción, que intenta facilitar el
entendimiento de la teoría y la lectura del libro, incluye una sinopsis (a
partir de la página~{\refpg{Sinopsis}}) que resume el contenido.

\null
{\bf El problema aparente} sostiene i) que el simbolismo es un sistema
extensible de convenciones que sirve para resolver problemas porque permite
la expresión de problemas, soluciones y resoluciones, y ii) que el
simbolismo fue diseñado por la evolución darwiniana porque la vida es un
problema aparente y porque el simbolismo proporciona, de una sola vez, la
resolución de problemas, la consciencia, el lenguaje y la libertad. De aquí
se sigue que el simbolismo está vacío de significados y que son los
problemas los que aportan los significados. La teoría es subjetivista porque
es el sujeto simbólico quien da significado al objeto. Pero como saber es un
medio, el fin es vivir, y el sujeto individual es una parte de la vida,
resulta que de donde el individuo toma los significados últimos es de la
vida, por lo que todos los seres vivos disponen de una fuente común de
significados que hace posible la comunicación.

\null
{\bf El problema aparente} trata del saber, por lo que se sitúa en el cruce
de todos los saberes específicos y, en consecuencia, puede ser atacado desde
muy diferentes posiciones. Algunas de las disciplinas que se ven afectadas,
en mayor o menor grado, por esta epistemología de carácter subjetivista son:
 ética, filosofía, epistemología,
 lingüística, lógica, matemáticas,
 informática, inteligencia artificial,
 psicología, neurología y biología.

\break % Contraportada

\background
\pdfWhite
\input ean13

\null
\vsize=\vpage \advance\vsize-72bp
\vskip-2\baselineskip

\ptextfont
\parindent=0pt
\parskip=0pt plus1pt minus1pt
\baselineskip=13.3pt plus1pt minus1pt %14.4pt

{\pboldtextfont El problema aparente}
presenta una teoría del conocimiento modelada matemáticamente,
de modo que sus conceptos son precisos y
sus consecuencias son verificables.
Pretende, por consiguiente,
sentar las bases de una epistemología científica.
Responde, entre otras, a cuestiones tales como:
 \par
 ¿Hay diferencia entre adaptación y aprendizaje?\par
 ¿Qué maneras hay de resolver un problema?\par
 ¿Hay diferencia entre saber y conocimiento?\par
 ¿Cuál es el último reducto del absolutismo?\par
 ¿Hay diferencia entre solución y resolución?\par
 ¿Qué razón evolutiva tiene el simbolismo?\par
 ¿Se pueden pensar varias cosas a la vez?\par
 ¿Qué conceptos son transcendentes?\par
 ¿Cómo es posible la comunicación?\par
 ¿Se puede fabricar una persona?\par
 ¿Cuál es el problema aparente?\par
 ¿Para qué sirve la consciencia?\par
 ¿Es mi cuerpo parte de mi yo?\par
 ¿Son peligrosas las paradojas?\par
 ¿Sabe sumar una calculadora?\par
 ¿Es inteligente un ordenador?\par
 ¿Qué es un comportamiento?\par
 ¿Por qué sufrimos ilusiones?\par
 ¿Qué niega el materialismo?\par
 ¿Para qué sirve la sintaxis?\par
 ¿Puede ser feliz un perro?\par
 Esta frase es falsa, ¿lo es?\par
 ¿Para qué sirve la lógica?\par
 ¿Qué significa significa?\par
 ¿Qué es la abstracción?\par
 ¿Qué es un problema?\par
 ¿Soy yo una persona?\par
 ¿Qué es la felicidad?\par
 ¿Puedo yo ser libre?\par
 ¿Qué es la libertad?\par
 ¿Explica la ciencia?\par
 ¿Por qué morimos?\par
 ¿Piensa un perro?\par
 ¿Qué es un vaso?\par
 ¿Qué es la vida?\par
 ¿Qué soy yo?\kern4.4cm\vbox to 0pt{\vss\pdfBlack
  \hbox{%
   \pdfliteral{q}% save current graphic state in the stack
   \pdfliteral{0 0 0 0 k 0 0 0 1 K}% set black stroke on white fill
   \pdfliteral{-5 -5 m}% moves to the origin
   \pdfliteral{110 -5 l}%
   \pdfliteral{110 92 l}%
   \pdfliteral{-5 92 l}%
   \pdfliteral{-5 -5 l}% 
   \pdfliteral{b Q}% close, stroke, fill, and restore graphic state
   \barheight=2.4cm 
   \vbox to 0pt{\vss\hbox to 0pt{ISBN 84-7774-877-2\hss}\kern\barheight\kern1pc}%
   \EAN 978-84-7774-877-9
   \relax
 }\kern1pc}\par
\endinput

